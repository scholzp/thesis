% You can choose style "numeric" instead which is common in many papers.
% Without "maxbibnames=99" the bibliography entries only contain "First Name et al."
\usepackage[backend=biber,style=alphabetic,alldates=long,maxbibnames=99]{biblatex}

% FONT SETTINGS & ENCODING
% By default this build setup uses lualatex which supports special characters
% (öäüß<>) out of the box. If you ever want to switch to pdflatex but also
% keep the support for lualatex, add these three packages:
% \usepackage[T1]{fontenc}
% \usepackage[utf8]{luainputenc}
% \usepackage{lmodern}

\usepackage[nospace]{varioref}           % nice refs
\usepackage{csquotes}
\usepackage{graphicx}           % graphics
\usepackage{caption}            % manipulate fugures
\usepackage{subcaption}         % allow for subfigures
% Also checkout "minted" instead of "listings" - looks much nicer and supports
% more languages but requires "pygmentize" to be available on the command line
\usepackage{listings, listings-rust}           % nice source code listings
\usepackage{xcolor}
\usepackage{booktabs}           % nice tables
\usepackage{microtype}          % better looking text borders
\usepackage{makecell}
\usepackage{siunitx}            % unified way of setting values with units
\usepackage{array}
\usepackage{fancybox}           % provide nice boxes
\usepackage{fancyvrb}           % algorithm-boxes
\usepackage{pdfpages}
\usepackage{hyphenat}
\usepackage{todonotes}
\usepackage{xspace}
\usepackage[mode=buildnew]{standalone}
\usepackage{tikz}
\usetikzlibrary{calc, snakes}
\usetikzlibrary{decorations.pathreplacing, arrows.meta}
\usepackage{setspace}
\usepackage{fancyhdr}           % enables cool header line and footer line manipulations
\usepackage{lastpage}           % enables the usage of the label "LastPage" to get the
% number of pages with \pageref{LastPage}

% use this one last
% (redefines some macros for compatibility with KOMAScript)
\usepackage{scrhack}
