\chapter{Implementation}
\label{sec:implementation}

% Hier greift man einige wenige, interessante Gesichtspunkte der Implementierung
% heraus. Das Kapitel darf nicht mit Dokumentation oder gar Programmkommentaren
% verwechselt werden. Es kann vorkommen, daß sehr viele Gesichtspunkte
% aufgegriffen werden müssen, ist aber nicht sehr häufig. Zweck dieses Kapitels
% ist einerseits, glaubhaft zu machen, daß man es bei der Arbeit nicht mit einem
% "Papiertiger" sondern einem real existierenden System zu tun hat. Es ist
% sicherlich auch ein sehr wichtiger Text für jemanden, der die Arbeit später
% fortsetzt. Der dritte Gesichtspunkt dabei ist, einem Leser einen etwas
% tieferen Einblick in die Technik zu geben, mit der man sich hier beschäftigt.
% Schöne Bespiele sind "War Stories", also Dinge mit denen man besonders zu
% kämpfen hatte, oder eine konkrete, beispielhafte Verfeinerung einer der in
% Kapitel 3 vorgestellten Ideen. Auch hier gilt, mehr als 20 Seiten liest
% keiner, aber das ist hierbei nicht so schlimm, weil man die Lektüre ja einfach
% abbrechen kann, ohne den Faden zu verlieren. Vollständige Quellprogramme haben
% in einer Arbeit nichts zu suchen, auch nicht im Anhang, sondern gehören auf
% Rechner, auf denen man sie sich ansehen kann.

In the following sections, I will explain technical implementation details of
TEECore and how I implemented features described in chapter~\ref{sec:design}.
First, I will explain implementation highlights of TEECore itself, which is the
kernel used to populate the isolated partition in
section~\ref{sec:implementation:teeKernel}. To demonstrate the prototype, I use
Linux as kernel running in the normal partition. The prototype of TEECore
requires Linux to set it up and prepare memory for it. In
section~\ref{sec:implementation:hostKernel}, I describe how I configured Linux
to set up TEECore. To enable Linux to communicate with TEECore, I implemented a
driver as a Linux kernel module. I describe this kernel module in
section~\ref{sec:implementation:kmod}. My prototype implementation of TEECore
targets an Intel Core i7 13700k processor as evaluation platform.

\section{TEECore}
\label{sec:implementation:teeKernel}
TEECore is a fork of
PhipsBoot\footnote{\url{https://github.com/phip1611/phipsboot}}, a Multiboot 2
compliant kernel written in the memory-safe programming language Rust. A feature
of PhipsBoot is that it is relocatable and supports arbitrary load addresses. As
a Multiboot 2 compliant kernel, PhipsBoot requires that the CPU is in 32-bit
protected mode with segmentation enabled~\cite{mb2}. It therefore depends on
other software that brings the CPU to this state before transferring control.
After the handover, PhipsBoot does all initialization steps to bring the CPU
into 64-bit mode with enabled paging. This part of PhipsBoot is written in x86
assembly code. I will refer to these assembly routines as low-level code in the
following sections. It initializes the page tables for its own use and sets up
huge pages before transferring the CPU into 64-bit mode. PhipsBoot sets up
additional page tables for the software it loads. Because of its memory mapping,
PhipsBoot expects to be loaded to a 2 MiB-aligned address. Its small size, less
than 2 MiB, allows PhipsBoot to set up the page table hierarchy using a single
page table that maps its kernel. PhipsBoot's dynamic memory allocation is backed
by memory contained in its \gls{elf} image. Besides setting up the general run
time, PhipsBoot does not implement any additional features. After bringing the
platform to 64-bit \gls{g_lmode}, PhipsBoot starts executing its Rust routines.
The Rust implementation, which I refer to as high-level code, is not
feature-complete. The features implemented are checking for a valid multiboot
header, a working \gls{idt} setup, and a memory allocator. Because of its
minimal design and its ability to be relocated to arbitrary memory addresses, I
chose PhipsBoot as the base for TEECore. In the following sections, I will
explain chosen features I added to PhipsBoot to create TEECore.\\

\subsection{Mapping of Arbritary Memory}
\label{sec:implementation:teeKernel:mem}
PhipsBoot does not support mapping of arbitrary memory. Instead, the setup
assembly routine of PhipsBoot maps the entire memory of the loaded PhipsBoot
binary. The page tables are mapped in this process as read-only. Moreover, the
last-level page table is not exposed to the Rust high-level code. TEECore
requires this functionality to access external memory, such as shared memory
resources or device memory. To allow mapping of arbitrary memory, I modified the
page table setup routine and changed the last page table to be writable instead
of read-only. Furthermore, I exposed the linking symbol denoting the last-level
page table's address to the Rust high-level code to propagate its address.\\

\begin{lstlisting}[language=Rust, caption=Mapping of external memory, label=code:map_external]
pub unsafe fn map_page_frame(phys_addr: u64) -> u64 {
    // LAST_INDEX initialized with 384
    // PAGE_TABLE_ADDR = address of last-level page table
    let pt_entry_ptr = PAGE_TABLE_ADDR.add(LAST_INDEX);
    let mut virt_addr: u64 = 0x0;
    // Create virtual address
    // Mask the 21 lowest bits
    let upper_bits = (PAGE_TABLE_ADDR as u64) & (!0x1FFFFF);
    // Set bits 21 to 9, set present bit (0x1)
    virt_addr = upper_bits + (LAST_INDEX << 12) | 0x1;
    // Write physical address to page table
    ptr::write(pt_entry_ptr, (Into::<u64>::into(phys_addr)));
    LAST_L1_INDEX += 1;
    return virt_addr;
}
\end{lstlisting}

The mapping routine needs to find the right place to map the memory. PhipsBoot
uses huge pages to map its assembly low-level code. Its Rust high-level code
uses 4 KiB mappings, but inherits the alignment. Because of this, it expects to
be loaded to a 2 MiB-aligned address. TEECore inherits these characteristics. To
accommodate the strict memory restriction TEECore faces, I decided to use a
single last-level page table. The alignment to 2 MiB allows the TEECore binary
to grow to up to 2 MiB in size without using more than a single last-level page
table. Alternatively, TEECore can use free page table entries not filled with
addresses of its binaries to map arbitrary memory. This is possible because
TEECore does not use the whole 2 MiB of memory. Instead, it uses the first 128
entries, which leaves the last 384 entries unused. This reflects the observed
binary size of around 300 KiB. For dynamic mapping of memory that is not
contained in the binary. I reserve the last 128 entries of the last-level page
table. These 128 page table entries allow TEECore to map up to 512 KiB of
arbitrary memory. This leaves TEECore room to grow to a size of 1536 KiB.
Overall, the total maximal size of 2 MiB can fit
in most L2 caches of Intel x86\_64 performance cores produced since
2022~\cite{intel_optimization}. The L2 cache is a core-exclusive resource in
these processors.\\

For mapping arbitrary memory, TEECore uses addresses derived from the virtual
address of the last-level page table. The uppermost 43 bits are the same as in
the last-level page tables's virtual address. Bits 21 to 9 are the index into the last
level page table, which is 384 for the first page to map and increments by one
for every additional mapped page. At this index, the page frame address is
stored in the last level of the page table. The last 9 bits are the index into
the page frame. Listing~\ref{code:map_external} shows the pseudocode of the
mapping algorithm explained.\\

Mapping arbitrary memory is important in implementing the shared memory
communication path. To obtain information about the address and layout of the
shared memory used for communication, the bootloader provides a memory map to
TEECore. This memory map is contained in the \gls{mbi} structure. The physical
address of the structure is loaded to the \textit{EBX} register by the
bootloader before it transfers control to TEECore. In the prototype
implementation, the Linux kernel module fills the role of the bootloader (c.f.
\ref{sec:implementation:kmod}). Other than the memory owned by TEECore, the
memory intended to be used by TEECore and Linux is mapped as uncacheable. This
mapping prevents cache pollution and false positives for allowed access of a
remote core to memory also used by TEECore.

\subsection{Attack Detection Mechanism}
\label{sec:implementation:teeKernel:pmcs}
To prepare the environment, TEECore needs to ensure that all memory used by it
is in the exclusive or modified state so that a remote core must trigger a
measurable event when it tries to access memory owned by TEECore. TEECore
achieves this by running a function that iterates over the page table entries.
For each entry that is marked present, all bytes of the mapped memory are
accessed. This function reads a given address and writes back the same value to
the same address to trigger the modified state of the cache line. This state
change leads to invalidating a remote core's cache lines referencing the same
memory item over the cache coherency protocol. As soon as a remote core accesses
the same memory, the cache line states in TEECore's CPU core changes. TEECore can
thus measure changes in the microarchitectural state of a cache line by keeping
them in an exclusive or modified state. TEECore can only in this way, for
example, detect a passive attacker that performs reading access from a remote
core to TEECore's memory. This is because the reading of cache lines, which are
in the shared state in the remote and target caches, do not trigger any
microarchitectural responses.\\

The preparation as mentioned earlier ensures that all access from remote cores
to TEECore's memory results in additional microarchitectural state changes,
namely cache line state changes. TEECore configures the hardware so that it can
measure these changes with the help of \glspl{pmc}.
% I first tried to use these offcore response events to measure the
% communication of the core executing TEECore with the remaining system. While I
% could see events working to count communication to the memory mapped as
% uncacheable, I could not measure any communication from and to the L3 caches
% of my test CPU. In contrast, after extensive testing, I found that some of the
% core's local events measured the activity of the L3 cache. This might be
% because... \todo{Add a plausible cause}.
Table~\ref{40:tab:events} lists the events that TEECore uses to implement its
attack detection mechanism. Requests issued by hardware prefetchers might not be
measured by performance monitoring counters~\cite{perfmon}. TEECore, therefore,
disables hardware prefetchers. Table~\ref{40:tab:events} contains information
about performance monitoring events as found at Intel's database for Raptor Lake
processors~\cite{perfmon}. It is important to note that these events are
specific to load instructions. This means that TEECore has to execute a load
to the respective address before it can detect cache line changes.

\begin{table}[!h]
  \centering
  \begin{tabular}{ |p{6.5cm}|p{1.35cm}|p{1.25cm}|p{3.5cm}| }
    \hline
    Event Name                  & Selector & UMask & Description                                                                            \\
    \hline
    MEM\_LOAD\_RETIRED.L2\_MISS & 0xD1     & 0x10  & Counts retired load instructions that missed in the L2 cache.                          \\
    MEM\_LOAD\_RETIRED.L3\_HIT  & 0xD1     & 0x04  & Counts retired load instructions with at least one \mu op that hit in the L3 cache.    \\
    MEM\_LOAD\_RETIRED.L3\_MISS & 0xD1     & 0x20  & Counts retired load instructions with at least one \mu op that missed in the L3 cache. \\
    \hline
  \end{tabular}
  \caption{Performance Counter Events used by TEECore}
  \label{40:tab:events}
\end{table}

The performance monitoring facilities of x86 CPUs I use, consist of a pair of
two \glspl{msr}, one \textit{IA32\_PERFEVTSELx} and \textit{IA32\_PMCx} each,
where $x$ is the index of the pair. The Intel Core i7 13700k implements eight
pairs of these \glspl{msr}. A given \textit{IA32\_PERFEVTSELx} \gls{msr}
contains the configuration of the performance counter. This includes the event
to measure and information about interrupt generation, as well as de-/activation
bits of the counter. Figure~\ref{fig:state:technical:perfsel} shows the layout
of \textit{IA32\_PERFEVTSELx}. \glspl{msr} of type \textit{IA32\_PMCx} \gls{msr}
contain the measurement values. Although all \textit{IA32\_PMCx} \glspl{msr}
have a size of 64 bits, fewer bits might be implemented in hardware. For
example, my test CPU implements 48 bits for \textit{IA32\_PMCx}.

\begin{center}
  \begin{figure}
    \centering
    \includestandalone{images/perfsel_msr.tex}
    \caption{Layout of IA32\_PERFEVTSELx MSRs}
    \label{fig:state:technical:perfsel}
  \end{figure}
\end{center}

To set up a \gls{pmc} for a specific event, TEECore sets the \textit{Events
Select} and \textit{UMASK} field in the respective \textit{IA32\_PERFEVTSELx}.
Moreover, TEECore set the \textit{US} and \textit{OS} bits to instruct hardware
to count events for user mode and privileged software. TEECore furthermore sets
the \textit{INT} bit to enable interrupt delivery on counter overflow. Because
the \gls{lapic} delivers the interrupt on an overflow of the corresponding
\textit{IA32\_PMCx}, TEECore has to additionally set all bits in this \gls{msr},
to ensures that the counter overflows on the first occurrence of the respective
event and triggers the delivery of a \gls{pmi}. After setting up
\textit{IA32\_PMCx}, TEECore enables the counter by setting the \textit{EN} bit
in the respective \textit{IA32\_PERFEVTSELx} \gls{msr}. From this point on,
writes to \textit{IA32\_PMCx} are forbidden until the counter is disabled.\\

To complete the setup, TEECore needs to configure the \gls{idt} accordingly, by
writing zero to the \gls{idtr}. Once the \gls{lapic} delivers an interrupt to
TEECore, this configuration leads to a triple fault, effectively resetting the
whole platform. This configuration allows for protection against single-step
attacks that use interrupts. Moreover, as described in
section~\ref{sec:30:tee_kernel}, TEECore uses \glspl{pmi} to make hardware
signal to it that it has been attacked. These kinds of interrupts are routed
through the \gls{idt} too and result in a triple fault in the same process as
any other interrupt.\\

TEECore must set up the \gls{lapic} to activate the delivery of \glspl{pmi}. It
does so by mapping the \gls{lapic} register space as described in
section~\ref{sec:implementation:teeKernel:mem} to a virtual address. At this
point, the \gls{lapic} is software disabled. While it responds to special
interrupts, such as \glspl{nmi} and INIT \glspl{ipi}, it does not forward
\glspl{pmi}. TEECore writes to the \textit{APIC Software Enable/Disable} bit in
the \textit{spurious-interrupt vector} register to software enable the
\gls{lapic}. Furthermore, TEECore configures the Performance Monitoring Register
of the \gls{lvt} to deliver \glspl{pmi} as \glspl{nmi}. To complete the setup of
the \gls{lapic}, TEECore clears the mask bit in the Performance Monitoring
Register of the \gls{lvt}. After this, the \gls{lapic} is ready to deliver
\glspl{pmi} to TEECore.

\subsection{Shared Memory Communication Path}
\label{sec:implementation:teeKernel:shared}
I implemented a structure to manage the shared memory channel. Access to the
shared memory follows a simple protocol that splits the memory, as shown in
figure~\ref{fig:impl:shared_mem_layout}. The first byte denotes the sender, the
second denotes the task identifier, and the remaining part is used for a payload
that can be sent along with the other tags. I assume there are only two
communication parties in the current prototype: TEECore and the
host kernel. A message is sent by writing the first byte. To receive a message,
the respective party polls the first byte of the shared memory and waits until
the signal is written by the other party. In the current prototype only one
communcation buffer for both directions exist. In the prototype, both parties
first write the payload bytes before writing the first byte of the buffer and
only do so, as long as the first bytes indicates a message from the other party.
As a side note, this can lead to race conditions between TEECore and the other
communication party, if both don't adhere to the same protocol. These race
conditions are not of concern for my prototype implementations but should be
mitigated if TEECore is employed in real world scenarios.\\

\begin{center}
  \begin{figure}[]
    \centering
    \includestandalone{images/shared_mem_layout.tex}
    \caption{Illustration of the Layout of the Shared Memory used For Communication}
    \label{fig:impl:shared_mem_layout}
  \end{figure}
\end{center}

After initialization, TEECore starts a state machine that
controls the execution of tasks. In the first state, TEECore polls the
shared memory until it receives a message containing the command to execute one
of the predefined tasks. Once this message is received, TEECore executes
the task as commanded. While the task runs, it can access the shared memory for
multiple reasons. The first reason is that the task can access the payload
section of the memory to receive additional input data from the party outside of
TEECore. The second reason is that the task can prepare an answer to the
outside party's request. For this, the task can write to fields dedicated to the
sender information, the command, and the payload. It is the task's
responsibility to write to the shared memory data needed to create a response to
the request once it is done. Currently, TEECore does not do any integrity checks
of the input. Before employing TEECore in real world scenarios such checks
should be implemented.

\section{Normal Partition Kernel: Linux}
\label{sec:implementation:hostKernel}
As a highly configurable open-source general-purpose operating system, Linux is
an example of the implementation of the normal partition's kernel. The Linux
kernel I used for my prototype is version 6.13. Linux allows its configuration
by passing command line arguments to the kernel on boot. This makes it able to
limit the resources Linux uses. \\

To isolate the isolated partition's CPU core, I use the command line option
\textit{nr\_cpus=n}, where $n$ is an arbitrary number that limits Linux to using
a maximum core count of $n$. I use $n=3$ to limit Linux to 3 cores. While Linux
supports CPU core
hotplugging\footnote{\url{https://docs.kernel.org/core-api/cpu_hotplug.html}},
which would allow cores to be turned off through the same feature,
\textit{nr\_cpu} introduces a hard limit. This limit originates from the fact
that management data structures for late CPU hotplugging need to be
preallocated. The \textit{nr\_cpu} limits the number of preallocated structures,
thus, cores excluded from initialization by this parameter are inaccessible for
Linux and cannot be hotplugged later~\cite{kernel-parameters}. \\

Another useful parameter is \textit{memmap}, which I use to add custom entries
to the BIOS memory map that Linux uses to set up its memory map. With this, I
add an entry with type \textit{persistent} of 4KiB size starting at the address
$0x9000$, which is within the first MiB of system memory and free for use. I
reserve this memory in order to use it later for booting one of the excluded
CPU cores, as x86 CPUs initialize in 16-bit mode and, therefore, can only access
memory with addresses lower than the 1 MiB boundary. Linux does not offer any
procedure to kernel modules to allocate memory in this address region, so this
is the only possible way.\\

The driver for TEECore, implemented as a Linux kernel module, performs
additional initialization steps that I explain in
section~\ref{sec:implementation:kmod}. Linux is the first software to be run in
my prototype setup. The initial RAM disk image embeds all other components. It
contains TEECore as an \gls{elf} image, the Linux kernel module, and the
\gls{elf} image of a user space application that allows for communication with
TEECore. This permits the creation of a unified kernel image that can be signed
for use with secure boot.\\

\section{Linux kernel module}
\label{sec:implementation:kmod}
As I described in section~\ref{sec:implementation:hostKernel}, the Linux kernel
module is the driver not only enabling Linux to communicate with TEECore on the
isolated core but also the component that initializes the isolated core and
loads TEECore into memory.\\

For this, the kernel module claims the memory reserved through the Linux command
line argument and copies the startup code to the reserved memory. I wrote the
startup code in assembly. It transfers the processor from real mode into the
32-bit mode with segmentation and places the addresses of the \gls{mbi}
structure into the EBX register. The resulting state is the one expected by
TEECore. The startup code is compiled before the kernel module and included as
an array of bytes. This allows for easier modification, as otherwise, the need
to edit a binary would arise. The \gls{mbi}, which TEECore expects, is created
by the kernel module at run time. Because of this, the address of the \gls{mbi}
is not known at compile time. After allocating the memory, the kernel module
modifies the startup code at runtime by replacing a placeholder with the actual
address of the \gls{mbi}.\\

In the next step, the kernel module locates the \gls{elf} binary of TEECore
and places it in memory. For this, the module allocates memory through the Linux
kernel's \textit{allocpages} interface at a physical address aligned to 2 MiB.
Once allocated, the kernel module does not deallocate the pages to prevent Linux
from reusing the memory and possibly overwriting TEECore. Once the required
memory is allocated and mapped in the kernel address space, the kernel module
begins to parse TEECore's \gls{elf} image and copies all necessary
parts to memory. The kernel module brings its own \gls{elf} loader for this. The
kernel module does not use the Linux' \gls{elf} file loader because it does not
allow me to obtain the physical address to which the \gls{elf} image is
loaded. The kernel module requires the TEECore binary's physical address to
calculate the TEECore code entry address. This address is a second important
address generated at runtime of the kernel module with which the boot code is
updated. The kernel module then sets up the shared memory region by allocating
the amount of memory specified before. It then creates a \gls{mbi} structure
containing the physical address of the shared region with a custom-defined
memory type. TEECore later parses the \gls{mbi} structure to find the physical
memory backing the shared memory communication channel.\\

Once all parts necessary to run TEECore are placed in memory, the kernel module
prepares to start the isolated core. At this point, Linux does not allow
interaction with any core not initialized by it, i.e., the kernel module cannot
address the isolated core with any kernel-provided functions. To circumvent this
shortcoming, I gained access to the \gls{lapic} of the core running the kernel
module to send the required INIT and startup \gls{ipi} sequence to the isolated
core. Another problem arises when implementing this routine: APIC IDs are not
bound to start at 0, nor do they have to increment by one for each CPU
core~\cite{intel_sdm}. To overcome this issue, I queried the APIC IDs of the
cores mapped by Linux and calculated the offset for each core. I used this
offset to calculate the APIC ID of the target core by adding the offset to the
ID of the last core mapped by Linux. After sending the \gls{ipi} sequence to the
remote core, the startup code will be executed, and a far jump to TEECore's code
will be performed by the secure partition's CPU core after runnint the startup
code.\\

After initializing the secure partition's CPU core, the kernel module manages
communication with TEECore. It, therefore, implements the protocol described in
\ref{sec:implementation:teeKernel}. To receive messages, the kernel module
installs a timer, which it uses to poll the shared memory periodically for new
messages. The kernel module also implements the remote part of tasks in the
prototype. To start a specific task, the kernel module sends a message
containing the task's ID and evaluates messages from TEECore to delegate them to
the respective task's handling routine. Lastly, the kernel module creates a
character device to allow user space applications to use TEECore. Writes and
reads to this device are implemented by the kernel module, which translates the
user requests into messages for TEECore.

%%% Local Variables: %% TeX-master: "diplom" %% End:
