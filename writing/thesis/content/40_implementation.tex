\chapter{Implementation}
\label{sec:implementation}

% Hier greift man einige wenige, interessante Gesichtspunkte der Implementierung
% heraus. Das Kapitel darf nicht mit Dokumentation oder gar Programmkommentaren
% verwechselt werden. Es kann vorkommen, daß sehr viele Gesichtspunkte
% aufgegriffen werden müssen, ist aber nicht sehr häufig. Zweck dieses Kapitels
% ist einerseits, glaubhaft zu machen, daß man es bei der Arbeit nicht mit einem
% "Papiertiger" sondern einem real existierenden System zu tun hat. Es ist
% sicherlich auch ein sehr wichtiger Text für jemanden, der die Arbeit später
% fortsetzt. Der dritte Gesichtspunkt dabei ist, einem Leser einen etwas
% tieferen Einblick in die Technik zu geben, mit der man sich hier beschäftigt.
% Schöne Bespiele sind "War Stories", also Dinge mit denen man besonders zu
% kämpfen hatte, oder eine konkrete, beispielhafte Verfeinerung einer der in
% Kapitel 3 vorgestellten Ideen. Auch hier gilt, mehr als 20 Seiten liest
% keiner, aber das ist hierbei nicht so schlimm, weil man die Lektüre ja einfach
% abbrechen kann, ohne den Faden zu verlieren. Vollständige Quellprogramme haben
% in einer Arbeit nichts zu suchen, auch nicht im Anhang, sondern gehören auf
% Rechner, auf denen man sie sich ansehen kann.

In the following sections, I will explain technical implementation details of
TEECore and how I implemented features described in chapter~\ref{sec:design}.
First, I will explain implementation highlights of TEECore itself, which is the
kernel used to populate the isolated partition in
section~\ref{sec:implementation:teeKernel}. To demonstrate the prototype, I use
Linux as kernel running in the normal partition. The prototype of TEECore
requires Linux to set it up and prepare memory for it. In
section~\ref{sec:implementation:hostKernel}, I describe how I configured Linux
to set uo TEECore. To enable Linux to communicate with TEECore finally, I
implemented a driver as a Linux kernel module. I describe this kernel in
section~\ref{sec:implementation:kmod}.\todo{Something something trusted
applications} As a target for the prototype implementation of TEECore I use an
Intel Core i7 13700k processor.

\section{TEECore}
\label{sec:implementation:teeKernel}
TEECore is a fork of
PhipsBoot\footnote{\url{https://github.com/phip1611/phipsboot}}. PhipsBoot is a
Multiboot
2\footnote{\url{https://www.gnu.org/software/grub/manual/multiboot2/multiboot.html}}
compliant kernel written in the memory-safe programming language Rust, to serve
as a bootloader for 64-bit x86 kernels. A feature of PhipsBoot is that it is
relocatable and supports arbitrary load addresses. It depends on a first-stage
bootloader\todo{maybe this is not a multi-stage bootloader} that brings the
system to a state defined in the Multiboot 2 specification. That is, the CPU is
in 32-bit protected mode with segmentation enabled~\cite{mb2}. After handover,
PhipsBoot does all initialization steps to bring the CPU into 64-bit mode with
enabled PEA paging. This part of Phipsboot is written in x86 assembly code. It
initializes the page tables and sets up huge pages before transferring the CPU
into 64-bit mode. For later usage, Phipsboot sets up additional page tables for
execution. Because of its memory mapping, PhipsBoot expects to be loaded to a 2
MiB-aligned address. Its small size, less than 2 MiB, allows PhipsBoot to set up
the page table hierarchy using a single page table that maps the kernel.
PhipsBoot uses memory statically reserved for dynamic memory allocation in the
kernel's Elf image. Besides setting up the general run time, PhipsBoot does not
implement more features but offers a great starting point for the implementation
of TEECore.\\

The first feature I introduced was the ability to modify page tables because
Phipsboot did not support mapping arbitrary memory. Instead, the setup assembly
routine of Phipsboot maps the entire memory of the loaded binary. The page
tables are mapped in this process as read-only. Moreover, the Rust high-level
code does not know the address of the last level page table. To allow mapping of
arbitrary memory, I modified the page table setup routine and changed the last
page table to be writable instead of read-only, and made it accessible in the
Rust high-level code by propagating its address. Note that TEECore is aligned to
a 2 MiB boundary because of the initial huge page mapping. Therefore, the
mappping of TEECore addresses starts in the first entries of the last level page
table until index $n$. Through observation, I found that no more than the first
128 entries in the page table are used to map all memory used by the kernel
binary. The number of initially mapped pages reflects that TEECore's ELF image
is around 300 KiB in size. For dynamic mapping of memory that TEECore's binary
does not contain, I, therefore, reserved the last 128 entries of the last level
page table. These 128 page table entries enable  TEECore to map up to 512 KiB of
arbitrary memory that was not contained in the binary while allowing for the
binary to grow to a size of 1536 KiB. Overall, the total maximal size of 2 MiB
can fit in most L2 caches of Intel x86\_64 performance cores produced since
2022. The L2 cache is a core exclusive resource in these processors.\\

For mapping arbitrary memory, TEECore uses addresses derived from the virtual
address of the last level page table. The uppermost 43 bits are the same as in
the last level page's virtual address. Bits 21 to 9 are the index into the last
level page table, which is 384 for the first page to map and increments by one
for every additional mapped page. At this index, the page frame address is
stored in the last level of the page table. The last 9 bits are the index into
the page frame. Listing~\ref{code:map_external} shows the pseudocode of the
mapping algorithm explained.

\begin{lstlisting}[language=Rust, caption=Mapping of external memory, label=code:map_external]
pub unsafe fn map_page_frame(phys_addr: u64) -> u64 {
    // LAST_INDEX initialized with 384
    // PAGE_TABLE_ADDR = address of last level page table
    let pt_entry_ptr = PAGE_TABLE_ADDR.add(LAST_INDEX);
    let mut virt_addr: u64 = 0x0;
    // Create virtual address
    // Mask the 21 lowest bits
    let upper_bits = (PAGE_TABLE_ADDR as u64) & (!0x1FFFFF);
    // Set bits 21 to 9, set present bit (0x1)
    virt_addr = upper_bits + (LAST_INDEX << 12) | 0x1;
    // Write physical address to page table
    ptr::write(pt_entry_ptr, (Into::<u64>::into(phys_addr)));
    LAST_L1_INDEX += 1;
    return virt_addr;
}
\end{lstlisting}

Mapping arbitrary memory is important in implementing the shared memory
communication path. To obtain information about the address and layout of the
shared memory used for communication, the bootloader provides a memory map to
TEECore. This memory map is contained in the \gls{mbi} structure. The physical
address of the structure is loaded to the \textit{EBX} register by the
bootloader before it transfers control to TEECore. In the prototype
implementation, the Linux kernel module fills the role of the bootloader (c.f.
\ref{sec:implementation:kmod}). Other than the memory owned by the kernel, the
memory intended to be used by TEECore and Linux is mapped as uncacheable. This
mapping prevents cache pollution and false positives for allowed access of a
remote core to memory also used by the \gls{tee}.\\

To prepare the environment, TEECore needs to ensure that all memory used by it
is in the exclusive or modified state so that a remote core must trigger a
measurable event when it tries to access memory owned by TEECore. For this, I
implemented a function that iterates over the page table entries and accesses
all bytes of all pages that are set present. This function reads a given address
and writes back the same value to the same address to trigger the modified state
of the cache line. This state change leads to invalidating a remote core's cache
lines referencing the same memory item. I discovered this necessity when
implementing my attack simulations (c.f.~\ref{eval:sec}). While TEECore could
detect the active attacker, the passive one was undetectable because the reading
of cache lines in the shared state in both caches does not trigger any
microarchitectural responses. It is, therefore, necessary to hold all memory
that TEECore should keep secret in the exclusive or modified state to detect
cache synchronization over the cache coherency protocol.\\

The aforementioned preparation ensures, that all access from remote cores to
TEECore's memory result in additional microarchitectural state changes, namely
cache line state changes. In the next step, TEECore configures the hardware so
that it can measure these changes through the help of \glspl{pmc}. x86\_64
processors manufactured by Intel and AMD offer special events dedicated to
monitoring the behavior of CPU components external to the processor core. Such
components, for example, include main memory, shared caches, and other shared
resources. In Intel processors, such components are referred to as uncore. Intel
processors implement so-called offcore-response events to measure the events
resulting from the uncore responding to the processor core's requests. I first
tried to use these offcore response events to measure the communication of the
core executing TEECore with the remaining system. While I could see events
working to count communication to the memory mapped as uncacheable, I could not
measure any communication from and to the L3 caches of my test CPU. In contrast,
after extensive testing, I found that some of the core's local events measured
the activity of the L3 cache. This might be because... \todo{Add a plausible
cause}. As a result, I decided to use the events listed in
table~\ref{40:tab:events} to implement TEECore's attack detection mechanism.
Table~\ref{40:tab:events} contains information about performance monitoring
events as found at Intels
database\footnote{\url{https://perfmon-events.intel.com/}} for Raptor Lake
processors.

\begin{table}[!h]
  \centering
  \begin{tabular}{ |p{6.5cm}|p{1.35cm}|p{1.25cm}|p{3.5cm}| }
    \hline
    Event Name                  & Selector & UMask & Description                                                                            \\
    \hline
    MEM\_LOAD\_RETIRED.L2\_MISS & 0xD1     & 0x10  & Counts retired load instructions missed L2 cache as data sources.                     \\
    MEM\_LOAD\_RETIRED.L3\_HIT  & 0xD1     & 0x04  & Counts retired load instructions with at least one \mu op that hit in the L3 cache.    \\
    MEM\_LOAD\_RETIRED.L3\_MISS & 0xD1     & 0x20  & Counts retired load instructions with at least one \mu op that missed in the L3 cache. \\
    \hline
  \end{tabular}
  \caption{Performance Counter Events used by TEECore}
  \label{40:tab:events}
\end{table}

To enable counting of the chosen events, I programmed the respective \gls{pmc}
\glspl{msr}. The performance monitoring facilities of x86 CPUs I use consist
of a pair of two \glspl{msr}, one \textit{IA32\_PERFEVTSELx} and
\textit{IA32\_PMCx} each, where $x$ is the index of the pair. A physical
processor supports at least four of those pairs. The \textit{IA32\_PERFEVTSELx}
\gls{msr} contains the configuration of the performance counter, including the
event to measure and information about interrupt generation, as well as
de-/activation bits of the counter, while \textit{IA32\_PMCx} \glspl{msr}
contain the measurement value. Although all \textit{IA32\_PMCx} \glspl{msr} have
a size of 64 bits and fewer bits might be implemented in hardware. For example,
my test CPU implements 48 bits for \textit{IA32\_PMCx}. The Layout of
\textit{IA32\_PERFEVTSELx} \glspl{msr} is shown by
figure~\ref{fig:state:technical:perfsel}.

\begin{center}
  \begin{figure}
    \centering
    \includestandalone{images/perfsel_msr.tex}
    \caption{Layout of IA32\_PERFEVTSELx MSRs}
    \label{fig:state:technical:perfsel}
  \end{figure}
\end{center}

To set up \gls{pmc} for a specific event, TEECore sets the \textit{Events
Select} and \textit{UMASK} field in the respective \textit{IA32\_PERFEVTSELx}
\textit{OS} bits to instruct hardware to count events for user mode and
privileged software. TEECore furthermore sets the \textit{INT} bit to enable
interrupt delivery on counter overflow. Because the \gls{lapic} delivers the
interrupt through the \gls{lapic} on an overflow of the corresponding
\textit{IA32\_PMCx}, TEECore has to additionally write all ones to this
\gls{msr}. This write ensures that the counter overflows on the first
occurrence of the respective event and triggers the delivery of a \gls{pmi}.
After setting up \textit{IA32\_PMCx}, TEECore enables the counter by setting
the \textit{EN} bit in the respective \textit{IA32\_PERFEVTSELx} \gls{msr}
after finishing its preparation phase. From this point on, writes to
\textit{IA32\_PMCx} are forbidden.

One of the first actions by TEECore is to set up the \gls{idt} accordingly. In
doing so, it writes zero to the \gls{idtr}, which leads to triple faulting once
the \gls{lapic} deliveres an interrupt to TEECore. Because I designed TEECore to
work without interrupts, every interrupt TEECore receives is considered an
attack. From then on, TEECore cannot be interrupted and thus is not prone to
single-step attacks. In a second step, TEECore configures the \gls{lapic} to
enable the delivery of \gls{pmi}. As described in
section~\ref{sec:30:tee_kernel}, TEECore uses \glspl{pmi} to signal itself that
it has been attacked. First, it uses the aforementioned functionality to map the
\gls{lapic} register space. \gls{lapic} registers are memory-mapped
CPU-exclusive registers with 32-bit widths. Relevant to my use case is the
\textit{spurious-interrupt vector} register and the \textit{\gls{lvt}}. The
spurious-interrupt vector register is important for software activating the
\gls{lapic} because, after reset, the \gls{lapic} might not be software enabled.
However, it responds to \gls{ipi} messages. To activate the \gls{lapic}, TEECore
writes to the \textit{APIC Software Enable/Disable} bit. After enabling the
\gls{lapic}, TEECore has to configure it accordingly to activate delivery of
\glspl{pmi}. For this, TEECore configures the Performance Monitoring Register of
the \gls{lvt} to deliver \glspl{pmi} as \glspl{nmi}. As a last step, TEECore
clears the mask bit in the Performance Monitoring Register of the \gls{lvt}.
After this, the \gls{lapic} is ready to deliver \glspl{pmi} to TEECore.\\

\begin{center}
  \begin{figure}
    \centering
    \includestandalone{images/shared_mem_layout.tex}
    \caption{Illustration of the Layout of the Shared Memory used For Communication}
    \label{fig:impl:shared_mem_layout}
  \end{figure}
\end{center}

I implemented a structure to manage the memory in question to enable
communication through the shared memory channel. Access to the shared memory
follows a simple protocol that splits the memory, as shown in
figure~\ref{fig:impl:shared_mem_layout}. The first byte denotes the sender, the
second denotes the task identifier, and the remaining part is used for a payload
that can be sent along with the other tags. I assume there are only two
communication parties in the current prototype: the \gls{tee} kernel and the
host kernel. A message is sent by writing the first byte. To receive a message,
the respective party polls the first byte of the shared memory and waits until
the signal is written by the other party. In the current prototype only one
communcation buffer for both directions exist. In the prototype, both parties
first write the payload bytes before writing the first byte of the buffer and
only do so, as long as the first bytes indicates a message from the other party.
As a side note, this can lead to race conditions between TEECore and the other
communication party, if both don't adhere to the same protocol. These race
conditions are not of concern for my prototype implementations but should be
mitigated if TEECore is employed in real world scenarios.\\

After initialization, the \gls{tee} environment starts a state machine that
controls the execution of tasks. In the first state, the \gls{tee} polls the
shared memory until it receives a message containing the command to execute one
of the predefined tasks. Once this message is received, the \gls{tee} executes
the task as commanded. While the task runs, it can access the shared memory for
multiple reasons. The first reason is that the task can access the payload
section of the memory to receive additional input data from the party outside of
the \gls{tee}. The second reason is that the task can prepare an answer to the
outside party's request. For this, the task can write to fields dedicated to the
sender information, the command, and the payload. It is the task's
responsibility to write to the shared memory data needed to create a response to
the request once it is done. Currently, TEECore does not do any integrity checks
of the input. Before employing TEECore in real world scenarios such checks
should be implemented.

\section{Normal Partition Kernel: Linux}
\label{sec:implementation:hostKernel}

As a highly configurable open-source general-purpose operating system, Linux is
an example of the implementation of the host kernel. The Linux kernel I used for
my prototype is version 6.13. Linux allows its configuration by passing command
line arguments to the kernel on boot. This makes it able to limit the resources
Linux uses. \\

To isolate the CPU core, I use the command line option \textit{nr\_cpus=n},
where $n$ is an arbitrary number that limits Linux to using a maximum core
count of $n$. I use $n=3$ to limit Linux to 3 cores. While Linux supports CPU
hotplugging\footnote{\url{https://docs.kernel.org/core-api/cpu_hotplug.html}},
which would allow enabling cores to be turned off through the same feature,
\textit{nr\_cpu} introduces a hard limit. This limit originates from the fact
that managemant data structures for late CPU hotplugging need to be
preallocated. The \textit{nr\_cpu} limits the number of preallocated structures,
thus, cores excluded from initialization by this parameter are inaccessible for
Linux and cannot be hotplugged later~\cite{kernel-parameters}. \\

Another useful parameter is \textit{memmap}, which I use to add custom entries
to the BIOS memory map that Linux uses to set up its memory map. With this, I
add an entry with type \textit{persistent} of 4KiB size starting at the address
$0x9000$, which is within the first MiB of system memory and free for use. I
reserve this memory in order to use it later for booting one of the excluded
CPUs, as x86 CPUs initialize in 16-bit mode and, therefore, can only access
memory with addresses lower than the 1 MiB boundary. Linux does not offer any
procedure to kernel modules to allocate memory in this address region, so this
is the only possible way.\\

The driver for TEECore, implemented as a Linux kernel module, performs
additional initialization steps that I explain in
section~\ref{sec:implementation:kmod}. Linux, nevertheless, is the first
software to be run in my prototype. All other components, namely the \gls{tee}
kernel as an ELF-file image and the Linux kernel module embedded in its initial
RAM disk image, which allows in theory, the signing of a single image for a
secure boot thus ensures the integrity of all components on startup.\\

\section{Linux kernel module}
\label{sec:implementation:kmod}
As I described in section~\ref{sec:implementation:hostKernel}, the Linux kernel
module is the driver not only enabling Linux to communicate with the \gls{tee}
kernel on the isolated core but also the component that initializes the isolated
core and loads the \gls{tee} kernel into memory.\\

For this, the kernel module claims the memory reserved through the Linux command
line argument and copies the startup code to the reserved memory. I wrote the
startup code in assembly. It transfers the processor from real mode into the
32-bit mode with segmentation and places the addresses of the \gls{mbi}
structure into the EBX register. The resulting state is the one expected by
TEECore. The startup code is compiled before the kernel module and included as
an array of bytes. This allows for easier modification, as otherwise, the need
to edit a binary would arise. The \gls{mbi}, which TEECore expects, is created
by the kernel module at run time. Because of this, the address of the \gls{mbi}
is not known at compile time. After allocating the memory, the kernel module
modifies the startup code at runtime by replacing a placeholder with the actual
\gls{mbi} address.\\

In the next step, the kernel module locates the \gls{elf} binary of the TEECore
and places it in memory. For this, the module allocates memory through the Linux
kernel's \textit{allocpages} interface at a physical address aligned to 2 MiB.
Once allocated, the kernel module does not deallocate the pages to prevent Linux
from reusing the memory and possibly overwriting the TEECore. Once the required
memory is allocated and mapped in the kernel address space, the kernel module
begins to parse the \gls{tee} kernel's \gls{elf} image and copies all necessary
parts to memory. The kernel module brings its own \gls{elf} loader for this. The
kernel module does not use the Linux' \gls{elf} file loader because it does not
allow me to obtain the physical address to which the \gls{elf} image is
loaded. The kernel module requires the TEECore binary's physical address to
calculate the TEECore code entry address. This address is the second important
address generated at runtime of the kernel module with which the boot code is
updated. The kernel module then sets up the shared memory region by allocating
the amount of memory specified before. It then creates a \gls{mbi} structure
containing the physical address of the shared region with a custom-defined
memory type. TEECore later parses the \gls{mbi} structure to find the physical
memory backing the shared memory communication channel.\\

Once all parts necessary to run the \gls{tee} kernel are placed in memory, the
kernel module prepares to start the isolated core. At this point, Linux does not
allow interaction with any core not initialized by it, e.g., the kernel module
cannot address the isolated core with any kernel-provided functions. To
circumvent this shortcoming, I gained access to the \gls{lapic} of the core
running the kernel module to send the required INIT and startup \gls{ipi}
sequence to the isolated core. Another problem arises when implementing this
routine: APIC IDs are not bound to start at 0, nor do they have to increment by
one for each CPU core. To overcome this issue, I queried the APIC IDs of the
cores mapped by Linux and calculated the offset for each core. I used this
offset to calculate the APIC ID of the target core by adding the offset to the
ID of the last core mapped by Linux. After sending the \gls{ipi} sequence to the
remote core, the startup code will be executed, and a far jump to TEECore's code
will be performed by the remote CPU.\\

After initializing the remote core, the kernel module manages communication with
the \gls{tee} kernel. It, therefore, implements the protocol described in
\ref{sec:implementation:teeKernel}. To receive messages, the kernel module
installs a timer, which it uses to poll the shared memory periodically for new
messages. The kernel module also implements the remote part of tasks in the
prototype. To start a specific task, the kernel module sends a message
containing the task's ID and evaluates messages from the \gls{tee} to delegate
them to the respective tasks routine. Lastly, the kernel module creates a
character device to allow user space applications to use TEECore. Writes and
reads to this device are implemented by the kernel module and used to form
messages send to TEECore.

% \section{Applications}
% \label{sec:implementation:attacks}
% To proof the effectiveness of the prototype I implemented three possible attacks
% as tasks. Furthermore, I impemented a simple \textit{ping} task that allows me
% to investigate ressource usage of the \gls{tee} allone. In the follwing
% subsection I want to show some implementationt details of the attacks and the
% \textit{ping} application.

% \subsection{Ping Application}
% \label{sec:implementation:attacks:ping}

% The \textit{ping} application is a simple application that shows how to work
% with the shared memory communication path. This application does not require to
% allocate additional memory in any form, it only used the shared memory. The
% purpose of this application is to increment a variable shared beyond partition
% boundaries on both, the isolated partition and the normal partition. For this,
% ping application in the secure partition writes an identifier to identify itself
% against in the normal partition and initliazes the first byte of the payload
% region in the shared memory with zero. The linux kernel module, upon recieving a
% message from the ping application, calls a matching handler function. If the
% message would traget from any user space application, this handler function
% would delegate the message to it. The handler function increases the shared
% counter in the payload field and write to the message identifier, triggering a
% response from the secure partition's ping application. After a specified number
% of iteration, the secure application terminates. Consequently TEECore changes to
% a state in which it awaits the next message form the normal partiton.

% WIP\todo{Describe some details of secure applications}
% This simple task does nothing more than to increment a value shared between the
% \gls{tee} and the host OS. It can be viewed as the minimal working set of the
% \gls{tee} to run tasks. Therefore, the result for memory usage is the minimum to
% expect for any other task. Measuring the resource usage with appropriate
% \gls{pmc} allows me to make assumptions about the resources a future payload can
% use. I collected data with the following events:
% \begin{itemize}
%     \item L1 replacements
%     \item L1 Miss (any)
%     \item Offcore events \todo{umask and so angeben um reproduzierbarkeit zu
%               wahren}
% \end{itemize}
% Between each ping cycle I used used the \textit{wbinvd} instruction of x86, that
% results in the core writing back all it's cache entries to memory and
% invalidating each cache line. Executing the following ping cycle until reading
% the \gls{pmc} again yields the number of cache lines filled with data necessary
% to execute a complete cycle. From the cache line size I can then conclude the
% total memory requirements of the runtime part of the \gls{tee}.

% \subsection{Stalling Attack}
% The denial of service attack floods the \gls{tee} with \glspl{nmi}. It is
% intended to prove that the \gls{tee} can defend against interrupts that the
% \gls{tee} cannot mask.
% \todo{maybe move this into the chapter intro}

% \subsection{Memory Attacks}
% \label{sec:implementation:attacks:memory}
% In theory, an attacker is able to leak data through the microarchitectural
% behavior of the CPU cache's implementation. The first attacker is passive and
% only interested in spying unobserved on the secrets of the \gls{tee}. The
% attacker would use their access to read the \gls{tee} memory. If the \gls{tee}
% cannot observe this, then secrets could be leaked through this channel upon
% writing. The second attacker is an active one who tries to write the \gls{tee}
% memory to influence its control flow. Both attacks follow the same workflow.\\

% To simulate these attacks, the \gls{tee} first initializes a memory area of the
% size of 4KiB and finds its physical location. Because I want to observe the side
% effects of the attack, this memory is not mapped as uncacheable on the \gls{tee}
% side but instead like any other memory in the \gls{tee}. The \gls{tee} then
% communicates the address through the shared memory channel to the Linux kernel
% module in which the respective attack implementations reside. The remote side
% now maps the physical address space into its address space and either reads or
% writes to the memory. When finished, the remote kernel module creates the
% respective answer and transfers control over the memory area on this way back to
% the \gls{tee}. The \gls{tee} now reports on the state of the performance
% counters.\todo{maybe add some code examples?}\\

% \section{Attacks}
% \label{sec:30:attack}
% The goal of the \gls{tee} prototype is to defend against an attacker with access to it
% is backing memory. An attacker could trigger leakage or modification through
% access to the right memory address with the help of cache coherence protocols.
% These are the same mechanisms used in the Spectre and meltdown class
% side-channel attacks. By simulating these kinds of attacks, we can study the
% behavior of the prototype's implementation.\\

% From this, we can derive two attackers based on their invasions. The first class
% of attackers are passive ones, those who can read memory and are only interested
% in spying secrets from the \gls{tee}. The second attacker is an active one that
% modified code protected in the \gls{tee} with malicious intent.\\

% From a technical perspective, the active attacker is easier to spot. Code that
% resides in the cache of the isolated core that attackers did not exfiltrate will
% be exclusive or shared in the MESI state. Shared denotes the state where the
% respective cache line is contained in another cache, possibly in an invalid
% state in the remote cache. Exclusive denotes the state of exclusive ownership of
% the cache line. In other words, the item is not contained in any other cache. If
% the attacker now attempts to change the memory item, this results in
% invalidating the cache line in the core owned by the \gls{tee}. The result on the test
% CPU is a snoop, which offcore performance counter events can track.\\

% The passive attacker is harder to spot. Because changing a cache line from
% exclusive to shared does not necessarily result in data transfer between the
% CPU-exclusive parts of the cache. For example, if both the \gls{tee} and a remote CPU
% have valid copies of a data item in their respective caches, the cache line in
% both CPUs is tagged as shared. If one core modifies the data item, the write
% access results in communication between the cores, invalidating the remote cache
% line. A scenario in which a passive attacker might not be spotted is if the
% attacker can somehow gain access to memory that the \gls{tee} did not modify after
% setting up the environment protection routines. Therefore, we assume the
% attacker can read arbitrary memory at any time. A successful defense is if those
% reading attempts can be reliably spotted by the \gls{tee}.\\

% A third class of attackers could try to send interrupts to the \gls{tee} with varying
% goals. Like SGX step\todo{cite}, an attacker could try to single-step the \gls{tee}.
% Regarding the isolation, this single stepping does not allow us to learn
% anything from the \gls{tee}. Another goal could be to flood the \gls{tee} with \glspl{nmi} to
% prevent the \gls{tee} from reacting to an attack. Theoretically, the \gls{tee} must react to
% the \glspl{nmi} before doing anything else. While the design creates an interrupt-free
% environment, I must consider the possibility of such an attack. The \gls{tee}
% implementation should react appropriately even if flooded by \glspl{nmi}.\\

% The last possible attack vector is \glspl{ipi}. If the attacker can reset the CPU and
% make it execute code that leaks secret, appropriate countermeasures should be
% thought of. How a CPU reacts to such \glspl{ipi} is implementation-specific. Therefore,
% the purpose of the simulation of such an attack is to collect data about this
% topic and not necessarily find a defense mechanism.\\

%%% Local Variables: %% TeX-master: "diplom" %% End:
