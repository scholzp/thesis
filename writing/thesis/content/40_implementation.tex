\chapter{Implementation}
\label{sec:implementation}

% Hier greift man einige wenige, interessante Gesichtspunkte der Implementierung
% heraus. Das Kapitel darf nicht mit Dokumentation oder gar Programmkommentaren
% verwechselt werden. Es kann vorkommen, daß sehr viele Gesichtspunkte
% aufgegriffen werden müssen, ist aber nicht sehr häufig. Zweck dieses Kapitels
% ist einerseits, glaubhaft zu machen, daß man es bei der Arbeit nicht mit einem
% "Papiertiger" sondern einem real existierenden System zu tun hat. Es ist
% sicherlich auch ein sehr wichtiger Text für jemanden, der die Arbeit später
% fortsetzt. Der dritte Gesichtspunkt dabei ist, einem Leser einen etwas
% tieferen Einblick in die Technik zu geben, mit der man sich hier beschäftigt.
% Schöne Bespiele sind "War Stories", also Dinge mit denen man besonders zu
% kämpfen hatte, oder eine konkrete, beispielhafte Verfeinerung einer der in
% Kapitel 3 vorgestellten Ideen. Auch hier gilt, mehr als 20 Seiten liest
% keiner, aber das ist hierbei nicht so schlimm, weil man die Lektüre ja einfach
% abbrechen kann, ohne den Faden zu verlieren. Vollständige Quellprogramme haben
% in einer Arbeit nichts zu suchen, auch nicht im Anhang, sondern gehören auf
% Rechner, auf denen man sie sich ansehen kann.

In the following section, I will explain some technical implementation details.
In particular, I want to highlight some of the implementation details of
components explained in chapter~\ref{sec:design}. First, I will refer to Linux,
which I use as the host kernel in the system. A special configuration was used
because I will make it run in parallel to the \gls{tee} kernel. Next, in
section~\ref{sec:implementation:teeKernel}, I will introduce phipsboot, which I
use as the \gls{tee} kernel, along with the modification I introduced to adapt
it to my use case. Another component I programmed is the Linux kernel module,
which I explain in section~\ref{sec:implementation:kmod}. In the last part of
the implementation chapter, I will explain how I implemented the attack
simulations.

\section{TEECore}
\label{sec:implementation:teeKernel}

PhipsBoot\footnote{\url{https://github.com/phip1611/phipsboot}} is a Multiboot 2
compliant kernel written in the memory-safe programming language Rust, with the
purpose to serve as a bootloader for 64 bit x86 kernels. A feature of PhipsBoot
is, that it is relocateable and so supports arbitrary load addreses. It depends
on a first-stage bootloader\todo{maybe this is not a multi-stage bootloader}
that brings the system to a state defined in the Multiboot 2 specification. That
is, the CPU is in 32-bit protected mode with segmentation enabled. After
handover, PhipsBoot does all initialization steps to bring the CPU into 64-bit
mode with enabled PEA paging. This part of Phipsboot is written in x86 assembly
code. It initializes the page tables and sets up huge pages before transfering
the CPU into 64-bit mode. For later usage, Phipsboot sets up additional page
tables for execution. PhipsBoot expects to be loaded to a 2 MiB-aligned address
because of it memory mapping. Its small size, which is less than 2 MiB, allows
PhipsBoot to set up the page table hierarchy using a single page table that maps
the kernel. PhipsBoot uses memory statically reserved for dynamic memory
allocation in the kernel's Elf image. Next to setting up the general run time,
PhipsBoot does not implement more feature, but offers a great starting point for
implementation of TEECore. \\

For the inent of my implementation, I forked PhipsBoot to begin developing
TEECore. The first feature I introduced was the ability to modify page tables,
because Phipsboot did not support mapping of arbitrary memory. Instead, the
setup assembly routine of Phipsboot mapps the memory of the page table as read
only. Moreover the address of the last level page tabel is not known by the Rust
high level code. To allow mapping of arbitrary memory, I modified the page table
setup routine and changed the last page table to be writable instead of
read-only. Through observation, I found that not more than the first 128 entries
in the page table are used to map all memory used by the kernel binary. For
dynamic mapping of memory that is not contained in TEECore's binary, I,
therefore, reserved the last 128 entries of the last level page table. This
enabled TEECore to map memory that was not contained in the binary.
\begin{lstlisting}[language=Rust, caption=Mapping of external memory]
    pub unsafe fn get_virtual_address(phys_addr: u64) -> u64 {
        // Calculate the last 12 bits from the offest in the page frame
        let offset_within_page: u64 = (phys_addr & 0xFFF).try_into();
        for x in 0..512 {
            // Calculate the upper 52 bits of the virtual address from the
            // L1 tables virtual address
            let pt_entry = (ptr::read((LAST_L1 as *mut u64).add(x)) & !(0xFFF)) & !(0x1_u64 << 63);
            if (phys_addr & !(0xFFF)) == pt_entry {
                return pt_entry | offset_within_page;
            }
        }
        return 0;
    }
\end{lstlisting}

This
feature is important to implement the shared memory communication path. The
address is communicated to PhipsBoot through the Multiboot 2 memory map. To
access this structure, PhipsBoot needs to be able to map the page containing it.
Moreover, the address in the memory map must be mapped by PhipsBoot to make the
shared memory accessible. Other than the memory owned by the kernel, the memory
intended to be used by the \gls{tee} kernel and the host kernel is mapped as
uncacheable. This prevents cache pollution and false positives for allowed
access of a remote core to memory, also used by the \gls{tee}.\\

To prepare the environment, the \gls{tee} kernel needs to ensure that all memory
used by it is in the exclusive or modified state so that a remote core must
trigger an offocre\todo{Irgendwo mal erklären was ein OFfcore event ist} event
when it tries to access memory owned by the \gls{tee} kernel. For this, I
implemented a function that iterates over the page table entries and copies all
bytes in place. This triggers the modified state of the cache line, which leads
to invalidating the cache line in remote cores referencing the same memory item.
I discovered this necessity when implementing my attack simulations
(c.f.~\ref{sec:implementation:attacks:memory}). While the \gls{tee} could detect
the active attacker, the passive one was undetectable because the reading of
cache lines, which are in the shared state in both caches, does not trigger any
microarchitectural responses. It is, therefore, necessary to hold all memory
that the \gls{tee} should keep secret in the exclusive or modified state to
detect cache synchronization over the cache coherency protocol.\\

After ensuring that all attempts to manipulate or access secret memory do not
result in additional microarchitectural state changes, it is time to make the
\gls{tee} detect the respective events resulting from access to the remote core
through monitoring performance counters. For this, the \gls{tee} writes the
respective events to the configuration MSRs. Once prepared, the \gls{tee} sets
the IDTR to null, which leads to triple faulting once an interrupt is delivered
to the \gls{tee}. The only expected interrupts in the \gls{tee} are \glspl{nmi},
\glspl{pmi}, and \glspl{ipi} from other cores. All other interrupts are masked
by clearing the interrupt flag in the EFLAGS register of the isolated core. \\

\glspl{nmi} can be sent by faulty hardware and the interrupt controller if
configured accordingly. Currently, the \gls{tee} has no other means of defense
against malicious \gls{smm} code. \glspl{ipi} originate from other cores and can
be sent by those at any time. Because \glspl{ipi} are routed through the IDT,
the same countermeasures as with \glspl{nmi} are taken. The invalid setting of
the IDTR to NULL results in a triple fault that ultimately results in a reset of
the whole machine. The same defense strategy is employed upon registering an
offshore event triggered through a read-or-write attempt from a remote core. By
defining a threshold for event occurrence upon which overflow emits a \gls{pmi},
the \gls{tee} can terminate as it wishes. A useful threshold is 1, which results
in a \gls{pmi} triggered by the first occurrence of the respective event. For
this, we use an event that counts all occurrences of off-core communication not
served by main memory. This works because all memory is placed in the cache, and
only the communication routine is accessing the uncacheable main memory. \\

\begin{center}
    \begin{figure}
        \centering
        \includestandalone{images/shared_mem_layout.tex}
        \caption{Illustration of the Layout of the Shared Memory used For Communication}
        \label{fig:state:technical:paging}
    \end{figure}
\end{center}

I implemented a structure to manage the memory in question to enable
communication through the shared memory channel. Access to the shared memory
follows a simple protocol that splits the memory, as shown in
figure~\ref{fig:shared-mem}. The first byte denotes the sender, the second
denotes the task identifier, and the remaining part is used for a payload that
can be sent along with the other tags. I assume there are only two communication
parties in the current prototype: the \gls{tee} kernel and the host kernel. A
message is sent by writing the first byte. To receive a message, the respective
party polls the first byte of the shared memory and waits until the signal is
written by the other party, \\

After initialization, the \gls{tee} environment starts a state machine that
controls the execution of tasks. In the first state, the \gls{tee} polls the
shared memory until it receives a message containing the command to execute one
of the predefined tasks. Once this message is received, the \gls{tee} executes
the task as commended. While the task runs, it can access the shared memory for
multiple reasons. The first reason is that the task can access the payload
section of the memory to receive additional input data from the party outside of
the \gls{tee}. The second reason is that the task can prepare an answer to the
third-party request. For this, the task can write to the sender information, the
command, and the payload fields as it wishes. It is the task's responsibility to
write to the shared memory data needed to create a response to the request once
it is done. After the task returns, the \gls{tee} creates a report from the
performance counters and sends the result to the third party over the shared
memory channel.\todo{Does it really?} \\

\section{Host Kernel: Linux}
\label{sec:implementation:hostKernel}

As a highly configurable open-source general-purpose operating system, Linux is
an example of the implementation of the host kernel. The Linux kernel I used for
my prototype is version 6.13. Linux allows the configuration at runtime by
passing command line arguments to the kernel on boot. This makes it able to
limit the resources Linux uses. \\

To isolate the CPU core, I used the command line option \textit{nr\_cpus=n},
where $n$ is an arbitrary number that limits Linux from using a maximum core
count of $n$. I used $n=3$ to limit Linux to 3 cores. While Linux supports CPU
hotplugging\todo{ref}, which would allow enabling core to be turned off through
the same feature, \textit{nr\_cpu} introduces a hard limit. Thus, cores excluded
from initialization by this parameter are inaccessible for Linux and can't be
hotplugged later. \\

Another useful parameter is \textit{memmap}, which I used to add custom entries
to the BIOS memory map that Linux uses to set up its memory map. With this, I
added an entry with type \textit{persistent} of 4KiB size starting at the
address $0x9000$ is within the first MiB of system memory and not marked free in
the BIOS specification\todo{cite}. I reserve this memory in order to use it
later for booting one of the excluded CPUs, as x86 CPUs initialize in 16-bit
mode and, therefore, can only access memory with addresses lower than the 1 MiB
boundary. Linux does not offer any procedure to allocate memory in this address
region, so this is the only possible way. \\

Other preparation steps are done by the Linux kernel module explained in
section~\ref{sec:implementation:kmod}. Linux Neverthelss serves is the first
software to be run in my prototype. All other components, namely the \gls{tee}
kernel as an Elf-file image and the Linux kernel module, are embedded in its
initial RAM disk image, which allows, in theory, to sign a single image for
secure boot and thus ensure the integrity of all components on startup.\\

\section{Linux kernel module}
\label{sec:implementation:kmod}
As I described in section~\ref{sec:implementation:hostKernel}, the Linux kernel
module is the driver not only enabling Linux to communicate with the \gls{tee}
kernel on the isolated core but also the component that initializes the isolated
core and loads the \gls{tee} kernel into memory.\\

For this, the kernel module claims the memory reserved through the Linux command
line argument and copies the startup code. This code sequence brings the
processor from real mode into 32-bit mode with segmentation and places the
addresses of Multiboot 2 structures into the registers. The resulting state is
the one expected by the \gls{tee} kernel. Because these addresses are not known
beforehand, the kernel module modifies the startup code on runtime to contain
the respective addresses.\\

In the next step, the kernel module locates the Elf binary of the \gls{tee}
kernel and places it in memory. For this, the module allocates memory through
the Linux kernel's \textit{allocpages} interface at a physical address aligned
to 2 MiB. Once allocated, the kernel module does not deallocate the pages to
prevent Linux from reusing the memory and possibly overwriting the \gls{tee}
kernel. Once the required memory is allocated and mapped in the kernel address
space, the kernel module begins to parse the \gls{tee} kernel's Elf image and
copies all necessary parts to memory. It then sets up the shared memory region
and updates. It generates a multiboot 2 Information structure containing the
region with a custom-defined memory type used by the \gls{tee} kernel to find
the shared memory.\\

Once all parts necessary to run the \gls{tee} kernel are placed in memory, the
kernel module prepares to start the isolated core. At this point, Linux does not
allow interaction with any core not initialized by it, e.g., the kernel module
cannot address the isolated core with any kernel-provided functions. To
circumvent this shortcoming, I gained access to the \gls{lapic} of the core
running the kernel module to send the required INIT and startup \gls{ipi}
sequence to the isolated core. Another problem arises when implementing this
routine: APIC IDs are not bound to start at 0, nor do they have to increment by
one for each CPU core. To come by this, I queried the APIC IDs of the core
mapped by Linux and calculated the offset for each core. I used this offset to
calculate the APIC ID of the target core by adding the offset to the ID of the
last core mapped by Linux. After sending the \gls{ipi} sequence to the remote
core, the startup code will be executed, and a far jump to the \gls{tee}
kernel's code will be performed by the remote CPU.\\

After initializing the remote core, the kernel module manages communication with
the \gls{tee} kernel. It therefore implements the protocol described in
\ref{sec:implementation:teeKernel}. To receive messages, the kernel module
installs a timer, which it uses to poll the shared memory periodically for new
messages. The kernel module also implements the remote part of tasks in the
prototype. To start a specific task, the kernel module sends a message
containing the task's ID and evaluates messages from the \gls{tee} to delegate
them to the respective tasks routine. \\
\todo{If anything ever uses the chardev then I should write about it}

\section{Attacks and Applications}
\label{sec:implementation:attacks}

To proof the effectiveness of the prototype I implemented three possible attacks
as tasks. Furthermore, I impemented a simple \textit{ping} task that allows me
to investigate ressource usage of the \gls{tee} allone. In the follwing
subsection I want to show some implementationt details of the attacks and the
\textit{ping} taks.

\subsection{Ping Task}
\label{sec:implementation:attacks:ping}
This simple task does nothing more than to increment a value shared between the
\gls{tee} and the host OS. It can be viewed as the minimal working set of the
\gls{tee} to run tasks. Therefore, the result for memory usage is the minimum to
expect for any other task. Measuring the resource usage with appropriate
\gls{pmc} allows me to make assumptions about the resources a future payload can
use. I collected data with the following events:
\begin{itemize}
    \item L1 replacements
    \item L1 Miss (any)
    \item Offcore events \todo{umask and so angeben um reproduzierbarkeit zu
              wahren}
\end{itemize}
Between each ping cycle I used used the \textit{wbinvd} instruction of x86, that
results in the core writing back all it's cache entries to memory and
invalidating each cache line. Executing the following ping cycle until reading
the \gls{pmc} again yields the number of cache lines filled with data necessary
to execute a complete cycle. From the cache line size I can then conclude the
total memory requirements of the runtime part of the \gls{tee}.

\subsection{Stalling Attack}
The denial of service attack floods the \gls{tee} with \glspl{nmi}. It is
intended to prove that the \gls{tee} can defend against interrupts that the
\gls{tee} cannot mask.
\todo{maybe move this into the chapter intro}


\subsection{Memory Attacks}
\label{sec:implementation:attacks:memory}
In theory, an attacker is able to leak data through the microarchitectural
behavior of the CPU cache's implementation. The first attacker is passive and
only interested in spying unobserved on the secrets of the \gls{tee}. The
attacker would use their access to read the \gls{tee} memory. If the \gls{tee}
cannot observe this, then secrets could be leaked through this channel upon
writing. The second attacker is an active one who tries to write the \gls{tee}
memory to influence its control flow. Both attacks follow the same workflow.\\

To simulate these attacks, the \gls{tee} first initializes a memory area of the
size of 4KiB and finds its physical location. Because I want to observe the side
effects of the attack, this memory is not mapped as uncacheable on the \gls{tee}
side but instead like any other memory in the \gls{tee}. The \gls{tee} then
communicates the address through the shared memory channel to the Linux kernel
module in which the respective attack implementations reside. The remote side
now maps the physical address space into its address space and either reads or
writes to the memory. When finished, the remote kernel module creates the
respective answer and transfers control over the memory area on this way back to
the \gls{tee}. The \gls{tee} now reports on the state of the performance
counters.\todo{maybe add some code examples?}\\

% \section{Attacks}
% \label{sec:30:attack}
% The goal of the \gls{tee} prototype is to defend against an attacker with access to it
% is backing memory. An attacker could trigger leakage or modification through
% access to the right memory address with the help of cache coherence protocols.
% These are the same mechanisms used in the Spectre and meltdown class
% side-channel attacks. By simulating these kinds of attacks, we can study the
% behavior of the prototype's implementation.\\

% From this, we can derive two attackers based on their invasions. The first class
% of attackers are passive ones, those who can read memory and are only interested
% in spying secrets from the \gls{tee}. The second attacker is an active one that
% modified code protected in the \gls{tee} with malicious intent.\\

% From a technical perspective, the active attacker is easier to spot. Code that
% resides in the cache of the isolated core that attackers did not exfiltrate will
% be exclusive or shared in the MESI state. Shared denotes the state where the
% respective cache line is contained in another cache, possibly in an invalid
% state in the remote cache. Exclusive denotes the state of exclusive ownership of
% the cache line. In other words, the item is not contained in any other cache. If
% the attacker now attempts to change the memory item, this results in
% invalidating the cache line in the core owned by the \gls{tee}. The result on the test
% CPU is a snoop, which offcore performance counter events can track.\\

% The passive attacker is harder to spot. Because changing a cache line from
% exclusive to shared does not necessarily result in data transfer between the
% CPU-exclusive parts of the cache. For example, if both the \gls{tee} and a remote CPU
% have valid copies of a data item in their respective caches, the cache line in
% both CPUs is tagged as shared. If one core modifies the data item, the write
% access results in communication between the cores, invalidating the remote cache
% line. A scenario in which a passive attacker might not be spotted is if the
% attacker can somehow gain access to memory that the \gls{tee} did not modify after
% setting up the environment protection routines. Therefore, we assume the
% attacker can read arbitrary memory at any time. A successful defense is if those
% reading attempts can be reliably spotted by the \gls{tee}.\\

% A third class of attackers could try to send interrupts to the \gls{tee} with varying
% goals. Like SGX step\todo{cite}, an attacker could try to single-step the \gls{tee}.
% Regarding the isolation, this single stepping does not allow us to learn
% anything from the \gls{tee}. Another goal could be to flood the \gls{tee} with \glspl{nmi} to
% prevent the \gls{tee} from reacting to an attack. Theoretically, the \gls{tee} must react to
% the \glspl{nmi} before doing anything else. While the design creates an interrupt-free
% environment, I must consider the possibility of such an attack. The \gls{tee}
% implementation should react appropriately even if flooded by \glspl{nmi}.\\

% The last possible attack vector is \glspl{ipi}. If the attacker can reset the CPU and
% make it execute code that leaks secret, appropriate countermeasures should be
% thought of. How a CPU reacts to such \glspl{ipi} is implementation-specific. Therefore,
% the purpose of the simulation of such an attack is to collect data about this
% topic and not necessarily find a defense mechanism.\\

\cleardoublepage

%%% Local Variables: %% TeX-master: "diplom" %% End:
