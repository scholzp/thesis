\chapter{Related Work}
\label{chap:related}

In the following sections we will see multiple implementations of CPU vendors,
learn about \glspl{tpm} and how a software solution can look. All of them serve
as input for my implementation as they show how to construct a \gls{tee} that
contains all necessary features.

\section{Intel Software Guard Extensions}
\label{sec:20:sgx}
Intel \gls{sgx} is an extension in x86\_64 processors manufactured by Intel,
that allows the creation of trusted execution environments. Intel first shipped
\gls{sgx} in 2015, with processors implementing the Skylake microarchitecture.
While server-grade CPUs are still implementing \gls{sgx}, Intel marked \gls{sgx}
as deprecated in 2021 in consumer-grade CPUs, beginning from CPUs implementing
the Rocket Lake microarchitecture. Costan et al. did an extensive study of
\gls{sgx} in 2016 and documented in detail how \gls{sgx} works and what its
security properties are~\cite{costan2016intel}. \\

Features of \gls{sgx} include the creation of enclaves. Enclaves are especially
access-protected and encrypted system memory regions, with \gls{sgx} preventing
direct memory access. In the creation process, memory pages are added to the
enclave page cache (EPC) and assigned to the enclave. Once assigned to an
enclave, \gls{sgx} protects the memory page from unprivileged access, which
includes all access attempts not originating from the memory-owning enclave.
After the system software adds all pages to the enclave, it is marked as
initialized. For the initialization process, system software uses privileged
instructions. After the enclave is marked initialized, no more pages can be
added to the EPC, and interaction with the enclave is only allowed by using
dedicated instructions available only in user space. The EPC is similar to an
inverted page table, which purpose is to protect an enclave's security by
ensuring the correctness of the allocation process. \gls{sgx} would, for
example, refuse operation if it detects that the same page frame was used for
two different enclaves.\\

The enclave code runs at the permission level of the application from which the
enclave was called. Intel equips each CPU with unique cryptographic keys that
the CPU uses to encrypt code and data placed in the EPC. These keys reside in
memory made of electronic fuses that cannot be reprogrammed. Intel programs
this memory in the factory process by burning some of the fuses. Applications
using \gls{sgx} services do not necessarily need to run as a whole in an
\gls{sgx} enclave. Because of restrictions on the size of the enclave's memory,
only parts of the data were handed to enclaves. Again, communications between
the enclave and user applications use special CPU instructions. In cases where
applications are split into parts residing in and outside of the enclave, an
application might want to verify the identity before sharing secrets. For this,
\gls{sgx} implements local attestation.\\

As mentioned, \gls{sgx} implements processor instructions dedicated to managing
and interacting with enclaves. Furthermore, \gls{sgx} implementations create at
least a launch enclave signed by Intel. Third party enclaves that are not signed
by Intel require the launch enclave for their successful initialization. It is
necessary in all cases when \gls{sgx} is used.\\

An attesting enclave can verify its identity with another enclave in the same
system by generating a report. The attesting enclave calls the \textbf{ERPEORT}
instruction to generate this report. As a result, the CPU measures the attesting
enclave and generates a report that contains the measurement, the enclave's
identity, and a message provided by the enclave. The attesting enclave then
ships the generated report to another enclave for local attestation.
\begin{center}
  \begin{figure}
    \centering
    \includestandalone{images/25_sgx_attestation.tex}
    \caption{Software Attestation Scheme of \gls{sgx}}
    \label{fig:state:tee:sgx_attestation}
  \end{figure}
\end{center}
Figure~\ref{fig:state:tee:sgx_attestation} shows the remote attestation scheme
implemented by Intel \gls{sgx}. It extends the local attestation scheme so that
a third party not part of the system can verify the identity of an enclave. The
quoting enclave serves as a root of trust in this process. It is a unique
enclave provided by Intel that has access to the CPUs infused keys and can thus
act as a trust anchor. Upon receiving a report from another enclave, the quoting
enclave verifies said report and signs it. A remote third party can then check
the quoting enclave's signature to verify the correctness of the report.

\section{Confidential Virtual Machine Extensions}
\label{section:20:confidential_vms}
The goal of confidential Virtual machines is to protect the entire \gls{vm} from
the influence of a malicious hypervisor or other privileged software. Intel and
AMD offer individual \gls{isa} extensions for their processors to host
confidential \glspl{vm}. Intel calls its solution Intel \gls{tdx}, while AMDs
solution is called AMD \gls{sev}~\cite{tdx_whitepaper,kaplan_amd_2020}. Both
solutions use the same fundamental building blocks to achieve the goals of
confidential \glspl{vm}. Misano et al. did a extensive comparison of both
technologies~\cite{misono_confidential_2024}. Intel uses the \gls{sgx} module
for its implementation. Additionally, to interact with a confidential \gls{vm},
the CPU must be in the dedicated CPU operation mode called SEAM mode. Memory
access is only allowed in SEAM mode to protect confidential \glspl{vm}. Once in
SEAM mode, the CPU uses its Virtual Machine Extensions (VMX) capabilities to
host and interact with an \gls{vm}. For cryptographical features, such as
signing and key generation, Intel processors utilize the Intel Management
Engine. The Intel management Engine is a coprocessor located in the CPU package
with special firmware and a separate \gls{os} that is isolated from the
remaining parts of the system.\\

For \gls{sev}, AMD uses the already implemented SEV capabilities and its
extensions. In its initial version SEV encrypts the memory of a virtual machine
with a dedicated key per virtual machine. SEV-ES extends the feature set by
additionally encrypting the system state, e.g. processor registers, too. With
the most recent version, AMD introduced a feature called nested paging, which
manages an inverted page table similar to \gls{sgx} to prevent attacks targeting
an \gls{vm}s page tables.\\

Unlike Intel's implementation, AMD processors do not utilize a dedicated CPU
mode but extend the existing \gls{vm} control structure by fields to enable
Secure Nested paging. For cryptography, the integrated AMD Platform Security
Processor, short PSP, is used. Both solutions encrypt an \gls{vm}'s memory to
protect an \gls{vm} from being manipulated by system software with one key per
\gls{vm}.\\

Both solutions use the appraiser's knowledge of the initial state of an \gls{vm}
image. The assumption that the approach follows is that if an \gls{vm} is
started in a known state and protected from manipulation by the hypervisor or
other privileged software, then an \gls{vm} can be trusted in the following. To
follow this approach, a measurement of the initial \gls{vm} image is created and
cryptographically bound to the respective \gls{vm} instance through a message
authentication code. Before the trustee interacts with an \gls{vm}, they
request an \gls{vm} to verify its identity. For this, an \gls{vm} requests the
cryptographic hardware to sign a report. By signing the report, the
cryptographic hardware attests to its correctness and then hands it to the
appraiser. The appraiser trusts the implementation in the CPU and verifies the
signature of the CPU signed report. With this, the appraiser knows if the
\gls{vm} images were expected. In the following, both the appraiser and the
\gls{vm} exchange keys for further communication.\\

\section{ARM TrustZone}
\label{sec:20:trustzone}
As another widely spread \gls{isa} ARM dominates the mobile sector. Like x86,
the ARM architecture offers technology to allow isolated program execution. On
ARM, this technology is called ARM TrustZone. TrustZone is optional for ARM
processor implementations and slightly differs between the ARM Cortex-A
application processors and the ARM Cortex-M microcontroller-aimed processors. In
the following, we concentrate on the implementation of ARM application
processors. Pinto and Santos did an extensive survey of ARM TrustZone in 2019.
In their work, they describe technical properties of ARM TrustZone and how to
use it for the implementation of \glspl{tee} and hypervisors. Moreover they
explain technical details of TrustZone and review its security properties
against other \glspl{tee}~\cite{pinto_demystifying_2019}.\\

Conceptually, ARM TrustZone-enabled processors offer three processor operation
modes. The most privileged mode is the Secure Monitor mode or short SM mode. The
Secure Monitor mode is the mode in which the processor boots and the firmware
and Bootloader executes. The second most privileged mode is the Secure World.
This mode is intended to execute code isolated from the third and least
privileged mode, the Normal World. The Bootloader is responsible for installing
software intended to run in the Secure World. Isolation is achieved by hardware.
For example, the stack pointer (SP) and return address (LR) registers exist
twice to allow fast context switches between the Normal and Secure World. The
TrustZone Address Space Controller can be used to partition memory into regions
only accessible from the Secure World and those accessible by both worlds.
Changing between worlds can be done synchronously using the dedicated SMC
(System Monitor Call) instruction or asynchronously due to an interrupt. The SMC
instruction also triggers an interrupt. These interrupts are served by invoking
the SM, which decides upon its configuration if the interrupt received serves as
an entry point to the secure world. If so, the Secure Monitor invokes secure
world code to serve the interrupt.\\

ARM TrustZone does not specify the \gls{tee} and remote attestation features.
Such functionality has to be implemented by the code running in the Secure World
and might the implementation of those features might be specific to system or
hardware vendors. \gls{tee} and remote attestation functionality can be
implemented by a bare metal application or by using a trusted \gls{os} to host
secure applications in the Secure World. The first solution minimizes code size,
while the second offers the ability to host multiple applications in the Secure
World. The trusted \gls{os} would be responsible for isolating services running
in the Secure World against each other because applications running in the
Secure World are not isolated from other applications in the Secure World by
hardware. Using a trusted \gls{os} brings the downside of an increased trusted
computing base compared to a bare metal trusted application. TrustZone does not
encrypt the memory of the secure world, instead it utilizes the physical
isolation of both worlds.\\

\section{Enma}
\label{sec:20:enma}
Enma is short for Enclave Manager and is a software solution to implement
\glspl{tee} for the L4Re operating system
framework~\cite{reitz_isolierende_2019}. L4Re uses the Fiasco.OC microkernel
that is part of the L4 microkernel family. The microkernel approach aims for
minimizing the amount of code to run in the supervisor mode. The creator of L4
defined the concept of microkernels as follows:

\begin{quote}
  \textit{ More precisely, a concept is tolerated inside the \mu-kernel only
    if moving it outside the kernel, i.e., permitting competing
    implementations, would prevent the implementation of the system's
    required functionality. \\
  } \mbox{ -- Liedke~\cite{liedtke1995micro}}
\end{quote}

Following this philosophy, Fiasco.OC only implements the mere minimal services.
All other services, such as memory management, are implemented as user space
applications. The minimal setup of a L4Re system consists of the Fiasco.OC
kernel, Sigma0, which manages memory, and root task moe, which is responsible
for loading applications. As a capability based system, the kernel ensures that
only applications with the right capabilities have access to process memory.
Initially, only the creator of a process posses these capabilities and the
ability to grant them to other processes. Enma uses these properties to create
isolated enclaves in which programs can execute isolated.\\

Enma follows the same approach as \gls{sgx} for implementing remote attestation
and \gls{tee} functionality. Enma hosts a quoting enclave that sings reports to
verify the state of an enclave. The appraiser can then check the signature of
the quoting enclave, upon which they decide how to interact with the enclave.
Enma encrypts the memory of its enclaves. Security-wise, Enma is vulnerable to
side-channel attacks as explicitly stated in the work, just as hardware
extensions like \gls{sgx} and \gls{sev}. The \gls{tcb} is also increased when
compared to hardware solutions. An application that trusts Enma must also trust
the L4Re runtime environment as much as the firmware and hardware. Enma is
designed to support different backend, with the default backend utilizing a
\gls{tpm} as the hardware root of trust for verifying the system state and that
the expected versions of L4Re and Enma were booted. The authors additionally
sketch how a back end implementation could look that uses \gls{sgx} and other
hardware solutions. It is therefore not aiming to replace such solutions but
rather complement them.

\section{Attack Mitigations and countermeasures}
\label{sec:20:mitigations}
This section lists mitigations and countermeasures for known side channel
attacks. These mitigations and countermeasures serve as examples for my
implementation and can be viewed as part of related work, as they show how to
either detect or prevent the usage of side channels and attacks that utilize
them.

\subsection{Isolation through System Management Mode}
\label{sec:20:isolation_smm}
An early work on how to isolate processes was done by Azab et al. in
2011~\cite{azab_sice_2011}. The work dates before introducing \gls{tee}
extensions in x86 hardware and uses the \gls{smm} to isolate tasks. The problem
SICE tries to solve is to protect the memory integrity of isolated tasks and
virtual machines. In principle, SICE uses the strong hardware-enforced isolation
of the \gls{smm} and its SMRAM to install applications into it. The authors used
an AMD platform for their practical implementation because AMD platforms allow
the adjustment of SMRAM size and location after the \gls{smm} code locks the
SMRAM~\cite{bios2014amd}. To switch to the isolated task residing in SMRAM, the
firmware \gls{smi} handler was modified to transfer control to the management
runtime of the isolation environment. The strong hardware isolation guaranteed
that even a malicious operating system could not access the memory of the
isolated task. A downside to this approach is that it works only on a small
amount of hardware. The implementation depends on the resizeable SMRAM to react
to the growing memory demands of applications isolated through SICE. As the
authors mentioned, they used AMD platforms for their implementation because of
this. Intel platforms did not support this feature, and adapting SICE to those
platforms was left as an open problem. Moreover, firmware modifications have to
be implemented by the user to install the correct \gls{smi} handler. These
modifications require the firmware to be open source to implement the \gls{smi}
handler. This requirement further reduces the amount of hardware used for this
approach. Moreover the authors state that practical deployment would require CPU
and hardware vendors to do extensive security reviews on their firmware and
\gls{smm} implementations. Additionally, multiple attacks are known targeting
the \gls{smm} breaking its isolation~\cite{wojtczuk2014attacking,
wojtczuk2009attacking, wojtczuk2009poisining}.

\subsection{Defenses Against Interrupt Based Side Channel Attacks}
\label{sec:20:mitigations:interrupt_sca}
To defend against interrupt-based side-channel attacks, Cui et al. proposed a
defense solution in 2023 that they call QuanShield~\cite{cui_quanshield_2023}.
QuanShield goal is to enable \gls{sgx} enclaves to detect interrupt-based
side-channel attacks and react adequately. For this, QuanShield isolates a CPU
core from the system to let it run an Intel \gls{sgx} enclave. The goal of the
isolation is to prevent the scheduler from interrupting the isolated core
because no other workload is to be executed by this core. The authors turned off
all other interrupts as far as possible and used a Linux kernel that runs in
tickless mode to ensure that the kernel does not send scheduling interrupts to
the isolated core. The authors built tickless kernels by using the kernel
KConfig option \textit{CONFIG\_NO\_HZ\_FULL=y}. In this mode, the kernel does
not send scheduling interrupts to cores that are either idle or for which only
one task is ready~\cite{linuxtickless}. All remaining interrupts are considered
to be attacks on the enclave. QuanShield uses unused parts of the state save
area to terminate the enclave upon receiving any interrupt. The state save area
is protected enclave memory in which the CPU stores its state on context
switches, e.g., when stopping to execute enclave code. QuanShield stores
non-canonical memory addresses in these unused parts. Once the control returns,
the enclave code uses one of these non-canonical addresses, which results in a
CPU exception and leads to the termination of the enclave, effectively stopping
an attack using interrupt based side channels.\\

QuanShield uses code instrumentation to make the enclave use one of the
manipulated addresses. For this, the authors added code to the LLVM compiler.
The compiler introduces load and store operations on each function entry to make
the code fault as fast as possible. QuanShield uses a library \gls{os} to
support legacy applications. It implements the protection mechanism by utilizing
SGX-LKL to manage the second stack in the state save area. SGX-LKL is a Linux
kernel port that can run in an \gls{sgx} enclave as a LibraryOS, similar to the
approach used by SCONE and Haven to mitigate effects of Iago attacks, which I
just want to name as a side node because these solutions are less relevant for
my work~\cite{priebe2019sgx,arnautov_scone_2016,baumann_shielding_2015,
checkoway2013iago}.

\subsection{Transient Execution Mitigations}
\label{sec:20:def_sca}
Because Spectre and Meltdown target microarchitectural behavior, a complete
redesign is necessary to entirely fix the issue. Software mitigations are
available for specific attacks. For example, the Linux Kernel uses techniques
called Retpoline and Kernel Page Table Isolation (KPTI) to mitigate Spectre
version 2 and Meltdown, respectively~\cite{retpoline}. On the other hand,
software mitigations can greatly impact performance, reaching from a 10\% to
800\% overhead, depending on the workload~\cite{low2018overview}.\\

Another approach for systems to defend against side-channel attacks, in general,
is active detection of the attack and reacting appropriately. Quanshield
implements such a solution for interrupt-based side channel attacks described in
chapter~\ref{sec:20:interrupt_sca}. For attacks abusing transient execution,
this approach of deactivating transient execution would come with a
high-performance penalty.\\

\section{Utilizing Performance Monitoring Counters}
\label{sec:20:pmc}
Instead, existing solutions attempt to monitor the cache and other
microarchitectural behavior through \gls{pmc} to find any anomaly. The idea
behind this approach is that a program shows specific behavior. This behavior
results from the instructions it executes and their order, which leaves a kind
of footprint. When monitoring the hardware performance counters closely enough,
the observer can deduce what parts of code have been executed by the CPU. If the
performance counter traces of the program are known beforehand, an Observer can
compare the values of the counters with the known state and then reason if the
control flow was highjacked, for example, by an ROP attack. As a side note,
monitoring through hardware Performance counters can also be used for attacks,
which is why performance counters are unavailable while the CPU operates in
\gls{sgx} mode~\cite{uhsadel2008exploiting,costan2016intel}.\\

Like ROP attacks, transient execution attacks show special behavior when
preparing the attack or side channel. Li et al. and Van Bulck et al. examined
how to trace the behavior of transient execution attacks with performance
counters on the examples of Spectre and Load Value injection attacks,
respectively~\cite{li_detecting_2021, van_bulck_lvi_2020}. They found that when
in the preparation phase, while the attacker trains the branch predictor, fewer
instructions are retired compared to typical workloads. Moreover, employing the
cache side channel leaves traces, too. For a cache-based side-channel to work,
the attacker tries to evict pages that map to addresses they want to use for the
attack. This results in a high amount of \gls{tlb} flushes. Later, addresses are
accessed by the side channel code to retrieve information. Because the cache was
flushed, the number of cache misses increased significantly. An observer can
detect all of these side effects by using performance counters. Still, an
attacker can hide their activities by slowing down their attack. While the total
number of cache misses induced by the attack, for example, does not change this
way, the attack distributes the misses more evenly over time. Because events
such as cache misses and \gls{tlb} flushes are normal behavior of systems
running a general purpose \gls{os}, the environment in which an attack runs
introduces noise that can help to hide an attack. Thus, detecting said attack
becomes nearly impossible if an attacker distributes the effects of their attack
over time. Consequently, the results of Li et al. and Van Bulck et al. lead us
to conclude that a detection approach using hardware performance counters in the
presence of other software is unreliable. Kosasih et al. came to a similar
conclusion in their survey of knowledge in 2024~\cite{kosasih2024sok}.

\section{Summary}
\label{sec:20:summary}
We have seen examples for \glspl{tee}, attacks on \glspl{tee}, possible
mitigations to those attacks and solutions that try to enhance existing
\gls{tee} implementations. \\

All \gls{tee} solutions have in common that they do not implement the whole
functionality exclusively in hardware. For example, Intel and AMD employ
dedicated security processors in their x86 \glspl{soc} as a hardware trust
anchor and implement the remaining \gls{tee} functionalities of \gls{sgx} and
\gls{sev} in CPU micocode/firmware and software modules, respectively. Together
with Enma, which uses the \gls{tpm} as a root of trust, these solutions
implement a signature-based remote attestation scheme. Next, they all share the
property of being vulnerable to side-channel attacks, with Spectre and Meltdown
being rather prominent. These attacks abuse observable (micro-) architectural
side effects to leak secrets through covert channels. Mitigations for some
attacks exist, but they impact performance or cannot be applied to all
side-channel attacks. Additionally, systems can observe these side effects
through \glspl{pmc} to some degree, but as we have seen in
section~\ref{sec:20:def_sca}, attacks can avoid being detected through reducing
their throughput or simply if the noise in the system is too high. With
QuanShield interrupt side channels were mitigated by Cui et al. by creating an
interrupt-free environment, which at least mitigates this attack vector for
\gls{sgx}.\\

Putting these together, I identified an approach that was not tested to the best
of my knowledge. This excludes architectural side channels by creating a
completely isolated execution environment that only uses core local resources
and monitors its integrity by utilizing \glspl{pmc}. While I will follow Enmas
approach of using the \gls{tpm} as a trust anchor, I will not use any vendor
specific \gls{isa} extensions to overcome vendor locking, problems present in
\gls{sgx}, \gls{sev} and \gls{tdx}. How to design such an \gls{tee} will be the
topic in chapter~\ref{sec:design}.

%%% Local Variables:
%%% TeX-master: "diplom"
%%% End:
