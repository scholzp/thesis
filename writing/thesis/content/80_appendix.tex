\chapter{Source Code}
% \ļabel{sec:appendix}

The source code and building instructions for Linux are publicly available at
GitHub. For reasons of reproducibility, I also provide the source code I used to
create measurements presented in chapter~\ref{sec:evaluation}. Each test
scenario can be found on a dedicated branch.\\
\todo{Räume mal die Repo URL auf}
Additionally, the source code of the Linux kernel module can be found as listed
below. With the help of the build scripts in this repository any person
interested can setup a Qemu based test environment, as well as images for
running a Linux kernel next to TEECore on bare metal.

\subsubsection{Main Branch of TEECore}
\begin{quote}
  \url{https://github.com/scholzp/phipsboot/tree/main}
\end{quote}

\noindent
This repository contains all Code of TEECore. The main branch contains code that
implements the reset procedure described in
chapter~\ref{sec:implementation:teeKernel}.

\subsubsection{SC1 Branch of TEECore}
\begin{quote}
  \url{https://github.com/scholzp/phipsboot/tree/mem_eval_sc1}
\end{quote}

\noindent
This branch contains all Code of TEECore used for SC1 in
chapter~\ref{sec:evaluation}.

\subsubsection{SC2 Branch of TEECore}
\begin{quote}
  \url{https://github.com/scholzp/phipsboot/tree/mem_eval_sc2}
\end{quote}

\noindent
This branch contains all Code of TEECore used for SC2 in
chapter~\ref{sec:evaluation}.

\subsubsection{Linux Kernel Module and Test Environment}
\begin{quote}
  \url{https://github.com/scholzp/thesis/tree/impl}
\end{quote}

\noindent
This branch contains all Code to build a Qemu based test environment, as well as
scripts to create images bootable on bare metal. In the test environment,
TEECore is run next to Linux, as described in~\ref{sec:30:tee_general}.
