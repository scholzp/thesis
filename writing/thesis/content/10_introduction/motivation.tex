\section{Motivation}
\label{sec:10:motivation}
As of the beginning of 2025, the service haveibeenpawned registered over 14
billion breached user accounts with personal data such as date of birth, IP
addresses, and names on over 855 websites.\cite{haveibeenpawned} Criminals can
use disclosed personal data, for example, for fraud, social engineering, and
identity theft. Furthermore, even without criminals, protecting privacy can be
of great value.\cite{solove2007ve} In an ideal world, the user would not only be
able to transmit encrypted data to a service but also to verify how the provider
processes their data. In extreme cases, the user even requests proof that the
server's operator cannot spy on the processing routine. A technology allowing
these verification forms is called a trusted execution environment (TEE).
Trusted execution environments have an isolation mechanism to defend against
spying and modifying attacks. Moreover, they implement protocols to verify those
isolation properties towards an appraiser over the network, a technology
referred to as remote attestation. \\

A tangible example of how remote attestation can help solve real-world problems
is the implementation of Signal's automatic contact discovery service. The
Signal Messenger aims to provide a privacy messaging service. It implements its
own end-to-end encryption protocol to protect its users'
communication.\cite{cohn2020formal} But communication alone is not all of
Signals features. The history of PGP has shown that virtually no one uses
privacy-enhancing technologies if their usability is low and their application
is too complicated.\cite{ruoti2015johnny} It is therefore not surprising that
Signal also aims to provide a good user experience. To do so, the developers of
Signal saw themself confronted by the problem of implementing a
privacy-preserving way for automatic contact discovery.\cite{SignalCd}\\

Traditionally Signal does not use phone numbers. Instead, phone numbers are used
as input to a hash function to generate user identifiers. The Signal client on
the users device generates these hashes, truncates them and transmits the
results to the server. Signal so does not learn the phone numbers of it's users
this way.\todo{Soll hier 'ne Sidenote rein, dass der Wertebereich des inputs
    klein ist und damit theortisch einfaches bruteforce möglich ist?}
\todo{bis hier ist neu}
When creating the contact discovery service, the implementers were confronted
with multiple problems. The first major problem was the social graph. The social
graph can either be centralized and stored on Signal's servers or decentralized
and stored on the user's phone. To minimize data stored on Signal's servers
about their users, the implementers decided to use the users' phonebooks stored
on their phones. The second problem arises from this decision. The user's phone
books must be processed so that neither Signal nor their server operators can
learn anything about the users. To solve these problems, encrypting the phone
books is not enough. For example, the user does not have any chance to conclude
from the communication alone with whom they are communicating. Even if the
communication partner proves with the help of signatures that they are the
Signal server application, the user cannot verify what version of the server
application is running or if it processes the data as expected. Until now, we
have not considered the server operator, which might be malicious, too. They can
manipulate and read the application's memory because they have privileged access
to the server's software and hardware. Such a powerful adversary could read
decrypted secrets directly from the application's memory or modify the server
application in a way that leaks secret data. In this example, the user wants to
verify that the Signal server application adheres to the following claims:
\begin{enumerate}
    \item The Signal server application executes the code the user expects
    \item The Signal server application is safe from being manipulated by
          privileged third parties
    \item Privileged third parties are unable to read the server applications'
          memory
\end{enumerate}
The server environment, therefore, needs a trusted security authority that can
prove these claims to the users. Additionally, hardware isolation mechanisms
must protect this security authority against tampering attempts of privileged
third parties. Otherwise an attacker can influence the operation of the trusted
third party or fake it completely. Trusted execution environments solve this
problem by integrating two building blocks. First, they can execute code in an
isolated execution environment. Hardware enforces isolation, and interaction
with the isolated code is only possible through defined interfaces of the
isolating hardware. This way, privileged software is not able to access
memory-isolated code. Additionally, to protect against memory bus-taping
attacks, isolation hardware often encrypts the memory. The second building block
implements remote attestation. Through remote attestation, TEEs can verify that
the execution environment runs the task under hardware-assisted isolation
protection. Furthermore, the TEE can attest to the state of the running
software, which the appraiser can use to compare the attestation result to a
state expected by the users. If the TEE fulfills the aforementioned claims, the
program running under isolation is safe, and the appraiser can construct a
secure communication channel.\\

Most commodity CPUs currently implement technologies that enable the creation of
TEEs. In that case, the CPU and its vendor act as security
authorities.\cite{tdx_whitepaper,kaplan_amd_2020,pinto_demystifying_2019,costan2016intel}
The story could be finished at this point if it was not the case that the
respective CPU implementations contain bugs that make it possible for attackers
to circumvent their hardware-assisted isolation mechanisms and leak secrets
through side
channels.\cite{kocher_spectre_2020,lipp_meltdown_2020,nilsson_survey_2020} While
software updates could mitigate specific side-channel attacks, the source of the
information leaks is still persistent in hardware, with researchers still
finding new side-channel attacks each
year.\cite{wikner2022retbleed,moghimi_downfall_2023,ragab_ghostrace_2024} On the
other hand, the example of Signal elucidates that proper working TEEs are
important because they might process critical data. It is, therefore, crucial to
find a way to allow the execution of programs in a completely isolated and
side-channel-free environment. A problem that I try to solve in this thesis.
