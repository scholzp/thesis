\section{Motivation}
\label{sec:10:motivation}
As of the beginning of 2025, the service haveibeenpawned registered over 14
billion breached user accounts with personal data such as date of birth, IP
addresses, and names on over 855 websites~\cite{haveibeenpawned}. Criminals can
use disclosed personal data, for example, for fraud, social engineering, and
identity theft. Furthermore, even without criminals, protecting privacy can be
of great value. After all, privacy also concerns how data is collected and
processed. While single pieces of information about an individual might not tell
much on their own, these pieces can be linked together to form a more
comprehensive picture of said individual~\cite{solove2007ve}. In an ideal world,
the user would not only be able to transmit encrypted data to a service but also
to verify how the provider processes their data. To increase privacy even
further, a user can request a proof that the server's operator does not spy on
the processing routine. A technology allowing these verification forms is called
a \gls{tee}. Trusted execution environments have an isolation mechanism to
defend against spying and modifying attacks. Moreover, they implement protocols
to verify those isolation properties towards an relying party over the network,
a technology referred to as remote attestation. \\

A tangible example of how remote attestation can help solve real-world problems
is the implementation of Signal's automatic contact discovery service. The
Signal Messenger aims to provide a privacy messaging service. It implements its
own end-to-end encryption protocol to protect its users'
communication~\cite{cohn2020formal}. However, communication alone is not all of
Signals features. The history of PGP has shown that virtually no one uses
privacy-enhancing technologies if their usability is low and their application
is too complicated~\cite{ruoti2015johnny}. It is therefore not surprising that
Signal also aims to provide a good user experience. To do so, the developers of
Signal saw themself confronted by the problem of implementing a
privacy-preserving way for automatic contact discovery~\cite{SignalCd}.\\

Consequently, Signal does not store phone numbers on their servers to identify
their users. Instead, users' phone numbers are processed by their devices to
create pseudonyms. These pseudonyms are the result of a hash function, called
hashes. The Signal client on the users' device generates these hashes, truncates
them, and transmits the results to the server. The sever can then compare these
pseudonyms with its database of registered users to connect the right contacts.
In this way, Signal does not learn the phone numbers of its users, but is able
to connect registered users with each other.\\

When creating the contact discovery service, the implementers faced multiple
problems. The first major problem was the requirement for a social graph. A
social graph connects people based on their relationships in a network. In the
example of Signal, a social graph would at least contain information about who
communicates with whom. The social graph can either be centralized and stored on
Signal's servers or decentralized and stored on the users' phone. To minimize
data stored on Signal's servers about their users, the implementers decided to
use the users' phonebooks stored on their phones. The second problem arises from
this decision. The users' phone books must be processed in a way so that neither
Signal, through their services, nor their server operators can learn anything
about the users. To solve these problems, encrypting the phone books is not
enough. For example, the user does not have any chance to conclude from the
communication alone with whom they are communicating. Do they communicate with
the Signal server, or are they a victim of a man-in-the-middle attack? Even if
the communication partner proves with the help of signatures that they are the
Signal server application, the user cannot verify what version of the server
application is running or if it processes the data as expected. Until now, we
have not considered the server operator, who might be malicious, too. They can
manipulate and read the application's memory because they have privileged access
to the server's software and hardware. Such a powerful adversary could read
decrypted secrets directly from the application's memory or modify the server
application in a way that leaks secret data. In this example, the user wants to
verify that the Signal server application adheres to the following claims:
\begin{enumerate}
  \item The Signal server application executes the code the user expects
  \item The Signal server application is safe from being manipulated by
    any other party
  \item Any privileged party is unable to read the server application's
    memory
\end{enumerate}
The server environment, therefore, needs a trusted security authority that can
prove these claims to the users. Additionally, hardware isolation mechanisms
must protect this security authority against tampering attempts of privileged
parties. Otherwise, an attacker can influence the operation of the trusted party
or fake it completely. Trusted execution environments solve these problems by
integrating two building blocks. First, they can execute code in an isolated
execution environment. Hardware enforces isolation, and interaction with the
isolated code is only possible through defined interfaces of the isolating
hardware. This way, privileged software is not able to access memory-isolated
code. Additionally, to protect against memory bus-tapping attacks, isolation
hardware often encrypts the memory. The second building block implements remote
attestation. Through remote attestation, \glspl{tee} can verify that the
execution environment runs the task under hardware-assisted isolation
protection. Furthermore, the \gls{tee} can attest to the state of the running
software, which the relying party can use to compare the attestation result to a
state expected by the users. If the \gls{tee} fulfills the aforementioned
claims, the program running under isolation is safe, and the relying party can
construct a secure communication channel.\\

So, what does this mean concerning the Signal example? How can \glspl{tee} help
in this example? With the assumption that Signal knows its registered users, the
blog post describes a recipe, which I cite in a simplified form:
\begin{enumerate}
  \item The encrypted contact discovery request containing the users' phonebook
    is transmitted from the user to the \gls{tee}
  \item The \gls{tee} decrypts the request and compares the phone numbers from
    the request with its database of registered users
  \item The \gls{tee} builds the response message from the comparison results
    and transmits the encrypted result back to the user
\end{enumerate}
For this, Signal only needs to know its registered users. It does not need to
store the users' phonebooks on its servers to provide the contact discovery
service. Neither Signal nor their server operators learns anything about the
users' phonebooks.

Most commodity CPUs currently implement technologies that enable the creation of
\glspl{tee}. In that case, the CPU and its vendor act as security
authorities~\cite{tdx_whitepaper,kaplan_amd_2020,pinto_demystifying_2019,costan2016intel}.
These technologies theoretically solve many problems around \glspl{tee}, but are
known to be insecure because respective CPU implementations contain bugs that
make it possible for attackers to circumvent hardware-assisted isolation
mechanisms and leak secrets through side
channels~\cite{kocher_spectre_2020,lipp_meltdown_2020,nilsson_survey_2020}.
While software updates could mitigate specific side-channel attacks, the source
of the information leaks is still persistent in hardware, with researchers still
finding new side-channel attacks each
year~\cite{wikner2022retbleed,moghimi_downfall_2023,ragab_ghostrace_2024}. On
the other hand, the example of Signal elucidates that proper working \glspl{tee}
are important because they might process critical data. In this thesis, I am
going to explore how to guarantee side-channel free execution of software in a
\gls{tee} by restricting the run time environment to core local resources.
Furthermore, I am going to explore a mechanism that leverages performance
measurement counters to monitor any information breaches.
