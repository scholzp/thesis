\section{Structure}
\label{sec:10_structure}
In the following section, I want to give an overview of the content of this work and how
it's structured.\\

In chapter~\ref{sec:state}, I want to give some background information and
introduce concepts to help us understand the remaining parts of my
thesis. First, in section~\ref{sec:20:technical}, I want to explain some
technical details about the x86 \gls{isa}, which are important to understand the
design decisions and implementation from a technical background. This includes
the different operation modes of x86\_64 processors, information on how
interrupt delivery works, how memory is managed using paging, and how caches
work. Moreover, I will give insight into how \glspl{pmc} work and how they are
programmed, as they form the technical backbone of my detection routine.

Following the technical background of the x86\_64 \gls{isa} I want to explain
some building block for \glspl{tee} (section~\ref{sec:20:chain_of_trust}) before
we review selected \gls{tee} implementations in section~\ref{sec:state:tee},
which serve as inspiration for my work.

These solutions have shortcomings, so it's important to see under
what conditions they fail to uphold their security promises. For this, I want
to explain selected attacks on \glspl{tee} in section~\ref{sec:20:attacks} and
in section~\ref{sec:20:mitigations} approaches to mitigate them or detect them.

After this, I summarize chapter~\ref{sec:state} in
section~\ref{sec:20:summary}.\\

Chapter~\ref{sec:design} will give insights into my design decisions for
the prototype implementation. Because the prototype is no standalone solution, I
will first explain the environment in which the prototype will operate and how
it interacts with other system components (c.f.
section~\ref{sec:30:environment}). Section~\ref{sec:30:tee_kernel} will then
describe the \gls{tee} component and its design in more detail before we
see how different attacks could mounted against the design in
section~\ref{sec:30:attack}.\\

Once we learn about the design, chapter~\ref{sec:40} will give
details about the implementation and some of its highlights. First, the kernel
used in the implementation and what modifications were made will be
explained in section~\ref{sec:implementation:teeKernel}. Next, in
section~\ref{sec:implementation:hostKernel}, I explain how Linux is used in the
system as the host kernel. Because Linux cannot communicate
out-of-the-box with my prototype, I programmed a kernel module that serves as a
driver. Implementation details on this module will be explained in
section~\ref{sec:implementation:kmod}. The last part of chapter~\ref{sec:40}
consists of details about applications and attacks that I use to demonstrate the
security properties of my prototype and benchmark different characteristics.
Those can be found in section~\ref{sec:implementation:attacks}. \\

In the evaluation chapter~\ref{sec:evaluation}, I will evaluate the prototype
regarding its defense probability against the attacks implemented in
section~\ref{sec:implementation:attacks}. Moreover, I measure the memory
requirements of the \gls{tee} runtime component of the prototype to conclude
what resources are left for payload tasks. \todo{add more about evluation}

\todo{add more about furuture work}
\todo{add more about conclusion}
