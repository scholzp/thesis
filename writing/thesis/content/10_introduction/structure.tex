\section{Structure}
\label{sec:10_structure}
In chapter~\ref{sec:state}, we will discuss technical Grundlagen that is
important to follow the rest of the work. The first part ~\ref{sec:20:technical}
we will discuss some details of the x86 microarchitecture. These details will
allow us to understand the attacks and what countermeasures to implement in the
remaining parts of the thesis. Section~\ref{sec:20:tee} will then discuss the
state of the art by reviewing related work. In this, we will look at how trusted
execution environments are implemented. First, we will review hardware
extensions followed by software solutions or software that enhances the security
of hardware extensions.

In chapter~\ref{sec:design}, I will explain the design of my implementation. In
particular, we will explain in~\ref{sec:30:system} the general design.
In~\ref{sec:30:attacker}, we will revisit the attacker model and what design
feature can be used to mitigate which attack.

Chapter~\ref{sec:implementation} will explain how I implemented the design
proposed in~\ref{sec:design}. We will look at specific parts of the
implementation that are important to fulfill the security guarantees. Moreover,
we will investigate what part of the design was problematic to implement.

In chapter~\ref{sec:evaluation}, we will revisit the implementation described in
chapter~\ref{sec:implementation}. We will evaluate the implementation in
chapter~\ref{sec:50:security} regarding how it protects against the attacker
model described in~\ref{sec:30:attacker}. Moreover, we will evaluate the
practical use of our design in~\ref{sec:50:security}.

After we evaluate the implementation, we will look at missing parts and how to
improve the implementation further. We will do all of this in
section~\ref{sec:futurework}, which gives an outlook on future work.

To come to terms with our results, we will give a short overview and summary of
the results and findings of our work in chapter~\ref{sec:conclusion}.
