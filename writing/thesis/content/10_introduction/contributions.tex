\section{Contributions}
\label{sec:10:contributions}
As described in the motivation (see chapter~\ref{sec:10:motivation}) chapter of
this thesis, research shows that side channel attacks impose a big threat to
current \glspl{tee} implementations. As such, the undermine security guarantees
of such solutions (c.f. chapter~\ref{sec:20:attacks}). To significantly reduce
the impact of side channels, I propose TEECore. TEECore is a software based
\gls{tee} that achieves isolation through levering core local resources. TEECore
tackles the following goals:

\begin{enumerate}
  \item I create an execution environment isolated from the remaining system by
    utilizing features found in all commodity CPUs. It is independent from
    vendor specific \gls{tee} solutions and the concept is applicable to at
    least all commodity x86 CPUs.
  \item I show how to isolate a single CPU core from a commodity \gls{os} to
    host TEECore and that TEECore is able to run next to such an \gls{os} and
    provide services to it, achieving high isolation in this way from the
    commodity \gls{os}. I show how to implement a shared memory communication
    channel for data exchange and communication.
  \item With TEECore, I propose a \gls{tee} design to detect side-channel
    attacks and react appropriately.
  \item My design ensures the usage of only core local resources. This means I
    implement a policy that is restricting execution to core local caches.
  \item I evaluate ways to enable such an execution environment to offer
    remote attestation features.
  \item I evaluate my implementation regarding system resources. In detail, I
    will evaluate what workload fits next to the runtime environment in the
    core's local caches.
  \item I evaluate against what attacker model TEECore, as a sole software
    solution, can defend.
\end{enumerate}

With complete isolation and the exclusion of possible side channels, I aim for a
\gls{tee} solution that can withstand various attacks. I show the properties and
concepts on a working proof of concept implementation. My solution can run next
to a commodity operating system such as Linux. The prototype implements
communication channels through a well-defined interface that does not influence
the security properties of my design, and I show that the prototype can protect
itself from being interfered with by the commodity \gls{os}. My solution still
depends on the commodity \gls{os} to install the environment. Therefore, the
prototype will not be a standalone solution but a technology demonstrator.
Designing system components in a way that makes TEECore a standalone component
is not the scope of this work and is left for future work.
