\section{Contributions}
\label{sec:10:contributions}
As described in the motivation (see chapter~\ref{sec:10:motivation}) for this
thesis, I see a big problem in the general vulnerability by side channel
attacks on commodity processors. To fully exclude side channels, I propose
<Fancy Name\todo{think about a really fancy name}> to tackle the following
goals:

\begin{enumerate}
    \item I create an execution environment fully isolated from the remaining
          system by utilizing features found in all commodity CPUs. This means
          that I do not use any vendor-specific ISA extensions.
    \item I propose a TEE design to detect side-channel attacks and react
          appropriately.
    \item Our design ensures the usage of only core local resources. This means
          I implement a policy that is restricted to core local caches.
    \item I evaluate ways to enable such an execution environment to offer
          remote attestation features.
    \item I evaluate our implementation regarding system resources. In detail,
          I will evaluate what workload fits next to the local cache's runtime
          environment.
    \item I evaluate the TCB and against what attacker model our sole software
          solution can defend.
\end{enumerate}

With complete isolation and the exclusion of possible side channels, I aim for
a TEE solution that can withstand a wide array of possible attacks. Because of
time constraints in this thesis, I cannot implement a fully working TEE ;
instead, I can show the properties and concepts on a working proof of concept
implementation. I show that our solution can run next to a commodity operating
system such as Linux, implement communication channels through a well-defined
interface, and show that I can protect our implementation from being interfered
with by the commodity OS. Because I explore the technical feasibility, our
solution still depends on the commodity to install the environment.
