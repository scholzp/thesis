\section{Contributions}
\label{sec:10:contributions}
As described in the motivation (see chapter~\ref{sec:10:motivation}) for this
thesis, I see a big problem in the general vulnerability by side channel
attacks on commodity processors. To fully exclude side channels, I propose
TEECore to tackle the following
goals:

\begin{enumerate}
    \item I create an execution environment fully isolated from the remaining
          system by utilizing features found in all commodity CPUs. The backend
          implementation might still be vendor-specific because those features
          might be specific to the respective CPU implementation. The overall
          concept stays the same on all CPUs. I show, that once running, system
          software cannot influence my design.
    \item I propose a \gls{tee} design to detect side-channel attacks and react
          appropriately.
    \item My design ensures the usage of only core local resources. This means
          I implement a policy that is restricted to core local caches.
    \item I evaluate ways to enable such an execution environment to offer
          remote attestation features.
    \item I evaluate my implementation regarding system resources. In detail,
          I will evaluate what workload fits next to the runtime
          environment in the core's local caches.
    \item I evaluate the \gls{tcb} against what attacker model my sole software
          solution can defend.
\end{enumerate}

With complete isolation and the exclusion of possible side channels, I aim for a
\gls{tee} solution that can withstand various attacks. I show the properties and
concepts on a working proof of concept implementation. My solution can run next
to a commodity operating system such as Linux. The prototype implements
communication channels through a well-defined interface that does not influence
the security properties of my design, and I show that the prototype can protect
itself from being interfered with by the commodity \gls{os}. My solution still
depends on the commodity \gls{os} to install the environment. Therefore, the
prototype will not be a standalone solution but a technology demonstrator.
Designing systems components in a way that makes my implementation a standalone
component is not the scope of this work and is left for future work.
