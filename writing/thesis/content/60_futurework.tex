\chapter{Future Work}
\label{sec:futurework}

\section{Missing Features}
The prototype of TEECore is missing several features described in
chapter~\ref{sec:design}, that would need to be implemented to use it to its
full potential.

The first feature missing is domain separation between the secure applications
and the TEECore kernel. Currently, secure applications are run with the same
privileges as TEECore, which is a potential security issue. Because TEECore is
designed to operate in an interrupt-free environment, the transition between
different privilege levels would need to be implemented using the x86\_64
specific \textit{syscall} and \textit{sysret} instructions.

Second, a secure channel would need to be established to exchange data between
secure applications and applications in the normal partition. The shared memory
communication path can be used for such a channel as a medium. To allow secure
communication via this channel, TEECore would need to implement some key
exchange algorithm for key exchange with a remote party and consequent
encryption of the shared memory.

Third, TEECore currently only works with tasks that are directly embedded as a
function into TEECore's code. This makes it complicated to adopt TEECore to new
tasks. To support easier adoption, TEECore should be able to load tasks from
binaries, such as \gls{elf} files. Such functionality would require the
implementation of a binary loader together with infrastructure to support
dynamic binary loading. Alternatively, TEECore could support linking of static
libraries to ship applications as binaries. Enhancements to the build system
and a clear API for application would then be required.

Moreover, the prototype only allows communication with one remote party, as it
does not enforce exclusive usage of the shared communication path. To allow
multiple applications to use TEECore's features, a more complex communication
protocol would need to be implemented.

Lastly, TEECore lacks any remote attestation feature. I sketched the general
design in chapter~\ref{sec:30:tee_ra_scheme}. TEECore must implement a \gls{tpm}
driver and measurement facilities to implement this feature. The required
measurement facilities include those for measuring the secure applications code,
the secure application communication, and the general state of TEECore.

\section{Other Platform Support}
Currently, TEECore is only implemented for one Intel platform. This platform
implements inclusive caches. An open question at this point is how TEECore
behaves on architectures implementing different cache coherency protocols. For
example, AMD processors implement the MOESI protocol, which would allow for a
comparison of how TEECore's approach performs in such an architecture.
Additionally, I would like to evaluate AMD processors regarding their
performance measurement facilities.

\section{Additional Features}
As mentioned in chapter~\ref{sec:evaluation:ipi}, the TEECore prototype
implementation is vulnerable against \glspl{ipi}. Combining TEECore with other
\gls{tee} technologies could help mitigate this flaw and increase the security
of these other technologies. So, as a possible research area, TEECore could be
combined with one of these technologies to increase overall security. Moreover,
other processor features could be used to further increase isolation. For
example, Intel offers a technology called cache allocation technology, which can
be used to partition L3 caches via quality of service. Such a technology could
increase resources available to TEECore by disallowing other cores or
applications to access cache lines allocated to TEECore. Some AMD processors
implement a similar feature that allows for the specification of quality of
service constraints for cache usage. Further, TEECore could use undocumented
features such as cache-as-ram (CAR). Finding ways to utilize a dynamic root of
trust for measurement would reduce TEECore's \gls{tcb}.

%%% Local Variables:
%%% TeX-master: "diplom"
%%% End:
