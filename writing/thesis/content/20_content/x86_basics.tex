\section{Technical Background}
\label{sec:state:technical}
This chapter gives us an overview of important mechanisms used by modern CPUs.
While there are other widely used architectures, such as ARM on mobile devices,
we explore these features in the example of the x86\_64 architecture. \\

I decided to do so because my proof of concept implementation targets the
x86\_64 architecture. Moreover, behavior and implementations of similar concepts
can differ between \glspl{isa} and even implementations of the
same \gls{isa} on the microarchitectural plane. I, therefore, note
differences in implementations of x86\_64 whenever appropriate.

\subsection{Operation Modes}
\label{sec:state:technical:modes}
The x86 architecture is rooted in the Intel 8086, designed in 1978. As a 16-bit
microprocessor design, the Intel 8086 physically can only address 64 KiB of
memory, which was enlarged by two segment registers to allow addressing nearly 1
MiB of memory. To maintain compatibility with the Intel 8086, later 32-bit
designs of the x86 \gls{isa} introduced an operation mode called \gls{g_rmode}.
The name \gls{g_rmode} originates from the fact that the 16-bit design used the
real location in memory for addressing. The 32-bit operation mode of this CPUs
is called \gls{g_pmode}, because it allows addressing memory through virtual
addresses, allowing for memory protection. All x86 processors boot in
\gls{g_rmode} to maintain compatibility to legacy software originally written
for 16-bit CPUs. With the 64-bit extension x86\_64, AMD introduced an operation
mode called \gls{g_lmode} that consists of two sub-modes called Compatibility
Mode and 64-bit mode. Compatibility mode allows 64-bit system software to
execute legacy 32-bit software. 64-bit mode extends the \gls{isa} by 64-bit
operands and addresses, adds eight new general purpose registers and additional
instructions. To use \gls{g_lmode}, system software has to prepare the processor
by enabling additional processor features and creating required data structures.
Because of the limitation to the first 1 MiB of memory, code that transfers a
processor from \gls{g_rmode} to any other operation mode has to reside in the
address range of 0x500 to 0x7FFFF. This address range is called low memory and
free for usage by system software.

\subsection{System Management Mode}
\label{sec:state:technical:smm}
Contrary to the operations modes described in
section~\ref{sec:state:technical:modes}, \gls{smm} does not activate additional
processor features. Instead, it is a hardware-assisted isolation mechanism that
protects firmware code from system software. In legacy systems, firmware used
the \gls{smm} to react to special hardware events that required the firmware to
react to platform events, e.g., energy management. The \gls{smm} code resides in
a specially protected area of the main memory, referred to as \gls{smram}.
\gls{smm} executes with the highest privileges in x86, and system software
cannot interfere with it. In fact, \gls{smm} can interrupt systems software
whenever necessary, not vice versa.

\subsection{Interrupts and Exceptions}
\label{sec:state:technical:interrupts}
Interrupts and Exceptions are used to call system-specific functions and respond
to special conditions in the CPU or system. Exceptions are raised by the CPU
upon executing software or detecting hardware errors. Interrupts, on the other
hand, are either the result of software interrupt instructions or signals
generated by external hardware, such as keyboard input. Exceptions can be
divided into three types by their origin or if they allow to restart the causing
instruction:
\begin{enumerate}
  \item Faults: Result of an error with the instruction to execute
  \item Traps: Result of breakpoint and software interrupt instructions
  \item Aborts: The causing instruction cannot be restarted
\end{enumerate}
Interrupts, on the other hand, can be divided into two classes:
\begin{enumerate}
  \item Maskable Interrupts: Can be masked. Masking an interrupt means that
    software can temporarily disable them. The interrupt controller holds
    them back until interrupts are enabled again.
  \item \Gls{nmi}: Software cannot turn off \glspl{nmi}. The interrupt
    controller delivers~ \glspl{nmi} to the CPU unless it currently serves
    another\gls{nmi}. When the CPU executes the IRET instruction in the
    interrupt handler, the interrupt controller can deliver the next \gls{nmi}.
\end{enumerate}

The x86 \gls{isa} assigns a vector number to each interrupt or exception. Vector
numbers range from 0 to 255 in newer implementations. The CPU uses the vector
number of an interrupt as an index to locate the respective handler function in
a data structure called the \gls{idt}. The \gls{idt} resides in the main memory
and contains the address of an interrupt handler routine for each vector. System
software defines the handler routines and writes their addresses to the
\gls{idt}. Once system software defines the handler routines, it makes the
\gls{idt} active by writing its address to the \gls{idtr}. System software
cannot write directly to the \gls{idtr} but has to prepare the special memory
descriptor from which the LIDT instruction loads the \gls{idt}. After setting up
and loading the \gls{idt}, the CPU executes interrupt handlers as defined by
system software. System software can enable maskable interrupts by executing the
\textit{STI} instruction.\\

Over time, interrupt controllers were improved and adapted to new use cases
similar to CPUs. In \gls{g_rmode}, the CPU falls back to using the Intel 8259 or
a compatible programmable interrupt controller (PIC), that delivers
interrupts to the CPU. Once the PIC delivers an interrupt to the CPU, the system
software must serve it and signal the PIC with an \gls{eoi}, that the serving
routine is done. \\

\begin{figure}
  \begin{center}
    \includestandalone{images/lapic.tex}
    \caption{Illustration of how \glspl{lapic} integrate in a
    multiprocessor system (Figure after \cite{amd_manual}, p. 620)}
    \label{fig:state:technical:lapic}
  \end{center}
\end{figure}

Intel later introduced the advanced programmable interrupt controller (APIC)
with its 486 processor line to allow the system to operate with multiple CPUs.
In modern CPUs, multiprocessor configurations are prevalent, and each logical
processor core has its own APIC. Because the APIC is CPU local, these APICs are
called local APICs or \gls{lapic}. \glspl{lapic} forward interrupts from
different sources to the respective CPU core. For example, the \gls{lapic}
receives interrupts such as \gls{ipi} from other \glspl{lapic} and forwards
legacy interrupts from the PIC via LINT0 pin. CPU external devices can deliver
their interrupts to the IOAPIC, which forwards them to the respective
gls{lapic}. Devices that implement PCI version 2.2 and later do not use the
IOAPIC. Instead, these devices use Message Signaled Interrupts, by directly
writing to a memory mapped register of the \gls{lapic}.
Figure~\ref{fig:state:technical:lapic} shows a schematic view on how the
\gls{lapic} of each CPU core integrates into the system. All APIC registers are
mapped to the 4-KiB APIC register space starting at the address specified
in APIC base address register. System software can then access APIC registers
with memory reads and writes to the APIC register space.

\subsection{Caches}
\label{sec:state:technical:caches}
Since 1980, the performance growth of memory and processors has diverged
steadily with an ever-growing gap. Hennessy et al. note a difference in
performance growth of factor 1,000 for a single CPU core and memory
technologies~\cite{hennessy2011computer}. To come by this disparity, CPU a
on-chip buffer called cache memory.\\

% With processors gaining capabilities to
% process more and more data in parallel, for example by increasing the core count
% or introducing new instructions, the demand for fast memory further grows. For
% example, from November 2023 to January 2024, the number of systems in the TOP500
% list that employed CPUs with 96 cores per socket increased from 0 to 3, with the
% former maximum number of cores per socket being 72~\cite{top500}. To hide
% latencies and bridge the gap between CPU demand and actual main memory
% bandwidth, CPUs today employ fast local on-chip memory to buffer data they
% already accessed. If they reaccess this memory item, they can use these buffers
% to speed up access. This on-chip buffer described is called cache.\\

The cache is an integral component organized in a multi-level hierarchy in
modern CPUs. In this hierarchy, the lowest and the nearest to the core level,
called the L1 cache, implements the fastest access. Most modern x86 CPUs divide
their L1 cache into two parts: L1D for data and L1I instructions. With
increasing levels, caches grow in size but tend to be slower. For example, while
the L1 Cache of Raptor Lake CPUs offers only five cycles of latency, the L3 Cache of
the same CPU has a latency of 52 cycles.\\

When the CPU tries to access data, it first queries the fastest cache. If the
CPU can locate the data in the cache, this is called a cache hit. On a cache
hit, the CPU can profit from reduced access time and improved bandwidth. The
opposite of a cache hit is called a cache miss, in which the CPU subsequently
queries the next level in the memory hierarchy. If it finds the data needed, it
loads these into the nearest cache for faster access. Caches can only store a
limited number of items, organized in cache lines. Each cache line is either 32
or 64 bytes in size, with x86\_64 processors mainly implementing a cache line
size of 64 bytes. Cache lines are a copy of the main memory, and the processor
uses this copy for all of its operations unless otherwise configured. The
processor loads data in cache line size granularity from the main memory. All of
this happens transparent to software and the processor automatically manages its
cache. \\

Caches use the principle of locality, which consists of two kinds of locality.
The first is called spatial locality. Spatial locality describes the observation
that if a program accesses data from the main memory, data located at nearby
addresses are the target of future accesses with a high probability. In this
case, without any prefetching, the CPU tries to guess the size of the loaded
structure to load parts missing in the cache in advance. If the CPU then
accesses the neighbor of the first data, it already resides in the cache,
lowering the latency. The second principle is temporal locality, which states
that a program soon reuses memory references with a high probability. The CPU
can, therefore, gain performance by storing recently used data in the cache for
reuse.\\

With cache lines being copies of main memory items, the need for synchronization
arises. For this, different strategies for writing data back to main memory
exist:
\begin{itemize}
  \item \textbf{write-back}: Data modified in the cache is stored and written
    back to the main memory later. Cache coherency protocols are required to
    allow multiple devices to access the same memory range.
  \item \textbf{write-through}: The changes in the cache are instantly written
    to the main memory. These writes can slow down program execution because of
    costly main memory writes, but ensure consistency with system memory and
    cache content. This type is relevant for devices that access system memory
    but do not access CPU cache.
  \item \textbf{cache-disable}: The CPU cache is disabled, and the CPU
    performs all memory operations using the main memory.
\end{itemize}
Legacy x86 software controls cache settings by setting or clearing configuration
bits in the control register CR0. In a system employing modern x86\_64
processors, systems software can set the caching strategy on page granularity
(c.f. Chapter~\ref{sec:state:technical:paging}) by setting and clearing the
respective bits in the page table entries. These settings are used in
conjunction with values of the Memory Type Range Registers (MTTR) and Page
Attribute Table (PAT) to calculate the effective caching strategy for a physical
address. X86\_64 processors ignore the write-through setting of CR0 and use the
page-level settings instead~\cite{amd_manual}. The default in x86\_64 processors
is to use a write-back strategy.

\subsection{Cache coherence protocols}
\label{sec:state:technical:caches_protocol}
An interesting difference in the x86\_64 implementation of Intel and AMD
processors are the cache coherency and inclusion policies. AMD processors use
MOESI as a cache coherency protocol~\cite{amd_manual}. Intel processors, on the
other hand, use MESIF as a coherency protocol~\cite{thomadakis2011architecture}.
MOESI and MESIF are extensions of the MESI protocol, introducing the
\textit{Owned} and \textit{Forward} state respectively. The meaning of the
states in the MESI, MOESI, MESIF protocols are as follows:
\begin{itemize}
  \item \textbf{Modified}: The copy in the processor's cache is the most
    recent and modified. The copy in the main memory needs to be updated.
    No other processor in the system maintains a copy.
  \item \textbf{Shared}:  The copy is the most recent and correct copy of the
    data. Other processors may hold copies, too. Main memory holds the
    most recent and correct copy, too.
  \item \textbf{Exclusive}: The processor's and the main memory's copy are the
    most recent and correct copies. No other processor holds a copy.
  \item \textbf{Invalid}: The cache line is invalid and needs to be replaced by
    a valid copy from main memory or other CPU's caches before it can be used.
  \item \textbf{Owned}: Only one processor can hold a cache line in this state.
    Other processors might hold a correct copy of the cache line in shared
    state. The owning core responds to requests. The copy residing in main
    memory might be stale. The cache line needs to be written back before it can
    be discarded, as it is dirty.
  \item \textbf{Forward}: Comparable to the \textbf{owned} state. It differs in
    that the content of the cache line is clean. Therefore, the processor does
    not require to write the line back to main memory before discarding it.
\end{itemize}

\begin{figure}
  \begin{center}
    \includestandalone{images/mesi.tex}
    \caption{MESI state transition of a cache line resulting from a (remote) read/write}
    \label{fig:state:technical:mesi}
  \end{center}
\end{figure}

All cache lines are tracked to be in one of the states.
Figure~\ref{fig:state:technical:mesi}\todo{quelle} shows transitions of MESI
states of a cache line as a result from a core's own or another core's remote
access. Important for my use case are the following state transitions:

\paragraph{Exclusive to Shared}
Processor $0$ has the exclusive copy of a cache line. Processor $1$ acquires a
copy to read. In this case, the state of the line in processor $0$'s
cache changes from exclusive to shared to reflect that other cores use the line.

\paragraph{Shared to Modified/Exclusive/Invalid}
\todo{Übergang shared -> exclusive in grafik fehlt}
Processor $0$ and processor $1$ both have the same data item in their local
caches. The state of both copies of the cache lines is \textit{shared}. Before
processor $0$ writes to the cache line it triggers the invalidation of the line
in processor $1$'s cache. Upon receiving the acknowledgement of invalidation
from processor $1$'s cache line, processor $0$ changes the state of its cache
line from shared to \textit{exclusive}. At this point the cache line is clean
and matches the copy in the main memory. Transferring a cache line to
\textit{exclusive} can be done in preparation to an anticipated write. After
writing to the cache line, its state is changed to \textit{modified}. If
processor $0$ performs an atomic read-write operation it will change its copy
directly from \textit{shared} to \textit{modified}. The state of the cache line
in processor $1$'s cache is now \textit{invalid}.

Processors write back their data to the main memory once a cache line is evicted
or when other processors request the same data item. If another processor
requests the data item, then the owning processor will make the item available
in a shared memory structure, such as the \gls{llc}.

\subsection{Cache Inclusivity}
\label{sec:state:technical:caches_inclusivity}
The use of different coherency protocols to synchronize data between multiple
CPU cores has a direct effect on the inclusivity of the cache implementation. On
all current x86\_64 multicore CPUs, the \gls{llc} is shared among all cores for
synchronization and uses one of the coherency protocols. Multicore CPUs produced
by Intel use an inclusive \gls{llc}~\cite{intel_optimization}. An inclusive
cache describes a cache containing items in lower cache levels. If an item is
modified in a lower cache level, the changes are automatically propagated to the
higher inclusive cache level. The opposite of an inclusive cache is an exclusive
cache design, as used by most AMD CPUs. Exclusive caches do not necessarily
contain items of lower cache levels, and the synchronization of modified items
needs to be propagated in other ways. The additional "owned" state of the MOESI
protocol solves this issue.\todo{this might be more understandable with a
graphic}

\subsection{Hardware Performance Monitoring Counters}
\label{sec:state:technical:hpc}
The first x86 CPU implementing hardware \glspl{pmc} and documenting them was the
Intel Pentium Pro in 1995, implementing the P6
microarchitecture~\cite{intel_sdm}. The x86\_64 \gls{isa} specificities four
freely programmable architectural hardware \glspl{pmc}~\cite{amd_manual}.
Concrete processor implementations can offer additional counters. Similarly, the
\gls{isa} specifies architectural events that must be present in every processor
implementing x86\_64. Additional events are vendor and implementation-specific.
The four counters can be programmed to count any event supported by the
respective processor implementation. Vendors of x86 CPUs publish what processor
supports what additional events in their manuals. Moreover, a CPU reports what
events it supports in the result of the \gls{g_cpuid} instruction. \\

In x86 hardware, performance counters are implemented by a set of two
\glspl{msr} per counter. One \gls{msr} can be programmed by system software with
the event to measure, while a second \gls{msr} counts the occurrence of the
respective event. As noted, programming has to be done by system software with
elevated privileges. Reading \glspl{pmc} can be done with the privileged RDMSR
instruction or from user space with the RDPMC instruction. In this way, a
program can poll the values of counters. The system software can also program a
threshold for a \gls{pmi}. Once the \gls{pmc} values exceed the threshold, the
\gls{pmi} is triggered, offering an alternative to expensive polling
techniques.\\

When using hardware \gls{pmc}, one must use the proper technique adopted to the
environment in which to obtain the counter values. Das et al. found in a
comprehensive survey that noise from the system is often present, e.g. context
switches influence the values of performance counters~\cite{das_sok_2019}.
Moreover, some counter events are over-counted while the CPU can undercount
others~\cite{weaver_non-determinism_2013}. It is therefore important to check the
right conditions for using hardware counters and verify that they work
correctly.

\subsection{Paging}
\label{sec:state:technical:paging}
With the evolution of computer systems not strictly processing data in a batched
manner anymore but instead allowing for multiple applications to run
concurrently, a need for complex memory management arose. For example, system
software must ensure that two applications programmed to use the same memory
addresses do not influence each other. Segmentation is a legacy technique used
in x86 systems to solve this problem. Segmentation allows the construction of
address spaces that allow transparent translation of application addresses
within an address space dedicated to the application. However, if applications
do not entirely fit into memory, system software must perform expensive swapping
operations between main and disk memory. In this swapping operation, programmers
either need to split up their application into parts that must reside in the
main memory simultaneously, or system software must swap the complete
application.\\

\begin{figure}
  \begin{center}
    \includestandalone{images/paging.tex}
    \caption{Illustration of Virtual Address Translation with 4 Levels of Page Tables (after \cite{amd_manual}, p. 142)}
    \label{fig:state:technical:paging}
  \end{center}
\end{figure}

A solution to this is virtual memory implemented through
paging~\cite{tanenbaum2015modern}. Paging splits the physical address space into
small, evenly sized-parts called pages. The main memory is split into parts of
the same size, called page frames, by which each page can be backed. By
splitting the address space into pages, system software can now perform swapping
more fine-grained. If an application accesses a page not backed by a frame,
hardware emits a fault, and system software can take the necessary steps to load
it into physical memory. On the other hand, applications are now implicitly
split into parts, allowing system software to only swap out parts of running
applications. Additionally, paging allows applications to theoretically use the
whole address space, with system software to decide what virtual address is
currently backed by what physical page frame through its mapping function. This
allows system software even to map a single physical page frame to the address
spaces of two applications simultaneously, for example, if both applications use
the same shared library. Knowing in detail how paging work on x86\_64 hardware
will help understand content of later chapters.\\

Hardware uses page tables that are hierarchically organized for the translation
of virtual to physical addresses. Each page table forms a level in the table
hierarchy and is the size of one page, storing references (addresses) to the
next lower level. The last page table in this hierarchy stores the address of
the physical page frame. By default, x86\_64 uses pages of size 4 KiB. Because
x86\_64 uses 64-bit addresses, each 4 KiB sized page table can store 512
entries. Figure~\ref{fig:state:technical:paging} shows the page translation of a
virtual 64-bit address in x86\_64 with a page table hierarchy of four levels and
a page size of 4 KiB. A virtual 64-bit address consists thereby of indices in
the respective page tables to find the address as an entry in the lowest-level
page table. Each page table in the hierarchy is referenced by its physical
address.\\

A virtual address for x86\_64 4-level paging with 4 KiB page size is divided
into a 12-bit field used as an index in the page frame and four 9-bit fields,
which the hardware uses as an index to access the content of the respective page
level. The remaining address bits are sign extensions of bit 52, forming a
canonical address. The hardware must perform a page table walk to resolve the
virtual address. For the first step of a page table walk, the processor locates
the address of the first page table, in this case, the PML4 table. The address
of the PML4 is written by system software to CR3 after it created the page
table. All page tables are aligned to 4 KiB, which leads to a value of 0 for the
12 least significant bits. Depending on the configuration, these 12 bits in CR3
are either used to determine the memory type of the PML4 or to store
process-context identifiers. To access PML4, the CPU reads its physical address
from CR3. It uses the 9-bit index of the PML4 offset field in the virtual
address to find the Page-Map-Level 4 entry containing the address of the next
page table, the Page-Directory-Pointer table. The CPU accesses this table and
all successor tables similarly until it locates the physical page. To access the
data that belongs to the virtual address, the CPU uses the last 12 bits of the
virtual address as an index to access the respective byte in the page frame. \\

As the page table walk is expensive, the \gls{mmu} automatically stores the
translated addresses in a cache called \gls{tlb}. System software has to manage
page tables and \glspl{tlb} for each CPU independently. Furthermore, system
software must maintain consistency between page tables and each \gls{tlb} by
invalidating individual entries or the entire \gls{tlb}. If system software
invalidates an entry, the CPU has to complete a page walk for this virtual
address, upon which the \gls{mmu} updates the \gls{tlb}. Similar to other
caches, filling entries in the \gls{tlb} works transparent to software. System
software can use the INVLPG instruction to explicitly invalidate the \gls{tlb}.

\begin{figure}
  \begin{center}
    \includestandalone{images/pte_rights.tex}
    \caption{Layout of the 12 least significant bits in a page table entry}
    \label{fig:state:technical:paging_rights}
  \end{center}
\end{figure}
\todo{referent auf AMD manual}
The unused lower 12 bits of each table entry are reused for management
properties, such as access rights management.
Figure~\ref{fig:state:technical:paging_rights} shows the layout of the lower 12
bits of a page table entry. The present bit is another property stored in the
lowest 12 if of a page table's entry. It indicates if the page the respective
entry points to was initialized and loaded to the page table. If this bit is
clear, the CPU generates a \gls{pf} exception and expects system software to
load the respective entry. The CPU also generates a \gls{pf} if software
violates access rights. A second bit important for this thesis is the Page-Level
Cache Disable (PCD) bit. If the PCD bit is set, the CPU cannot cache the
respective page. Bit 3 (PWT) controls page-level write-through. If system
software sets this bit, the page table has a write-through caching policy.
