\section{Summery}
Next to the introduction to the technical aspects of x86 processors we have seen
examples for \glspl{tee}, attacks on \glspl{tee} and possible mitigations to
those attacks. \\

All \gls{tee} solutions have in common, that they do not implement the whole
functionality exclusively in hardware. For example, Intel and AMD employ a
dedicated security processors in their x86 \gls{soc} to implement \gls{sgx} and
\gls{sev} features respectively. Together with Enma, which root of trust is the
\gls{tpm}, all of these solutions a signature based remote attestation scheme.
Next to this they all share the property of being vulnerable to side channel
attacks, with Spectre and Meltdown being rather prominent ones. These attacks
abuse observable (micro-) architectural side effects to leak secrets through
covert channels. For some attacks mitigation exist, but they either have a
impact on performance or can't be applied to all side channel attacks.
Additionally, systems can observe these side effects through \glspl{pmc} to some
degree, but as we have seen in section~\ref{sec:20:def_sca}, attacks can avoid
being detected through reducing their throughput or simply if the noise in the
system is to high. With QuanShield interrupt based side channels were mitigated
by Cui et al. by creating an interrupt free environment, which at least
mitigates this attack vector for \gls{sgx}.\\

Putting these together, I identified a approach not tested to the best of my
knowledge. This is to exclude architectural side channels by creating a
completely isolated execution environment, that only uses core local resources
and monitors its integrity by utilizing \glspl{pmc}.
