\section{Side Channel Attacks}
\label{sec:20:attacks}
In this this section we review side channel attacks that TEECore mitigates
against. While many side channels exists in a computers system, I want to focus
on two classes of architectural side channels. First, in
section~\ref{sec:20:transientattacks} we will lear about transient execution
side channels. The second class, interrupt based side channels, will be
explained in section~\ref{sec:20:interrupt_sca}. Both classes have in common
that they rely on observable architectural changes in the processor to
exfiltrate information.

\subsection{Transient Execution Attacks}
\label{sec:20:transientattacks}
In 2018, researchers published the Spectre and Meltdown
attacks~\cite{kocher_spectre_2020, lipp_meltdown_2020}. These attacks were the
first to exploit the side effects of transient execution in modern CPUs and
affected all commodity processor architectures. As an example, CPU designs of
the AMD, Intel, Qualcomm, and ARM were identified to be vulnerable by transient
execution attacks~\cite{wikner2022retbleed,moghimi_downfall_2023,ragab_ghostrace_2024}.\\

Modern processor often execute instructions without the knowledge that their
execution preconditions are met to gain performance. As a result, processors
need to revert all results from executing an instruction from which they later
learn that its precondition was not satisfied. An instruction executed in this
way is called a transient instruction. Because of performance reasons, a
processor might not revert all microarchitectural side effects of transient
execution. Spectre and Meltdown abuse different race conditions in transient
execution to exfiltrate data through observing these microarchitectural changes.
In particular, the original Spectre and Meltdown attacks use cache based covert
channels. To use these channels, both attacks measure the access time through
special crafted memory addresses from which they can conclude the presence of
data. They therefore rely on the presence of a medium shared by the attacking
and the victim thread to communicate information beyond process boundaries. \\

% \subsubsection{Spectre Attack}
% \subsubsection{Meltdown Attack}
% Both attacks abuse a race condition between transient executed instructions and
% memory access (Spectre) or exception delivery (Meltdown). In strictly sequential
% instruction processing, each instruction would only be executed if the results
% of the one before arrived. With speculative execution, the CPU executes
% instructions before the results of previous instructions arrive. With branching
% or exception delivery, this can lead to a race condition in which instructions
% are executed before the branching decision is committed or an exception arrives.
% Both attacks aim to make the window in which instructions are executed
% speculatively as large as possible to observe the microarchitectural side
% effects later. With carefully crafted code, these race conditions can be used to
% access arbitrary memory (Meltdown) or memory readable by the attacked process
% (Spectre). \\

% Both attacks have in common the fact that they transmit data through a covert
% channel. In the original attack descriptions, both attacks use a cache covert
% channel to which they transfer data with the help of transient executed
% instructions.

The class of transient execution attacks is still highly relevant to this day.
Since 2023, security researchers discovered at least five attacks that use
microarchitectural effects to exfiltrate data, similar to Spectre and Meltdown~
\cite{ormandy2023zenbleed,trujillo2023inception, moghimi2023downfall,ragab_ghostrace_2024, wilke2024tdxdown}.
Current hardware \gls{tee} solutions are affected, too, because these attacks
enable attackers to read arbitrary memory. The problem persists, and no solution
exists to mitigate transient execution attacks in general.\\

\subsection{Interrupt Based Attacks}
\label{sec:20:interrupt_sca}
In section~\ref{sec:20:transientattacks}, we have seen that data can be leaked
from various \gls{tee} implementations over side channels if an attacker can
interrupt a \gls{tee} in a high temporal resolution. With SGX-step Van et al.
presented the first framework with which an attacker can mount attacks on
\gls{sgx} with instruction-level granularity~\cite{van2017sgx}. Next to the
aforementioned covered side channels, this allows the attacker to learn access
patterns of the \gls{tee} and tightly monitor its control flow. Wilke et al. and
Kou et al. implemented similar attacks for AMD and ARM
processors~\cite{wilke2023sev, kou2021load}.\\

On x86 processors, the attack scheme uses the \gls{lapic}, while on ARM, it uses
a separate core to send inter-processor interrupts to overcome the boundaries
set by the two worlds in TrustZone. While Kou et al. use these interrupts to
mount high-precision flush+evict attacks, both x86 attacks additionally describe
how to monitor \gls{tee} behavior by manipulating the page tables of the
\gls{tee}. By setting the accessed bit to 0, the attacker can at least learn of
memory accesses on page granularity. All three attacks are, therefore, able to
break the security guarantees of the \gls{tee} while not granting an attacker
more power than the attacker model of the respective solution does.
