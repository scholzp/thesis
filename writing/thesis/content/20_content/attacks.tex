\section{Attacks on Trusted Execution Environments}

% \subsubsection{Iago Attacks}
% \label{sec:20:iago}
% Checkoway et al. first published the class of Iago attacks in
% 2013.\cite{checkoway2013iago}. The attacker model in this attack class can
% neither manipulate the application's code nor read the data from its memory. The
% application is assumed to be unmodified and linked against unmodified libraries
% but running on a malicious kernel. These application properties equal those of
% SGX. The attack aims to manipulate the application through malicious system call
% returns answered by the kernel. Such unexpected return values can cause the
% application to work against its security interest or even manipulate the control
% flow. SGX applications are potential targets for Iago attacks because they
% depend on the runtime outside the enclave. Intel TDX and AMD SEV-SNP rely on an
% untrusted hypervisor.\cite{tdx_whitepaper,kaplan_amd_2020} This untrusted
% hypervisor could mount Iago attacks similar to malicious kernels. The WeSee
% attack on AMD SEV-SNP, published by Schlüter et al. in 2024, shows that such an
% attack is possible on AMD SEV-SNP by injecting specific interrupts into the
% confidential VM.\cite{schluter2024wesee}\\

\subsection{Interrupt Based Side Channel Attacks}
\label{sec:20:interrupt_sca}
An attacker can learn about memory access patterns and behavior by using
interrupt-based side-channel attacks. The adversary in this class of attacks can
disrupt the execution of an enclave or confidential VM by sending interrupts.
These interrupts cause a context switch from the enclave or VM to the interrupt
handler of the OS or hypervisor. The malicious system software can then inspect
the state of the hardware, such as the L1 cache, the TLB, or the accessed or
dirty bits of pages. System software thus can single-step the enclave or VM with
a high enough frequency of malicious interrupts at instruction level
granularity. Attacks that utilize interrupts to learn about the state of trusted
execution environments exist for Intel SGX, ARM TrustZone, and AMD
SEV.\cite{van2017sgx, kou2021load, wilke2023sev}\\

\subsection{Transient Execution Attacks}
\label{sec:20:transientattacks}
In 2018, researchers published the Spectre and Meltdown
attacks.\cite{kocher_spectre_2020, lipp_meltdown_2020} These attacks were the
first to exploit the side effects of transient execution in modern CPUs and
affected all commodity architectures. For example, CPUs of the vendors AMD,
Intel, Qualcomm, and other ARM designs were affected. This class of new
transient execution side-channel attacks abuses the speculative execution
feature of modern CPU and defines an entirely new class of attacks. Furthermore,
they are the first class of attacks that abuse microarchitectural bugs. We
review this class of attacks in more detail as I aim to implement a TEE that
can defend against such an attack.\\

Modern CPUs use speculative execution to hide memory latencies. If, for example,
the control flow forks, the decision of which path to take often depends on a
value stored in memory. If this value is not persistent in the CPU cache, the
CPU needs to fetch the value from the main memory. While waiting for the result,
the CPU precalculates the most likely path. The CPU uses a dedicated buffer
called Branch Target Buffer (BTB) to decide what path to precalculate. The
Buffer records the n-th last paths taken, from which the CPU decides the most
likely path. If the value arrives from the main memory, the CPU checks its
decision and corrects its mistake, if any, to take the correct path. Overall,
speculative execution can lead to a huge performance increase. Older
publications see a performance increase from 27\% up to
87\%.\cite{espasa1997out, mock2005empirical}\\

In some cases, the decision about what path to precalculate is wrong. For
performance reasons, the CPU does not revert its microarchitectural state, like
cache content, on mistakes. Furthermore, if the wrongly speculatively executed
code produces an exception, the CPU does not serve it because the CPU should
never have executed the code path. Spectre and Meltdown-like attacks exploit
this design decision to read arbitrary memory. The attacker trains the BTB to
predict a path dependent on a memory address to prepare an attack. In the
training phase, the attacker uses valid addresses when the control flow
branches, which leads the CPU in the actual attack phase to predict that the
path in the training will be valid. In the attack phase, the attacker chooses an
arbitrary address. In preparation, the attacker evicts the value on which the
chosen path depends from the cache. Now, the attacker chooses a value that would
result in the decision to take a different code path. Because the attacker
trained the CPU before to take the now invalid path, it executes this one
speculatively until the requested value arrives. In the meantime, the CPU
executes a malicious read to the main memory using the address chosen by the
attacker. Later, the result of the malicious read returns and is stored in the
CPU cache. In the meantime, the CPU notices its mistake and takes the other
path. The value load resulting from the manipulated address still leaves traces
on the microarchitectural state of the CPU (in this example, the cache), and the
CPU does not throw an exception because the CPU should have never executed the
in this path. The attacker can now extract the secret through a side channel of
their choice. Because these attacks target microarchitectural behavior, a
complete redesign is necessary to fix the issue. Software mitigations are
available for specific attacks. For example, the Linux Kernel uses techniques
called retpopline and Kernel Page Table Isolation (KPTI) to mitigate Spectre
version 2 and Meltdown, respectively. On the other hand, software mitigations
can greatly impact performance, reaching from a 10\% to 800\% overhead,
depending on the workload.\cite{low2018overview} The class of transient
execution attacks is still highly relevant today, with at least five attacks
published in the last since 2023.
\cite{ormandy2023zenbleed,trujillo2023inception, moghimi2023downfall,ragab_ghostrace_2024, wilke2024tdxdown}
TEE solutions are affected, too, because these attacks enable attackers to read
arbitrary memory. The problem persists, and no solution exists to mitigate
transient execution attacks in general.\\

