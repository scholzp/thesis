% -*- Mode: Latex -*-

%  Zusammenfassung

% Zu einer runden Arbeit gehört auch eine Zusammenfassung, die
% eigenständig einen kurzen Abriß der Arbeit gibt. Eine halbe bis ganze
% DINA4 Seite ist angemessen. Dafür läßt sich keine Gebrauchsanweisung
% geben (für irgendetwas müssen die Betreuer ja auch noch da
% sein).
% - TEE important to protect code and data
% - Different technoligies allow for creation of TEE
% - Most are vendor specific extensions to processors, such as SGX and SEV-SNP
% - Other are optional, such as TrustZone
% - vendor locking as a result
% - Sie channel attacks pose a thread
% - Spectre/Meltdown effectively circumvent all security guarantees
% - Current work focuses on mitigating side channels used by transient execution attacks
% - no general solution
% - I propose an approach that uses a feature found in all commodity CPUs to detect the presence of side channels
% - for this i create a highly isolated execution environment
% - i show a design that allows for remote attestation using a tpm
% - i show what shortcomming my soltuion have and evaluate what constraints my isolation mechanism imposes to workloads
% - in the end, i evaluate my solutions regardning security to achive an argue why its not a feasable standalone solution with current available hardware

Trusted execution environments allow the execution of code in an environment
isolated from the remaining systems and effectively aim to protect data against
unauthorized access and program code from being tampered with without detection.
To form a trusted relationship with the user, trusted execution environments
offer remote attestation, by which they can prove claims about the code running
inside of the environment and its state to the user.\\

Most hardware vendors offer special hardware extensions to allow the creation of
trusted execution environments. For commodity CPUs, these extensions are called
SGX in Intel processors, SEV-SNP in AMD processors, and TrustZone in Arm
devices. Vendors of these CPU extensions make a strong claim about their
security properties. With the advent of microarchitectural vulnerabilities in
2017, these claims have been proven by research to be broken or only held under
special circumstances. So-called transient execution attacks abuse the
microarchitectural behavior of modern CPUs. Software mitigations to certain
vulnerabilities exist but do not mitigate the source of the problems but rather
channels through which they extract information. Consequently, researchers
discover new vulnerabilities each year. \\

In this thesis, I want to develop a software-based approach to trusted execution
environments. Instead of using vendor-specific hardware extensions, my approach
is to use features present in nearly all commodity processors. By isolating a
single CPU core to run my environment, I create a highly isolated environment
that mitigates some of the attacks by design. For remaining attacks, I use
performance monitoring counters to detect data breaches. Together with this
isolation solution, I sketch a protocol that allows for remote attestation using
the Trusted Platform Module as a trust anchor and, therefore, allowing the
creation of a trusted execution environment without any vendor-specific hardware
extensions.\\

I show that this solution can reliably detect data breaches and isolate against
other forms of side-channel attacks. I also evaluate my prototype with regard to
what constraints it imposes on workloads. Finally, I will show that with the
currently available hardware, my solution can not alone protect sufficiently
from attacks and needs to be paired with additional solutions. I evaluate what
of the already existing solutions can be paired with my solution to increase
overall security properties.

% Das ab hier ist nur ein Platzhalter. Das ist der Abstract vom Einleitungsvortrag
% Remote attestation is used in confidential computing to prove to a client over a
% computer network that a system in question fulfills a set of defined properties.
% In a computer system, software can use remote attestation to verify that, for
% example, the bootflow of a system was not compromised by malware. One
% requirement for remote attestation is a hardware device that forms the root of
% trust\cite{coker2011principles}. For the bootflow example, the hardware root of
% trust in x86 systems is given by a Trusted Platform Module which, short TPM.

% A technology enabling the concept of remote attestation to a per-application
% basis is Intel's Software Guard Extensions (SGX). It allows applications to run
% code in a secure enclave, isolated from the operating system or hypervisor. SGX
% protects the integrity of the code to run in an enclave by providing remote
% attestation about the state of the enclave. An explication can decide on the
% provided attestation results if it wants to exchange secrets with the enclave.
% Confidentiality of secrets exchanged with the enclave through is protected by
% cryptography when using SGX. The root of trust is formed by the correctly
% implemented special SGX instructions. Enclave keys are embedded in the firmware,
% protected by the Intel management engine.\cite{costan2016intel}

% In recent years many flaws in the implementation of SGX were discovered by
% security researchers. With the advent of side-channel attacks, such as Spectre,
% a new attack vector was discovered. While not part of the threat model of SGX,
% special Spectre attacks can be used to leak enclave
% content.\cite{chen2019sgxpectre, van2018foreshadow} Even without side channels,
% architectural vulnerabilities can be used to leak SGX content and even
% keys.\cite{borrello2022aepic}

% All these vulnerabilities work because they expose data through the memory
% subsystem. To mitigate this attack vector, I want to elaborate on the
% possibility of constructing an enclave-like partition that runs code entirely in
% local cache structures. For this, I want to isolate a single CPU core so that it
% can not be tampered with from other system components. Software running in this
% enclave shall be able to maintain its confidentiality by observing performance
% counters, from which it can decide to react appropriately if it detects a data
% breach. Moreover, I want to review algorithms for remote attestation applicable
% to the isolated CPU core, so that its state can be attested to applications
% running outside of the enclave. Combining both results in an enclave immune to
% side channel attacks running on commodity hardware without the need for
% additional hardware support. With the additional support of a hardware root of
% trust like the TPM, this concept can achieve security guarantees similar to
% Intel SGX.

%%% Local Variables:
%%% TeX-master: "diplom"
%%% End:
