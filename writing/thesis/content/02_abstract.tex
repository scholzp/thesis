% -*- Mode: Latex -*-

%  Zusammenfassung

% Zu einer runden Arbeit gehört auch eine Zusammenfassung, die
% eigenständig einen kurzen Abriß der Arbeit gibt. Eine halbe bis ganze
% DINA4 Seite ist angemessen. Dafür läßt sich keine Gebrauchsanweisung
% geben (für irgendetwas müssen die Betreuer ja auch noch da
% sein).
% - TEE important to protect code and data
% - Different technoligies allow for creation of TEE
% - Most are vendor specific extensions to processors, such as SGX and SEV-SNP
% - Other are optional, such as TrustZone
% - vendor locking as a result
% - Sie channel attacks pose a thread
% - Spectre/Meltdown effectively circumvent all security guarantees
% - Current work focuses on mitigating side channels used by transient execution attacks
% - no general solution
% - I propose an approach that uses a feature found in all commodity CPUs to detect the presence of side channels
% - for this i create a highly isolated execution environment
% - i show a design that allows for remote attestation using a tpm
% - i show what shortcomming my soltuion have and evaluate what constraints my isolation mechanism imposes to workloads
% - in the end, i evaluate my solutions regardning security to achive an argue why its not a feasable standalone solution with current available hardware

Trusted execution environments allow the execution of code in an environment
isolated from the remaining system and effectively aim to protect data against
unauthorized access and program code from being tampered with without detection.
To form a trusted relationship with the user, trusted execution environments
offer remote attestation, by which they can prove claims about the code running
inside of the environment and its state to the user.\\

Most hardware vendors offer special hardware extensions to allow the creation of
trusted execution environments. For commodity CPUs, these extensions are called
SGX in Intel processors, SEV-SNP in AMD processors, and TrustZone in Arm
devices. Vendors of these CPU extensions make a strong claim about their
security properties. With the advent of microarchitectural vulnerabilities in
2017, these claims have been proven by research to be broken or only held under
special circumstances. So-called transient execution attacks abuse the
microarchitectural behavior of modern CPUs. Software mitigations to certain
vulnerabilities exist but do not mitigate the source of the problems but rather
channels through which they extract information. Consequently, researchers
discover new vulnerabilities each year. \\

In this thesis, I develop a software-based approach to trusted execution
environments. Instead of using vendor-specific hardware extensions, my approach
is to use features present in nearly all commodity processors. By isolating a
single CPU core to run my environment, I create a highly isolated environment
that mitigates some of the attacks by design. For remaining attacks, I use
performance monitoring counters to detect data breaches. Together with this
isolation solution, I sketch a protocol that allows for remote attestation using
the Trusted Platform Module as a trust anchor and, therefore, allowing the
creation of a trusted execution environment without any vendor-specific hardware
extensions.\\

I show that this solution can reliably detect data breaches and isolate against
other forms of side-channel attacks. I also evaluate my prototype with regard to
what constraints it imposes on workloads. Finally, I will show that with the
currently available hardware, my solution can not alone protect sufficiently
from attacks and needs to be paired with additional solutions. I evaluate what
of the already existing solutions can be paired with my solution to increase
overall security properties.

%%% Local Variables:
%%% TeX-master: "diplom"
%%% End:
