\chapter{Design}
\label{sec:design}

% Ist das zentrale Kapitel der Arbeit. Hier werden das Ziel sowie die eigenen
% Ideen, Wertungen, Entwurfsentscheidungen vorgebracht. Es kann sich lohnen,
% verschiedene Möglichkeiten durchzuspielen und dann explizit zu begründen,
% warum man sich für eine bestimmte entschieden hat. Dieses Kapitel sollte -
% zumindest in Stichworten - schon bei den ersten Festlegungen eines Entwurfs
% skizziert werden. Es wird sich aber in einer normal verlaufenden Arbeit
% dauernd etwas daran ändern. Das Kapitel darf nicht zu detailliert werden,
% sonst langweilt sich der Leser. Es ist sehr wichtig, das richtige
% Abstraktionsniveau zu finden. Beim Verfassen sollte man auf die
% Wiederverwendbarkeit des Textes achten.

% Plant man eine Veröffentlichung aus der Arbeit zu machen, können von diesem
% Kapitel Teile genommen werden. Das Kapitel wird in der Regel wohl mindestens 8
% Seiten haben, mehr als 20 können ein Hinweis darauf sein, daß das
% Abstraktionsniveau verfehlt wurde.

In this chapter, I will explain the design of the TEECore and how a system that
utilizes the TEECore prototype looks. The design tries to mitigate the risk of
covered side channels by fully excluding them on an architectural base.\\

In the first section (c.f.~\ref{sec:30:tee_general}), I will describe the
general system and usage of the TEECore prototype implementation. The second
section(section~\ref{sec:30:tee_kernel}) will explain TEECore in more detail and
how it tries to achieve its goals. TEECore alone does not implement any
workloads that third-party software can use. Such workloads and under what
constraints they must be implemented will be explained by
section~\ref{sec:30:tee_apps} of this chapter. I will then explain the remote
attestation scheme of TEECore in section~\ref{sec:30:tee_ra_scheme} before I
sketch the envisioned boot flow in section~\ref{sec:30:tee_boot_chain}. With all
components explained, I want to conclude this chapter by discussing the attacker
model that TEECore can defend and not defend against in
section~\ref{sec:30:tee_attacker_model}.

\section{General System Design}
\label{sec:30:tee_general}
In this section we will review how TEECore integrates in a system with other
components. Before we take a more detailed look on some of the components in
such a system, this section describes how system parts interact with each other.

\label{sec:30:system_overview}
\begin{figure}
    \begin{center}
        \includestandalone{images/30_system.tex}
        \caption{TEECore running in a System next to a commodity \gls{os}}
        \label{fig:30:tee_system_design}
    \end{center}
\end{figure}

Figure~\ref{fig:30:tee_system_design} shows an overview of a system integrating
TEECore next to a commodity \gls{os}. Speaking on a higher level, the bootloader
is responsible for partitioning the system into two parts: The first is managed
by TEECore, and the second is running the commodity \gls{os}. Because TEECore is
designed to run the local CPU cache, the first partition hosting consists of a
single CPU core. Thus, the remaining system resources, such as I/O ports,
additional CPU cores, and memory, are part of the second partition running the
commodity \gls{os}.\\

The design goal of TEECore is to enable the commodity \gls{os} to run workloads
in a shielded location so that no information can be leaked to other
applications. In figure~\ref{fig:30:tee_system_design}, such a workload is
defined as a trusted application.

TEECore uses a \gls{tpm} as a hardware trust anchor to store data and keys. It
utilizes measured boot to determine if the system is trustworthy when setting up
its services. Extend operations to the content of the \gls{tpm}'s \glspl{pcr}
form the backbone of TEECore's attestation feature and allow it to fulfill all
requirements of a \gls{tee}.\\

The second partition runs the commodity \gls{os} and all its applications. The
commodity \gls{os} implements system functions that allow applications running
on top of it to access services hosted in the partition running TEECore. The
commodity \gls{os} needs a special driver that implements the routines necessary
to communicate with TEECore and its services. Both partitions communicate via a
shared memory region. In the setup phase, this shared memory region is reserved
and placed into the memory map of both partitions for their systems to include
and map it.\\

The idea of partitioning the system is much more similar to the idea behind ARM
TrustZone\todo{ref} then to \gls{sgx} or \gls{sev}. This is because TEECore
tries to use the given hardware's functionality to mimic an isolated trusted CPU
core and works much more like the idea of a trusted world in ARM TrustZone
instead of \gls{sgx} and \gls{sev}. While AMD indeed integrates an ARM Core in
their \glspl{soc} and uses TrustZone to implement \gls{sev}, Intel uses a of not
specified architecture dedicated CPU core too to implement its Management
Engine and \gls{sgx} on top of it. Therefore, TEECore's goal is to provide a
solution for x86 that allows the implementation of a \gls{tee} without using
proprietary vendor features.

\section{TEECore}
\label{sec:30:tee_kernel}
TEECore employs isolation mechanisms that protect data from being leaked while
protecting itself from modification by both the trusted application and the
commodity \gls{os} running on the rest of the system. For protecting TEECore
against malicious workloads, I Use hardware-assisted domain separation through
different privilege levels, as widely used in other \gls{os}. TEECore uses
\glspl{pmc} to observe the platform state and build protection measurements upon
those observations. The key component for this is hardware events that are
triggered when the CPU core under TEECore's control is forced to share data with
the remaining systems. Using respective interrupts, TEECore can react to these
leaks appropriately. I decided that the consequences of such a leak should lead
to a platform reset, effectively undermining an attack attempt.\\

TEECore kernel utilizes CPU core-exclusive \gls{pmc} registers to monitor data
flow between cores for isolation. The assumption on this part of the kernel is
that shared parts of the CPU can be used to create a hidden channel for
transferring data. If the CPU uses such resources, for example, a shared L3
cache, then those effects trigger respective performance monitoring events. With
respectively programmed \glspl{pmc}, those events can be counted by the TEECore.
\glspl{pmc} can be programmed to emit a \gls{pmi}, which the \gls{tee} uses to
register forbidden events. Because the source of those events must be known to
conclude the presence of an attacker safely, only selected events can be used by
TEECore and the \gls{tee} must ensure their regular operation does not trigger
the respective events. TEECore, therefore, must ensure the use of local CPU
resources only. On Intel platforms, I use events that measure interaction with
the so-called offcore. Offcore events result from the core interacting with
resources that are not local. These events include readings/writings to shared
caches, RAM, or other components. From the fact that TEECore allows only usage
of core local resources, a memory constraint results, which confines the size of
usable memory to that of CPU local caches.\\

Before the TEECore can transfer control to tasks, it must bring the environment
to a defined state. All memory is stored in the CPU's local cache in the
exclusive or modified state. To achieve this, TEECore writes to all memory it
has to protect, invalidating shared cache lines in other cores in this process.
To protect a task against interrupt-based attacks, TEECore has to set up the
\gls{idt} correctly so that any occurrence of any interrupt results in a system
reset. TEECore achieves this by setting the \gls{idtr} to zero, which results in
a general protection fault once the CPU tries to locate the \gls{idt}. Because
this provokes a second fault after the first, the CPU generates a double fault,
ultimately resulting in a triple fault and, thus, a system reset. Furthermore,
it programs the \gls{pmc} to detect attacks on itself or its payload as soon as
possible. TEECore uses the possibility to generate \glspl{pmi} on the overflow
of a \gls{pmc}. As explained, the resulting \gls{pmi} leads to a system reset.
Important at this point is the fact that TEECore can only react to attacks in
this way. It cannot prevent them entirely.\\

External memory, such as the shared memory region and memory mapped registers of
the \gls{tpm}, are mapped as un-cacheable. One reason is that it prevents
external memory from polluting the cache. Another more relevant cause for this
is that TEECore can single out what offcore event was caused by malicious
interactions through specific events. For example, read or write requests served
by DRAM are allowed for these types of memory, while the same operations being
served by other CPUs are not allowed.\\

The payload of TEECore is called a trusted application. In the proof of concept,
implementation tasks are implemented as functions and are thus statically
embedded into the TEECore binary. A trusted application is allowed to access the
shared memory communication channel. TEECore monitors this access and makes it
part of its measurements. After a trusted application run, TEECore
creates a Snapshot from the \gls{pmc} registers and a report based on this
information.

\section{Trusted Applications}
\label{sec:30:tee_apps}
Trusted applications run in user space in the secure partition and provide
services to software in the normal partition. Compared to applications running
on top of commodity \gls{os}, applications running in an environment managed by
TEECore must respect multiple restrictions.\\

For now, the design does not support the dynamic installation of user space
application in the \gls{tee} environment. This restriction is the consequence of
TEECore not being a fully fledged \gls{os} and the missing availability of
\gls{os}-specific libraries such as a standard C library. As a result, an
application must be implemented as a function in the TEECore using the
respective framework. Changing the application requires the recreation of the
whole TEECore binary. While this reduces the flexibility of running systems, it
allows for easier integrity measurement of applications because they are
statically embedded in the TEECore binary, which is already measured in the boot
flow through the measured boot. Moreover, embedding the application in the
TEECore binary as a function removes the necessity to implement an executable
loader, which reduces the amount of code and, therefore, the size of the binary.
Another restriction is that the whole run time environment has to fit together
with the application into the CPU's local cache.\\

The small size of available memory also restricts the size of data the workload
can generate. Working with large data sets is not possible, as exceeding the
capacity of the cache will lead to replacements in the local cache, which will
result in leakage to a shared part of the memory subsystem, an event considered
by TEECore as an attack. TEECore then reacts similarly, which results in a reset
of the whole system. Moreover, the organization of cache structures can
influence memory usage. I will evaluate such effects in chapter\todo{hier die
    Kapilnummer nennen}. The last aspect regarding memory is how it is managed.
Trusted applications are disallowed for the prototype of TEECore to allocate
heap memory freely. This is because cache lines must be set to either exclusive
or modified by TEECore to enable protection, which in turn means that all used
memory addresses must be known beforehand. Applications can circumvent the last
restriction by allocating large enough memory chunks through structures of size
known at compile time.\\

Another restriction is that the secure partition only uses local resources from
a single core. In the current model, trusted applications are forbidden to
access any I/O beyond the boundaries of the local resources of the core that the
secure partition forms. The only access allowed is to the shared memory path
managed by the TEECore kernel. While it would be possible to use the same
resources, for example, I/O port, as the commodity \gls{os} in the normal
partition, this would require the implementation of complex protocols for each
shared resource in TEECore and the commodity \gls{os}'s kernel, potentially
leading to reduced performance in both systems by the introduced overhead. On
the side of TEECore, this also requires a way to uphold its security guarantees
while permitting the use of core external resources. Therefore, implementing the
prototype will forbid the usage of core external resources and access to I/O
paths.

\section{Remote Attestation Scheme of TEECore}
\label{sec:30:tee_ra_scheme}
\begin{figure}
    \begin{center}
        \includestandalone{images/30_remote_attestation.tex}
        \caption{Remote Attestation Architecture of TEECore}
        \label{fig:30:tee_ra}
    \end{center}
\end{figure}
As described in section~\ref{sec:20:remote_attestation}, remote attestation
verifies a claim made about the system's state. In the case of TEECore, this
means that TEECore must provide evidence that it holds its defense properties,
was set up correctly and that the trusted application was executed as expected.
To fulfill these tasks, TEECore implements two key components:
\begin{enumerate}
    \item An attestation manager: This is a privileged module of
          TEECore. TEECore implements services an attestation target can use to
          request an attestation report.
    \item Measurement tools: Implemented as a privileged function in TEECore.
          TEECore has insight into a target's memory because of its elevated
          privileges and can thus measure the target's memory. Moreover, it can
          monitor communication, which can be part of the measurement.
\end{enumerate}

To measure a part of the target means to create a hash of it that can be used to
extend a \gls{pcr} in the \gls{tpm}. TEECore uses the \gls{tpm}'s gls{pcr} to
integrate a hardware root of trust. Additionally, the \gls{tpm} is required to
act as a hardware token to identify the platform. With its security features,
the \gls{tpm} forms a trust anchor for external parties by attesting its
genuineness. It furthermore contains the results of the measured boot chain as a
result of a technology called measured boot. A measured boot allows the creation
of a chain of trust, by which each step in the boot chain measures the software
used for the next step. This allows for comparing the hash value with an
expected good value. Additionally, each hash value extends a specific \gls{pcr}
in the \gls{tpm}. Beginning from the firmware, each involved measurement is
added to a measurement log that tracks what component was measured and what
\gls{pcr} was extended. A given \gls{pcr} can then prove the integrity of the
log. TEECore uses this chain in two ways. First, TEECore can verify on the
contents of the \gls{pcr} that its setup phase is performed in a known good and,
therefore, trusted state of the platform. Second, by making the \gls{tpm} attest
to its \gls{pcr} values, TEECore can prove that it started from a trusted state
and that the values that left TEECore were not modified. The latter aspect is
important because it prevents man-in-the-middle attacks. \\

A \gls{tpm} never uses its \gls{ek} to sing or encrypt data. Instead, it uses
key derivation functions to create keys on demand with the help of its
\glspl{ek}. The attestation manager of TEECore requests such a \gls{aik} that
can be used to sign \gls{pcr} values by the \gls{tpm}. A problem arises when
TEECore has to prove that the \gls{aik} originates from a genuine \gls{tpm}. For
this, the signing processes for the \gls{aik} need to be performed with the help
of a \gls{ca} trusted by the appraiser and TEECore. The \gls{aikca} serves this
role. The appraiser and TEECore trust it, and it serves as a proxy to issue a
certificate for TEECore that the \gls{tpm} it uses is genuine and that the
\gls{aik} used by it for attestation of its \glspl{pcr} belongs to its \gls{ek}.
The \gls{aikca} in this process learns the identity of the \gls{tpm} through
it's \gls{ek}'s certificate. On the other hand, the appraiser does not, as it
only knows that the \gls{aikca} issued the certificate. It can, therefore, prove
the correctness of the \gls{aik}'s certificate by proving that it was issued by
the \gls{aikca}.\\

The attestation manager uses the \gls{aikca} and \gls{tpm} to generate a report
on behalf of the attestation target. The attestation manager uses digests
generated by the measurement tool as a third data source. The report created by
the attestation manager contains the following:
\begin{itemize}
    \item The \gls{aik} certificate: The appraise can check with this that the
          \gls{aik} is trusted by the \gls{aikca} and the report data were
          therefore signed by a genuine \gls{tpm}.
    \item The measured boot report together with \gls{pcr} contents: This attests
          to the platform integrity.
    \item \gls{pcr} content of the target measurement results.
    \item The \gls{pcr} value of the report itself.
\end{itemize}

The appraiser is able to verify, based on the attestation report, that the
respective target was executed and what operations it performed. Moreover, the
appraiser can verify that TEECore was run and its integrity was not
violated.

\section{Boot Chain}
\label{sec:30:tee_boot_chain}
\begin{figure}
    \begin{center}
        \includestandalone{images/30_bootflow.tex}
        \caption{Bootchain of TEECore in the Prototype}
        \label{fig:30:tee_bootchain}
    \end{center}
\end{figure}
Figure~\ref{fig:30:tee_bootchain} shows the boot chain as implemented in the
prototype. This boot chain deviates from the boot chain that a standalone
implementation of TEECore would use. In this, the bootloader would load TEECore
so that TEECore can set itself up. After this, TEECore would either instruct the
bootloader to load the commodity \gls{os} or chain load its bootloader.
Implementing the standalone approach would require me to implement bootloader
functionality in TEECore or to extensively modify existing bootloader and boot
chains of supported commodity \glspl{os}. The latter is necessary if TEECore
would be booted before the commodity \gls{os} because this would require a way
to hand the measurement log to TEECore and from TEECore to the commodity
\gls{os}. Both would induce much work without contributing to the ultimate goal
of this thesis: to evaluate the security properties of a \gls{tee} defending
itself with the help of \glspl{pmc}. Because of this, the prototype's design
follows another approach. It uses already-built infrastructure to set up an
environment that is trusted until TEECore is fully set up and running. I leave
the implementation of a standalone version of TEECore to future work.\\

The alternative boot chain as pictured in figure~\ref{fig:30:tee_bootchain}
works by utilizing the commodity \gls{os} to install TEECore. For this, the
driver required by the commodity \gls{os} to communicate to TEECore also sets up
TEECore. Because of its wide use in cloud environments, open-source nature, and
many freely available tools, I decided to use Linux as the commodity \gls{os}
kernel. I will use these terms interchangeably because drivers in the Linux
world are called kernel modules. The first process run by a Linux system is
called the init process. In this prototype, I initialize TEECore and its
respective kernel module in the init process to ensure that no other malicious
activities happen. Using Linux as a test bed also allows me to use already
existing infrastructure that implements measured boot. Moreover, Linux often
uses a minimal image from which it initializes the system. This minimal image,
called initrd, contains all binaries necessary for setting up the system. By
integrating the TEECore image and the kernel module into Linux's initrd, I can
make them part of the measurement of the Linux performed by the bootloader.
Therefore, I do not have to trust the Linux kernel to measure TEECore correctly.
\\

As described in section~\ref{}\todo{habe ich eine für measured boot?}, each
piece of the boot chain measures the software it loads for the next step in the
boot chain and extends a chosen \gls{pcr} in the \gls{tpm} with its measurement.
The root of trust is formed by the root of trust for measurement, which is not
necessarily a part of the firmware but, in all cases, specially protected. With
the knowledge of each component's source and build chains, an appraiser can
generate the same binaries and perform similar measurements. These measurements
and the measured boot log enable the appraiser to verify that each loaded binary
is a known correct version. The appraiser has only to trust the root of trust
for measurement because it can detect all other malicious modifications of other
software components by comparing the respective hash values. Still, the
\gls{tcb} includes all software components loaded by the system before TEECore,
as the measured boot does not protect from software bugs contained in binaries
that are assumed to be safe.

\section{Attacker Model}
\label{sec:30:tee_attacker_model}
The design of TEECore aims mainly to prevent any covert side channel that uses
architectural components to form the side channel. In particular, I aim to
detect cache-based side channels and actively defend against attacks such as
Spectre and Meltdown. The main defense mechanism is the configuration of the
\gls{pmc} and the \gls{idt} that lead to a system reset upon registering an
attack. Because TEECore does not prevent the attack, information leakage is
still possible in the time frame between detection and taking consequences. How
big this time frame is and how an attacker could abuse it is an open question I
will investigate in chapter~\ref{eval}\todo{correct reference}. TEECore also
protects against interrupt-based single-stepping attacks like SGX-Step
(\todo{verwies auf attacks}). With the special configuration of the \gls{idt},
all interrupts received by TEECore lead to the same result: a platform reset. To
summarize, TEECore can defend against an attacker who reads and writes to memory
addresses contained in the cache of the isolated core, as these actions result
in MESI state changes. This attacker can have arbitrary privileges if the attack
is not executed from within the \gls{smm}. Likewise, I/O ports are not
considered an attack vector by me, as the application model of TEECore forbids
access to I/O devices. Moreover, TEECore is designed to run in an environment
with any form of Simultaneous Multithreading, as it can not defend against
sibling threads that share resources of a CPU core.\\

The \gls{smm} serves as an exception from the aforementioned security guarantees
because it can effectively prevent TEECore from reacting to the occurrence of
measurement events. The main reason for this is that \gls{smm} can deny TEECore
from executing and thus reacting to performance monitoring events. Additionally,
Intel allows for freezing debug features while \gls{smm} code is executed. This
feature allows system software for masking events originating from
\gls{smm}.\cite{intel_sdm} Moreover, malicious \gls{smm} software could modify
the \gls{pmc} configuration. To summarize, \gls{smm} could be used to mount an
attack from different vectors. How the actual interaction with \gls{smm} would
look like is, at the point of writing this, not clear. Evaluating such attacks
would require a platform that allows modification of \gls{smm} software and is
therefore left for future work. Another interrupt-based attacker could be the
\gls{pmi} sent by the commodity \gls{os} or any other software to reset the core
that runs the isolated partition. The initialization sequence to bring up
additional application processors in x86 systems is to send a \gls{ipi} sequence
of Init-Startup-Startup. Other than resetting a CPU, in the case of the Init
\gls{ipi}, the Intel SDM states that the state of the cache, the \gls{pmc}, and
their configuration registers remain unchanged. The \gls{idtr} is initialized
with 0. In theory, this should lead to a configuration similar to TEECore's.
Because the \gls{pmc} configuration stays the same, \glspl{pmi} should be
generated, leading to a system reset upon reading data contained in the
receiving core. This attack is checked later in \todo{reference and check if
    this attack leads to a triple fault. Is this even the right spot to think about
    attacks?}.\\

TEECore is immune to other software that spies on its memory access patterns
because all memory accesses happen in the cache. Once the setup phase succeeds,
an attacker could only observe access to the shared memory region that TEECore
and software in the normal partition use to communicate.\\

TEECore can detect if it is running inside a \gls{vm}, or at least if the
platform uses a genuine \gls{tpm}. This would require network access or a
protocol that allows TEECore network communication through software running in
the normal partition. Physical attacks that aim at manipulation or taping the
bus through which the \gls{tpm} is connected to the system can not be prevented
by TEECore. The \gls{tcb} of TEECore consists of software involved in
system initialization. This includes the root of trust for measurement,
firmware, bootloader, commodity \gls{os}, and TEECore. On the hardware side, the
\gls{tcb} consists of the \gls{tpm} and the CPU.

\cleardoublepage

%%% Local Variables: %% TeX-master: "diplom" %% End:
