\chapter{Introduction}
\label{sec:intro}

% Die Einleitung schreibt man zuletzt, wenn die Arbeit im Großen und
% Ganzen schon fertig ist. (Wenn man mit der Einleitung beginnt - ein
% häufiger Fehler - braucht man viel länger und wirft sie später doch
% wieder weg). Sie hat als wesentliche Aufgabe, den Kontext für die
% unterschiedlichen Klassen von Lesern herzustellen. Man muß hier die
% Leser für sich gewinnen. Das Problem, mit dem sich die Arbeit befaßt,
% sollte am Ende wenigsten in Grundzügen klar sein und dem Leser
% interessant erscheinen. Das Kapitel schließt mit einer Übersicht über
% den Rest der Arbeit. Meist braucht man mindestens 4 Seiten dafür, mehr
% als 10 Seiten liest keiner.
\todo{Wenn's geklärt ist unbedingt die richtige Verwendung von We/I prüfen!}
\section{Motivation}
\label{sec:10:motivation}
Remote attestation is the ability of a trusted execution environment to prove claims about its state to an appraiser
over a computer network. \cite{coker_principles_2011} What first sounds like a solution to a rather abstract problem can
quickly prove helpful when considering the following example.
We find our example in the implementation of the automatic contact discovery of the Signal messenger. Signal itself is a
messenger that values the privacy of its users and the confidentiality of their communication. In doing so, Signal
implements a secure end-to-end encryption protocol to protect its users' chats.\cite{cohn2020formal}
To increase the user experience, Signal saw itself confronted by the problem of implementing a privacy-preserving way
for automatic contact discovery.\cite{SignalCd} The implementers were confronted with multiple problems. The first major
problem was the social graph. The social graph can either be centralized and stored on Signal's servers or decentralized
and stored on the user's phone. To minimize data stored on Signal's servers about their users, the implementers decided
to use the user's phone book. The second problem arises from this decision. The user's phone books must be processed so
that neither Signal nor their server operators can learn anything about the users.
To solve these problems, encrypting the phone books is not enough. For example, the user does not have any chance to
conclude from the communication alone with whom they are communicating. Even if the communication partner proves with
the help of signatures that they are the Signal server application, the user cannot verify what version of the server
application is running or if it processes the data as expected. Until now, we have not considered the server operator,
which might be malicious, too. They can manipulate and read the application's memory because they have privileged access
to the server's software and hardware. Such a powerful adversary could read decrypted secrets directly from
memory or modify the server application in a way that leaks secret data.
In this example, the user wants to verify that the Signal server application adheres to the following claims:
\begin{enumerate}
    \item The Signal server application executes the code the user expects
    \item The Signal server application is safe from being manipulated by privileged third parties
    \item Privileged third parties are unable to read the server applications' memory
\end{enumerate}
The server environment, therefore, needs a trusted security authority that can prove these claims to the users.
Additionally, hardware isolation mechanisms must protect this security authority against tampering attempts of
privileged third parties. Trusted execution environments solve this
problem by integrating two building blocks. First, they can execute code in an isolated execution environment.
Hardware enforces isolation, and interaction with the isolated code is only possible through defined interfaces of
the isolating hardware. This way, privileged software is not able to access memory-isolated code. Additionally, to
protect against memory bus-taping attacks, isolation hardware often encrypts the memory. The second building block
implements remote attestation. Through remote attestation, TEEs can verify that the execution environment runs the task
under hardware-assisted isolation protection.

If the TEE fulfills the aforementioned claims, it is safe to use. Moreover, most commodity CPUs currently
implement technologies that enable the use of TEEs. In that case, the CPU and its vendor act as
security authorities.\cite{tdx_whitepaper,kaplan_amd_2020,pinto_demystifying_2019,costan2016intel}
The story could be finished at this point if it was not the case that the respective CPU implementations contain bugs
that make it possible for attackers to circumvent their hardware-assisted isolation mechanisms and leak secrets through
side channels.\cite{kocher_spectre_2020,lipp_meltdown_2020,nilsson_survey_2020} While software updates could mitigate
specific side-channel attacks, the source of the information leaks is still persistent in hardware, with researchers
still finding new side-channel attacks each year.\cite{wikner2022retbleed,moghimi_downfall_2023,ragab_ghostrace_2024}
On the other hand, the example of Signal elucidates that proper working TEEs are important because they might process
critical data. It is, therefore, crucial to find a way to allow the execution of programs in a completely isolated and
side-channel-free environment.

\section{Contributions}
\label{sec:10:contributions}
As described in the motivation (see chapter~\ref{sec:10:motivation}) for this
thesis, we see a big problem in the general vulnerability by side channel
attacks on commodity processors. To fully exclude side channels, we propose
<Fancy Name\todo{think about a really fancy name}> to tackle the following
goals:

\begin{enumerate}
    \item We create an execution environment fully isolated from the remaining
          system by utilizing features found in all commodity CPUs. This means
          that we do not use any vendor-specific ISA extensions.
    \item We propose a TEE design to detect side-channel attacks and react
          appropriately.
    \item Our design ensures the usage of only core local resources. This means
          we implement a policy that is restricted to core local caches.
    \item We evaluate ways to enable such an execution environment to offer
          remote attestation features.
    \item We evaluate our implementation regarding system resources. In detail,
          we will evaluate what workload fits next to the local cache's runtime
          environment.
    \item We evaluate the TCB and against what attacker model our sole software
          solution can defend.
\end{enumerate}

With complete isolation and the exclusion of possible side channels, we aim for
a TEE solution that can withstand a wide array of possible attacks. Because of
time constraints in this thesis, we cannot implement a fully working TEE ;
instead, we can show the properties and concepts on a working proof of concept
implementation. We show that our solution can run next to a commodity operating
system such as Linux, implement communication channels through a well-defined
interface, and show that we can protect our implementation from being interfered
with by the commodity OS. Because we explore the technical feasibility, our
solution still depends on the commodity to install the environment.

\section{Structure}
\label{sec:10_structure}
In chapter~\ref{sec:state}, we will discuss technical Grundlagen that is important to follow the rest of the work. The first part
~\ref{sec:20:technical} we will discuss some details of the x86 microarchitecture. These details will allow us to understand the attacks
and what countermeasures to implement in the remaining parts of the thesis. Section~\ref{sec:20:tee} will then discuss the state of
the art by reviewing related work. In this, we will look at how trusted execution environments are implemented.
First, we will review hardware extensions followed by software solutions or software that enhances the security of
hardware extensions.

In chapter~\ref{sec:design}, I will explain the design of my implementation. In particular, we will explain in~\ref{sec:30:system} the general
design. In~\ref{sec:30:attacker}, we will revisit the attacker model and what design feature can be used to mitigate which attack.

Chapter~\ref{sec:implementation} will explain how I implemented the design proposed in~\ref{sec:design}. We will look at specific parts of
the implementation that are important to fulfill the security guarantees. Moreover, we will investigate what part
of the design was problematic to implement.

In chapter~\ref{sec:evaluation}, we will revisit the implementation described in chapter~\ref{sec:implementation}. We will evaluate the implementation in
chapter~\ref{sec:50:security} regarding how it protects against the attacker model described in~\ref{sec:30:attacker}. Moreover, we will evaluate
the practical use of our design in~\ref{sec:50:security}.

After we evaluate the implementation, we will look at missing parts and how to improve the implementation further. We
will do all of this in section~\ref{sec:futurework}, which gives an outlook on future work.

To come to terms with our results, we will give a short overview and summary of the results and findings of our work in
chapter~\ref{sec:conclusion}.


%%% Local Variables:
%%% TeX-master: "diplom"
%%% End:
