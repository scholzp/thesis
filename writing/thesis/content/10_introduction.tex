\chapter{Introduction}
\label{sec:intro}

% Die Einleitung schreibt man zuletzt, wenn die Arbeit im Großen und
% Ganzen schon fertig ist. (Wenn man mit der Einleitung beginnt - ein
% häufiger Fehler - braucht man viel länger und wirft sie später doch
% wieder weg). Sie hat als wesentliche Aufgabe, den Kontext für die
% unterschiedlichen Klassen von Lesern herzustellen. Man muß hier die
% Leser für sich gewinnen. Das Problem, mit dem sich die Arbeit befaßt,
% sollte am Ende wenigsten in Grundzügen klar sein und dem Leser
% interessant erscheinen. Das Kapitel schließt mit einer Übersicht über
% den Rest der Arbeit. Meist braucht man mindestens 4 Seiten dafür, mehr
% als 10 Seiten liest keiner.

\cleardoublepage

%%% Local Variables:
%%% TeX-master: "diplom"
%%% End:
