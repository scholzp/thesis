\chapter{Conclusion And Outlook}
\label{sec:conclusion}

%  Schlußfolgerungen, Fragen, Ausblicke

% Dieses Kapitel ist sicherlich das am Schwierigsten zu schreibende. Es
% dient einer gerafften Zusammenfassung dessen, was man gelernt hat. Es
% ist möglicherweise gespickt von Rückwärtsverweisen in den Text, um dem
% faulen aber interessierten Leser (der Regelfall) doch noch einmal die
% Chance zu geben, sich etwas fundierter weiterzubilden. Manche guten
% Arbeiten werfen mehr Probleme auf als sie lösen. Dies darf man ruhig
% zugeben und diskutieren. Man kann gegebenenfalls auch schreiben, was
% man in dieser Sache noch zu tun gedenkt oder den Nachfolgern ein paar
% Tips geben. Aber man sollte nicht um jeden Preis Fragen, die gar nicht
% da sind, mit Gewalt aufbringen und dem Leser suggerieren, wie
% weitsichtig man doch ist. Dieses Kapitel muß kurz sein, damit es
% gelesen wird.

In this thesis, I presented TEECore, a \gls{tee} framework that aims to be
side-channel-free. TEECore prevents side channels by utilizing only CPU core
exclusive resources of multi-core CPUs. This approach ensures no communication
paths to other cores of the same CPU exist. I have shown that in a carefully
crafted environment, \glspl{pmc} can be used to detect side channels reliably.
By sending interrupts on information breaches, TEECore can react to attacks. At
this point, it is important to differentiate between preventing and reacting to
side channels. TEECore does not protect from side-channel attacks. As I have
shown, this comes from the cache coherency algorithms implemented by the Intel
Core i7 14700k I used for testing, which makes it possible for carefully crafted
attacks to retrieve data. Furthermore, I demonstrated that TEECore, while able
to detect side channels and create a widely free interrupt environment, is still
vulnerable to x86 INIT \gls{ipi}. This \gls{ipi} transfers the CPU running
TEECore to a state that makes it impossible for TEECore to respond to any
attack. On its own, TEECore, therefore, is not suitable to protect data.
Nevertheless, as I argued in chapter~\ref{eval:compare:security}, TEECore could
assist in protecting other \gls{tee} solutions and make them detect side-channel
attacks.

\cleardoublepage

%%% Local Variables:
%%% TeX-master: "diplom"
%%% End:
