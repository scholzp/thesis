\chapter{Conclusion And Outlook}
\label{sec:conclusion}

%  Schlußfolgerungen, Fragen, Ausblicke

% Dieses Kapitel ist sicherlich das am Schwierigsten zu schreibende. Es
% dient einer gerafften Zusammenfassung dessen, was man gelernt hat. Es
% ist möglicherweise gespickt von Rückwärtsverweisen in den Text, um dem
% faulen aber interessierten Leser (der Regelfall) doch noch einmal die
% Chance zu geben, sich etwas fundierter weiterzubilden. Manche guten
% Arbeiten werfen mehr Probleme auf als sie lösen. Dies darf man ruhig
% zugeben und diskutieren. Man kann gegebenenfalls auch schreiben, was
% man in dieser Sache noch zu tun gedenkt oder den Nachfolgern ein paar
% Tips geben. Aber man sollte nicht um jeden Preis Fragen, die gar nicht
% da sind, mit Gewalt aufbringen und dem Leser suggerieren, wie
% weitsichtig man doch ist. Dieses Kapitel muß kurz sein, damit es
% gelesen wird.

In this thesis, I presented TEECore, a \gls{tee} framework that aims to be able
to detect side channel attacks. TEECore prevents side channels by utilizing only
CPU core exclusive resources of multi-core CPUs, as described in
chapter~\ref{sec:30:tee_kernel}. This approach ensures no communication paths to
other cores of the same CPU exist. In chapter~\ref{eval:sec}, I have shown that
in a carefully crafted environment, \glspl{pmc} can be used to detect side
channels reliably. TEEcore can react to attacks by programming the performance
counting facility to send \glspl{pmi} on the occurrence of such events as shown
in chapter~\ref{sec:implementation:teeKernel}. At this point, it is essential to
differentiate between preventing and reacting to side channels. TEECore does not
protect from side-channel attacks. As I have shown in
section~\ref{sec:evaluation:ipi}, this comes from the cache coherency algorithms
implemented by the Intel Core i7 13700k I used for testing. This enables an
attacker to carefully craft attacks to retrieve data. This is exceptionally
critical as software running in the normal partition can also accidentally
interrupt TEECore with an INIT \gls{ipi}. These interrupts are part of the
regular platform initialization routine and can be broadcast to all CPUs as a
shortcut. Privileged software in the normal partition must therefore be adapted
to the use of TEECore, to prevent it from accidentally attacking TEECore.
Nevertheless, as I argued in chapter~\ref{eval:compare:security}, TEECore could
assist in protecting other \gls{tee} solutions and make them detect side-channel
attacks.

%%% Local Variables:
%%% TeX-master: "diplom"
%%% End:
