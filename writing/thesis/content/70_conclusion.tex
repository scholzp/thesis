\chapter{Conclusion And Outlook}
\label{sec:conclusion}

%  Schlußfolgerungen, Fragen, Ausblicke

% Dieses Kapitel ist sicherlich das am Schwierigsten zu schreibende. Es
% dient einer gerafften Zusammenfassung dessen, was man gelernt hat. Es
% ist möglicherweise gespickt von Rückwärtsverweisen in den Text, um dem
% faulen aber interessierten Leser (der Regelfall) doch noch einmal die
% Chance zu geben, sich etwas fundierter weiterzubilden. Manche guten
% Arbeiten werfen mehr Probleme auf als sie lösen. Dies darf man ruhig
% zugeben und diskutieren. Man kann gegebenenfalls auch schreiben, was
% man in dieser Sache noch zu tun gedenkt oder den Nachfolgern ein paar
% Tips geben. Aber man sollte nicht um jeden Preis Fragen, die gar nicht
% da sind, mit Gewalt aufbringen und dem Leser suggerieren, wie
% weitsichtig man doch ist. Dieses Kapitel muß kurz sein, damit es
% gelesen wird.

\ldots conclusion \ldots

\todo{write conclusion}

\cleardoublepage

%%% Local Variables:
%%% TeX-master: "diplom"
%%% End:
