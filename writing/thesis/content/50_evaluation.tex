\chapter{Evaluation}
\label{sec:evaluation}

% Zu jeder Arbeit in unserem Bereich gehört eine Leistungsbewertung. Aus
% diesem Kapitel sollte hervorgehen, welche Methoden angewandt worden,
% die Leistungsfähigkeit zu bewerten und welche Ergebnisse dabei erzielt
% wurden. Wichtig ist es, dem Leser nicht nur ein paar Zahlen
% hinzustellen, sondern auch eine Diskussion der Ergebnisse
% vorzunehmen. Es wird empfohlen zunächst die eigenen Erwartungen
% bezüglich der Ergebnisse zu erläutern und anschließend eventuell
% festgestellte Abweichungen zu erklären.

\ldots evaluation \ldots

\todo{write evaluation}
- measure time between attack registration and shutdown
\begin{itemize}
    \item Evaluate memory constraints on ping task
    \item evaluate what of the attacks were detected
    \item Include measurements of the PMC events for each evaluation task
    \item measure time between attack registration and shutdown
\end{itemize}

\cleardoublepage

%%% Local Variables:
%%% TeX-master: "diplom"
%%% End:
