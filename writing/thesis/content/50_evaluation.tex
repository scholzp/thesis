\chapter{Evaluation}
\label{sec:evaluation}

% Zu jeder Arbeit in unserem Bereich gehört eine Leistungsbewertung. Aus
% diesem Kapitel sollte hervorgehen, welche Methoden angewandt worden,
% die Leistungsfähigkeit zu bewerten und welche Ergebnisse dabei erzielt
% wurden. Wichtig ist es, dem Leser nicht nur ein paar Zahlen
% hinzustellen, sondern auch eine Diskussion der Ergebnisse
% vorzunehmen. Es wird empfohlen zunächst die eigenen Erwartungen
% bezüglich der Ergebnisse zu erläutern und anschließend eventuell
% festgestellte Abweichungen zu erklären.

In the following section, I evaluate TEECore's properties regarding its security
and the constraints it imposes on workload tasks. For this, I use a slightly
modified version of TEECore that allows me to gather data. For example, I
replaced the \gls{idt} to implement rudimentary interrupt handling routines.
These routines do nothing more than to ensure that all debug data is transferred
and prevent the system from being reset. Nevertheless, the implemented interrupt
handling routines ensure that TEECore becomes unresponsive after receiving an
interrupt, by executing an endless loop. To further evaluate the content of
\glspl{pmc}, I configure them to not yield any \gls{pmi}. This configuration
allows me to investigate the behavior of the CPU cache in more detail. TEECore
prints this information through a serial connection.\todo{If this worked out:
Write about serilazing instructions}\\

The test environment I used for the following evaluation consists of an Intel
Core i7 14700k processor. It is part of the Intel Raptor Lake hybrid
microarchitecture and ships with 20 CPU cores of two different kinds. The first
kind of CPU cores implement the Raptor Cove microarchitecture and are called
performance Cores (P-cores). In contrast, the second kind of cores implement the
Intel Gracemont microarchitecture and are called efficiency cores (E-Cores). In
the following sections, I will focus on the P-Core implementation, as these
cores offer more core-exclusive cache than the E-Cores. The Intel Core i7 14700k
has 8 performance cores. To ensure that tests run only on P-Cores, I disabled
E-Cores through UEFI settings. Moreover, I disabled the Hyper-Threading feature
of the P-Cores, so they only provide one logical thread instead of two. P-Cores
provide an 80 KiB-sized L1 cache, which is divided into a 32 KiB-sized
instruction and a 48 KiB-sized data cache. The second level cache (L2) has a
size of 2 MiB. It is a unified cache, meaning that it is not divided into data
and instruction parts. The third and, thus, last level cache is shared among all
CPU cores. It is implemented as core-associated slices with a size of 3 MiB
each, summing up to a total of 20 MiB of shared L3 cache. All caches are
inclusive, meaning they contain all data of the lower cache levels
(see~\ref{sec:state:technical:caches_inclusivity}). Moreover, the cache line
size in this CPU is 64 bytes.

\section{Security Properties}
\label{eval:sec}
To prove that TEECore can detect and defend against different attacks, I
implemented multiple attack simulations to test different attack vectors. In all
tested cases, I assume the attacker to have privileged rights as described in
section~\ref{sec:30:tee_attacker_model}. The attacker can, therefore, do
everything the privileged software in the normal partition is capable of.\\

\subsection{Passive Attacker}
\label{sec:evaluation:passive}
The first test cases test basic functionality to detect attacks that use memory
access and the inclusive caches as an attack vector. These attacks are to be
detected by TEECore through changes in the cache lines that result from the
cache coherency protocol. I found two possible test cases, based on the behavior
of an attacker. An attacker either tries to read or write the memory used by
TEECore. Because the cache coherency protocol ensures the reader sees the
correct data, an attacker could exfiltrate secrets. By writing to memory used by
TEECore, which in turn would update TEECore's memory, an attacker could inject
data into the isolated environment. To simulate these kinds of attacks, I
implemented a small trusted application that allocates memory of the size of 4
KiB, which I call the target memory. The trusted application then communicates
the physical address of the target memory to the normal partition using the
shared memory communication path. Upon receiving the physical address of the
target memory, the kernel module maps it into its address space and performs the
respective operation. After mapping the target memory in the Linux kernel
module, the CPU remote to TEECore has not yet been interacting with it, and it
is not stored in a remote cache. Only the cache of the CPU core running TEECore,
to which I refer as the target core, has the target memory cached. After the
initialization phase (see section~\ref{sec:30:tee_kernel}), the target memory is
at least in the modified state in this cache. Upon reading the target memory,
the kernel module's CPU core receives a copy of the data. Both CPU cores
maintain their copy in the shared state. The level 3 cache is involved as a
communication path between both cores, which leads to the target memory
being\todo{Prüfen ob auch übertragungen von remote l1/l2 caches kommen können,
ohne dass L3 involviert ist.} moved from the L2 to the L3 cache. Therefore, the
expected result of this test is to see L2 cache misses and L3 cache hits in the
first attempt by TEECore to access the target memory after the remote core has
read it. The attacker, in this case, does not modify the memory, so the
performance counters do not trigger for subsequent access by the remote core as
long as the target core does not modify the target memory. Concerning the number
of L2 misses and L3 hits, I expect to count 64 occurrences per event. The number
of occurrences results from the fixed size of the target memory of 4096 Bytes.
Counting a combination of fewer L2 misses and L3 hits would mean that data was
not shared among cores. After running the test, the results fully met my
expectations. For the first time, the remote core reads the target memory, and
TEECore registers 64 events. After the second read, TEECore does not register
any more events. The result demonstrates the importance of the cache line state
transition to either exclusive or modified in the initialization phase of
TEECore. This test showed a shortcoming of TEECores protection capabilities in
an early implementation phase because an attacker that only reads could not be
detected without the initial write to all memory.\\

\subsection{Active Attacker}
\label{sec:evaluation:active}
The simulation for an attacker that writes to the target memory is similar in
setup but different in its effects. As with the read attack, the secure
application shares the physical address of the target memory with the Linux
kernel module. In contrast, the kernel module now writes to the target memory.
As a result, the cache line, which was in a modified state in the target CPU
core's cache, now changes to invalid because the remote CPU has the correct copy
in its cache. If the target CPU now reads or writes the target memory, it has to
query the changes from the remote CPU. The cache line changes from invalid to
shared or modified, dependent on the operation of the target CPU. This change is
accompanied by changes in the L3 cache, which triggers the \gls{pmc} events.
Subsequently, I should measure one event per affected cache line, totaling 64
events counted. In contrast to the reading attack, performing writes on the
remote side and reads on the target side should yield events because of the
involved cache line state modifications for each operation. TEECore shows
exactly this behavior. It, therefore, can detect the attacker described.\\

\subsection{False Positives}
\label{sec:evaluation:fp}
Another aspect is the number of false positives of events that TEECore detects.
For this, I used another modification of the attack described before. Once
again, the secure application allocates the target memory and transmits the
physical address of its page frame to the kernel module. The kernel module then
does nothing more than map the target memory and return control to TEECore. As a
result, TEECore should measure neither L2 cache misses nor L3 cache hits.
Repeating this test in 10 iterations showed that TEECore did not measure any
false positives. All \glspl{pmc} measured zero occurences.\\

\subsection{Inter-processor Interrupts}
\label{sec:evaluation:ipi}
As an additional attack vector, I identified interrupts. For Interrupts such as
the default vectors, \glspl{nmi} and \glspl{pmi}, which are routed through the
\gls{idt}, the countermeasures described in~\ref{sec:30:tee_kernel} are taken.
As a result the \gls{idt} configuration of TEECore ultimately leads to a triple
fault and, thus, a platform reset. Sending an \gls{nmi} from the kernel module
to the target core resulted in the expected behavior. As described in
\todo{WO?!?!?} \glspl{ipi} such as INIT are not routed through the \gls{idt}.
Instead, the \gls{lapic} sends these directly to the CPU. As a result, TEECore
cannot reset the platform upon receiving an INIT \gls{ipi}. Instead, the CPU
handles the INIT as described in the Intel SDM. The target CPU, therefore,
changes into a state that stops program execution. It furthermore ignores all
interrupts but STARTUP \gls{ipi} messages. Malicious system software running on
any remote core can effectively starve TEECore and prevent its execution. The
remaining question is whether the target core still participates in the cache
coherency protocol. If so, a remote core can retrieve memory content from the
secure partition without allowing TEECore to detect this exfiltration. To test
this, I used the same setup as in the other tests on the side of TEECore. The
trusted application allocates memory, writes a predefined value to it, and then
transmits the physical address to the kernel module. Before the kernel module
reads the target memory, it sends a INIT \gls{ipi} to the target core. Comparing
the read value with the one written by the trusted application before allows me
to conclude if the attack was successful. The test indeed showed that TEECore is
vulnerable to this kind of attack. Because the target core is halted, TEECore
can neither detect nor prevent such attacks.

\section{Memory Constraints}
\label{eval:mem_constraints}
In this section, I evaluate what memory constraints exist for TEECore and its
secure applications. From a theoretical, abstract point of view, TEECore should
not use more memory than the size of the largest private cache. For my test CPU,
this means that TEECore should, at most, use 2 MiB of memory. The difference
between this cache size and the number of cache lines used by TEECore should
yield the maximum number of cache lines a payload could use in the form of a
secure application. In practice, memory alignment can restrict the actual number
of usable cache lines for TEECore. Because Raptor Cove's L1D and L2 cache
associativity is 12, respectively 16, not all possible memory addresses can be
stored in all cache lines. Table~\ref{50:tab:cache_size} lists Raptor Cove's
cache parameters as stated in the official data
sheet~\cite{raptorlake_spec_sheet}. The L2 and L3 caches in Raptor Cove are
non-inclusive.\\

\begin{table}[ht]
  \centering
  \begin{tabular}{ |l||c|c|c|c|c| }
    \hline
    Array Size & Size   & Associativity   & Line Size & Number of Lines & Lines per Set \\
    \hline
    L1I Cache  & 32 KiB & 8               & 64 Bytes  & 512             & 64            \\
    L1D Cache  & 48 KiB & 12              & 64 Bytes  & 768             & 64            \\
    L2 Cache   & 2 MiB  & 16              & 64 Bytes  & 32,768          & 2,048         \\
    L3 Cache   & 3 MiB  & 12              & 64 Bytes  & 49,152          & 4,096         \\
    \hline
  \end{tabular}
  \caption{Cache Parameters of Intel Raptor Cove Microarchitecture}
  \label{50:tab:cache_size}
\end{table}

To evaluate memory constraints, I will use a simple trusted application which
allocates all available memory. After allocating memory, this application access
the memory by first reading and the writing an incremented value back the each
64 bytes field. These memory accesses are those that I measure. The application
does no communication with the normal partition. I used different performance
counter events to evaluate different aspects of TEECore. These events are shown
in table~\ref{50:tab:events}. For the following tests I differentiate between
high-level Rust code, that is used to program the run time environment and
low-level assembly code that is run before the high-level code to setup the
environment.

\begin{table}[ht]
  \centering
  \begin{tabular}{ |p{6cm}|p{1.35cm}|p{1.25cm}|p{4cm}|}
    \hline
    \makecell[l]{Intel Perfmon Event Name \\ (Abbreviation)} & Selector & UMask & Description                                                                      \\
    \hline
    \makecell[l]{FRONTEND\_RETIRED.L1I\_MISS\\ (L1I\_MISS)}  & 0x80     & 0x04  & Counts cycles where a code line fetch is stalled due to an L1 instruction cache miss. The decode pipeline works at a 32 Byte granularity. \\
    \makecell[l]{L1D.REPLACEMENT \\ (L1D\_REPLACEMENT)}      & 0x51     & 0x01  & Counts cache lines replaced into the L0 and L1 d-cache.                          \\
    MEM\_LOAD\_RETIRED.L2\_HIT (L2\_HIT)                     & 0xD1     & 0x02  & Counts retired load instructions with L2 cache hits as data sources.             \\
    MEM\_LOAD\_RETIRED.L2\_MISS (L2\_MISS)                   & 0xD1     & 0x10  & Counts retired load instructions with at least one uop that hit in the L3 cache. \\
    \hline
  \end{tabular}
  \caption{Performance Monitoring Events for evaluating memory constraints}
  \label{50:tab:events}
\end{table}
\FloatBarrier

\subsection{Code Size}
\label{sec:evaluation:mem:code}
To measure the size of the run time environment I use the event
\textit{L1I\_MISS}. It serves as an indicator of TEECore's usage of the
instruction cache. For this, I modified the low-level setup code to program a
\gls{pmc} to measure \textit{L1I\_MISS} right before transferring control to the
high-level code. Without any warmup run, L1I misses yield the number of lines
that were fetch to execute TEECore. This effectively tells us the size of
TEECore's code. This event only counts one miss per cache line and requires me
to additionally program a PEBS \gls{msr}. Because the code was never run before,
I expect the counter to be not zero. Table~\ref{50:tab:code_size} shows the
measurement results.

\begin{table}[ht]
  \centering
  \begin{tabular}{ |l||c| }
    \hline
    Event            & Value \\
    \hline
    L1I\_MISS        & 29 \\
    \hline
  \end{tabular}
  \caption{Unmodified Test Case results}
  \label{50:tab:code_size}
\end{table}

In total, xxx\todo{Zahlen bitte} lines were replaced in the L1 instruction
cache. This means, TEECore requires yyy bytes for its code.

\subsection{Total Memory Consumption}
I use the remaining event to evaluate TEECore with respect to memory constraints
implied by the hardware it runs on. The event \textit{L1D.REPLACEMENT} indicates
the number of times a cache line in the L1 data cache was replaced. This is the
case whenever the CPU needs to swap lines between the L1 cache and any other
resource. After a warm up run, any values not zero indicate that TEECore uses
more than 48 KiB of cache for data. The event
\textit{MEM\_LOAD\_RETIRED.L2\_HIT} indicates that TEECore is using more memory
than the L1 data and instruction cache can provide. Events of type
\textit{MEM\_LOAD\_RETIRED.L2\_MISS} indicate that TEECore is using more than 2
MiB and, therefore, requires more memory than the L1 and L2 caches can provide.
This event is also useful to make claims about the effectiveness about TEECore's
initialization phase. The \textit{MEM\_LOAD\_RETIRED.L2\_MISS} event further
indicates if memory is spilling into shared resources and, therefore, indicates
that TEECore can no longer uphold its security guarantees. To summarize, any
occurrences of this event indicates that TEECore uses shared resources.
Therefore, I regard any occurrence as unacceptable. In the remaining part of
this chapter, I will refer to all events by their abbreviation as listed in
table~\ref{50:tab:events}.

The remaining parts of this section split into two parts. In
section~\ref{eval:mem_constraints:influences}, I investigate what effects the
implementation of Raptor Cove's cache has on TEECore. In
Section~\ref{eval:mem_constraints:size} I evaluate results of the performance
counters for different allocated sizes.
\FloatBarrier

\subsection{Effects of Cache Implementation on Measurements}
\label{eval:mem_constraints:influences}
While evaluating the constraints the cache implementation imposes on TEECore, I
found that the initialization routine did not work as intended. In this section
I will explain why additional steps are necessary to correctly implement
TEECore's initialization routine. Table~\ref{50:tab:init} shows performance
counter results for repeatedly accessing the memory allocated by the test
application.
\begin{table}[ht]
  \centering
  \begin{tabular}{ |l||c|c|c|c| }
    \hline
    Event        & MEM\_OPS\_TOTAL & L1D\_REPLACEMENT & L2\_HIT & L2\_MISS \\
    \hline
    1st Pass   & 3a815 & 7503 & 488 & 7061         \\
    2nd Pass  & 3a815 & 7503 & 1c79 & 5709          \\
    3rd Pass   & 3a815 & 7504 & 623a & 1119         \\
    4th Pass  & 3a815 & 7504 & 7371 & 0          \\
    \hline
  \end{tabular}
  \caption{Performance Monitoring Result without Initialization Phase}
  \label{50:tab:init}
\end{table}

To reproduce the effect of the initialization phase, I ran and measured the
memory access in multiple passes. Each pass performs the same memory operations
as described in section~\ref{eval:mem_constraints}. To prove that all passes
perform the same number of memory operations, I replaced the instruction event
with one that measures the total number of memory operations (MEM\_OPS\_TOTAL).
The test application operates on the maximum available memory, that is, 1,876
KiB of consecutive memory. As described in the design
chapter\ref{sec:implementation:teeKernel}, the expected behavior would be that
after the initialization phase, no L2 misses occur. In this test, the
initialization phase is emulated by the first pass. As the measurement results
in table~\ref{50:tab:init} show, this is not the case. Instead, L2 misses occur
until the fourth pass.\\

Because TEECore uses 2 MiB of consecutive memory, which is aligned at a 2 MiB
boundary, no conflict misses can occur. To create conflict misses in the L2
cache, TEECore would need to access more pages that hash to the same cache tag
than the number of sets would allow for. This is only the case if TEECore maps
memory whose physical addresses lie 2 MiB apart. Another point that contradicts
the idea that conflicts cause these misses is the fact that the number of
conflicts decreases with the number of passes. If the cause was conflict misses,
repeated access to the same memory with the same access pattern would yield the
same amount of conflict misses.\\

Further investigation shows that the cache replacement policy can have an
influence that I did not account for in my design and initial implementation.
Research suggests that Intel implements a replacement policy called \gls{qlru}
in the L2 and L3 caches of their recent processors \cite{briongos2020reload+,
Abel20b}. With this algorithm, each cache line has an age state stored in two
bits. The minimum age is thus zero, and the maximum age is three. Concerning the
implementation, either the leftmost or rightmost line in a set is replaced that
has an age of three. If no lines have an age of three, the hardware increments
all age counters by one. It now searches for a line with age 3 to replace. New
lines are initialized with an age of 0 in Icelake processors. Additionally, the
\textit{WBINVD} instruction does not reset the age state~\cite{Abel20b}. As
Icelake is the successor of Raptor Cove, I assume the same to be the case for
Raptor Cove. This assumption fits the observed caching behavior.\\

\begin{figure}
  \begin{center}
    \includestandalone{images/qlru.tex}
    \caption{Example of Cache Line Replacement using the \gls{qlru} Algorithm in a Cache Set}
    \label{fig:50:qlru}
  \end{center}
\end{figure}

A simplified example that needs two initialization passes is shown in
Figure~\ref {fig:50:qlru}. Assume we have a working set of data $W$ that
consists of $W = \{w_1, w_2, w_3, w_4\}$ and a second set of data $I$ that fills
the cache. The initial data set $I = \{i_1, i_2, i_3, i_4\}$ was loaded before
the initialization phase to the cache, e.g., while executing boot code. The
topmost box shows the initial state before running each pass in Figure~\ref
{fig:50:qlru}. First, the cache only contains items from the initial data set
with different age states. Upon accessing $w_1$, a cache miss is generated.
Because the age of $i_3$ is three, the cache replaces it with $w_1$. The initial
age of any item brought to the cache is 1. The second access is done to $w_2$.
Again, a cache miss occurs. Because no item has an age of 3, all age states are
incremented. As a result, $i_1$ reaches an age of 3 and is replaced by $w_2$
with an age of 1. When accessing $w_3$, the cache misses again. As a result, all
ages are incremented again, and $i_3$ and $w_1$ reach an age of 3. The leftmost
item, which is $i_3$, is replaced by $w_1$ and an age of 1. Because now $w_1$
has reached the age of 3, it is now replaced by $w_4$ in the next step. After
the first pass the cache contains $w_2, i_2, w_3$ and $w_4$ although all items
of $W$ were accessed. To fill all cache lines with data from $w$, a second pass
is required. The first two accesses work in the same manner as before. The last
two accesses hit the cache. They result in the reduction of the age of the
respective items.\\

Due to the possibility of cache items having an age of zero and a maximum age of
3, at least three initialization passes are necessary to ensure that the age of
all items already residing in the cache reaches the maximum value. A fourth run
then ensures that all of those items are replaced with values from the working
set.

\FloatBarrier
\subsection{Effects of increasing Working Set}
\label{eval:mem_constraints:size}

I measure two scenarios for all test cases. For the first scenario (SC1), I will
measure all high-level code TEECore. The high level code includes all program
code written in Rust, but excludes all assembly code required for transferring
the CPU to a state that can execute TEECore. This scenario reflects the memory
usage of the current implementation. For the second scenario (SC2), I will
measure all code from the beginning of the execution of the test application.
This test case reflects the minimum required memory to run the application. The
difference between SC1 and SC2 in memory usage reveals potential for
optimization. To measure SC2, I invalidate the whole caches by using the
\textit{WBINVD} instruction. \\

My first test case is the baseline. I run the unmodified test application to
identify the minimal memory consumption for both scenarios.
Table~\ref{50:tab:ping_base} shows the measurement results. \todo{add values,
evaluate}

\begin{table}[ht]
  \centering
  \begin{tabular}{ |l||c|c|c| }
    \hline
    Event            & SC1 & SC2 & Difference \\
    \hline
    L1I\_MISS        & 29  & 9   & 20         \\
    L1D\_REPLACEMENT & 66  & 32  & 34         \\
    L2\_HIT          & 44  & 22  & 22         \\
    L2\_MISS         & 0   & 0   & 0          \\
    \hline
  \end{tabular}
  \caption{Unmodified Test Case results}
  \label{50:tab:ping_base}
\end{table}

For the unmodified test case, we can see a clear difference between measuring
the application together with TEECore initialization (SC1) and the application
alone (SC2). All measured values stay consistent after the first test run. This
indicates, that after bringing the values first to the L1 cache, no further
replacements are required. It is therefore save to asume, that TEECore together
with this example application fits the L1 cache. We can see that the difference
between SC1 and SC2 in L1 misses is nearly equal to the number of L2 cache hits.
This comes from the fact, that the code resides in the L2 cache after
initialization. Because it was not yet run, it was not loaded by the CPU to its
L1 instruction cache. With the first time running the code, the CPU misses in
its L1I cache and loads it from L2 cache to L1. Furthermore, we can see that
L1D\_Replacements do not indicate missing data from the L1 cache, as L2 hits do
not increase in a similar manner as the L1D replacements. A possible cause for
this can be, that the CPU modifies data resident in L1D cache and store it in
its bill buffer. In subsequent access, The CPU does not hit in the L1D cache but
in the fill buffer, triggering a replacement of the line in L1D cache. No hits
in the L3 cache signal that the whole environment's memory consumption is below
2 MiB. TEECore alone has a memory consumption of around 2,600 bytes, while a
trusted application requires a minimum of 1,400 bytes of memory. This makes up
around of 5\% of available resources, so it is save to assume that more complex
workloads can be run within TEECore.\\

To evaluate the behavior of TEECore further concerning cache associativity, I
chose to modify the test application to allocate memory of a chosen size once at
the first invocation for an array of 8 byte sized integers. In all subsequent
iterations, the test application will increase the counter in the shared memory
region as well as each field in the allocated array. The result will be that the
content of this array will fill L1 data and unified L2 cache resources, leading
to the eviction of other data. On its way, each access of the array will
perform at least one read and one write operation per increment. Each operation
can either result in a miss or hit in the respective caches. As mentioned
before, as soon as events of type \textit{MEM\_LOAD\_RETIRED.L3\_HIT} occur, I
consider the array size too big, as TEECore cannot then uphold its security
guarantees. I measure scenario one for a realistic evaluation of the prototype.
This means that all TEECore codes are measured. Because the L2 cache is a
unified cache, the result of this test reflects the maximum size for code and
data for the whole runtime combined with the payload. Table~\ref{50:tab:size}
shows the measurement result for this test.

Placeholder\todo{Table entries for 48KiB missing}
\begin{table}[ht]
  \centering
  \begin{tabular}{ |l||c|c|c|c| }
    \hline
    Array Size & L1I\_MISS & L1D\_REPLACEMENT & L2\_HIT & L2\_MISS \\
    \hline
    16 KiB     & 1a        & 34                & 26       & 0        \\
    32 KiB     & 1c        & 45               & 34       & 0        \\
    48 KiB     & tba        & tba             & tba    & 0        \\
    64 KiB     & 1a        & 423            & 3e0    & 0        \\
    128 KiB     & 19        & 825            & 7d4    & 0        \\
    256 KiB    & 1a        & 1025            & fba   & 0        \\
    512 KiB    & 23        & 2025            & 1940   & 63b       \\
    1024 KiB   & 17        & 4025            & 13b7   & 202b       \\
    1536 KiB   & 16        & 6024           & 2714  & 37ac    \\
    1832 KiB   & 19        & 7565           & 2b7a  & 4833    \\
    \hline
  \end{tabular}
  \caption{Measurement Result of Scenario 1 with increasing Memory Consumption}
  \label{50:tab:size}
\end{table}

As can be seen in the first row of table~\ref{50:tab:size}, the number of
L1I\_MISSES is comparable in all runs of the benchmark. This is not surprising,
as the instructions executed for each benchmark are the same. The only
difference is the amount of them being executed by the processor, as with an
increasing size of the array, more read and write operations are performed. This
difference does not affect the instruction cache. As a side note, the
measurement results might be subject to over or undercounting, as I expected
them to be identical in all runs. \\
When it comes to the data cache events, the counter results in an array size of
16 KiB are similar to those of SC1 in table~\ref{50:tab:ping_base} and thus to
that of the base line. 16 KiB of memory, therefore, is without concern and can
fit the L1 cache. When increasing the array size to 32 KiB, the count of L1D
replacement increases but does not double. The cause could be that the cache
still contains some of the memory from the initialization process. Because since
then, no larger data structures were loaded by the application, these lines were
not evicted until I ran the benchmark. After all, I conclude from the
measurement values that most of the memory access is done with an array size of
32 KiB hit the L1 data cache. It is only when allocating a 48 KiB-sized array
that the number of L1D\_Replacements and L2\_HITS increase significantly.
Because the cache is not only filled with data from the array but also from data
necessary for program execution, cache lines are evicted from the L1D cache. As
a consequence, the CPU hits the L2 cache when accessing lines evicted from L1
cache. The number of occurrences counted for this event increased rapidly. In a
similar fashion, L1 replacements increase as cache lines loaded from the L2
cache replace conflicting lines from the L1 cache. As these accesses all use the
private cache of the isolated core, none of them are alarming.\\
This changes with the occurrence of L2 cache misses, as they result in access to
shared resources. The first of these misses occur as early as after allocating
an array size of 512 KiB, yet this is only a fourth of the size of the L2 cache.
The cause for these might be conflict misses. Conflict misses occur when all
available cache lines in a set are filled, and another line targeting the same set
is loaded to the cache. Because the cache can store the incoming line only in a
free line of a fitting set, a cache line is evicted. For the cache parameters
listed in table~\ref{50:tab:cache_size}, conflict misses can theoretically occur
for data distributed in 256 KiB step sizes in memory. With increasing array
sizes, the probability of accessing data with this distribution increases. The
results therefore indicate that TEECore can only safely use 256 KiB of data.
\todo{Das ist wahrscheinlich eher ein Bug, weil hier keine Confict misses auftreten sollten bei einer 2 MiB aligned addresse}

\section{Comparison to other TEE Solutions}
\label{eval:compare}
In this section, I compare TEECore with other \gls{tee} solutions mentioned in
chapter~\ref{chap:related} regarding different properties, such as portability,
protection goals, and actual security offered.

\subsection{Portability}
\label{eval:compare:portability}
One goal of TEECore is to implement a \gls{tee} solution that programmers can
use without concerning the underlying hardware. In the end, this goal must be
evaluated from two points of view. While TEECore can serve as an abstraction
layer between hardware and software that wants to use \gls{tee} functionality,
some components of TEECore remain hardware-dependent. This dependency comes from
the fact that \gls{pmc} facilities are highly dependent on the microarchitecture
implemented by a CPU. This means that even CPUs of the same vendor that
implement the same \gls{isa} and are part of the same product line can differ in
their \gls{pmc} facility implementation. An example of such processors are
Intel's most recent hybrid architectures that even employ CPU cores of two
different microarchitectures on the same interposer. An example of such a
processor is the Intel Core i7 14700k I used to evaluate TEECore. For a more
detailed comparison of both microarchitectures, refer to
section~\ref{sec:evaluation}. While I targeted P-Cores for my prototype
implementation, full support for this processor would also mean implementing
TEECore support for E-Cores. This means that supporting many CPUs of different
vendors and architectures would mean additional implementation efforts in
TEECore. On the other hand, \gls{isa} extensions such as \gls{sgx} or \gls{sev}
also come in different versions that support different features that might need
adaption from software. Intel \gls{sgx}, for example, can support different
memory sizes. Furthermore, without an abstraction layer like TEECore, an
application must adapt to the different vendor-specific \gls{isa} extensions
anyway. To conclude, the portability advantage is mainly on the application
side, for applications receive a consistent interface to support processors for
which TEECore is implemented. On the other hand, with the usage of \glspl{pmc},
TEECore uses a mechanism to detect side-channel attacks present in most modern
processors. To adopt TEECore's detection primitives to other processors boils
down to finding the right events and programming the \glspl{pmc} the right way.
Except for \gls{isa} specific extensions, the only hardware requirements of
TEECore are the presence of core exclusive resources such as a private cache and
\gls{pmc} facilities that can signal to TEECore that data spilled from private
resources to shared ones. Such facilities are present not only in x86 processors
but also in ARM as an optional but recommended extension. Compared to Enma,
TEECore is independent of other \gls{os}, as TEECore runs as a runs as
an independent instance next to the \gls{os}. This allows other \gls{os}es to
use TEECore, while Enma is restricted to L4Re.

\subsection{Security}
\label{eval:compare:security}
TEECore can defend against most attacks described in
chapter~\ref{sec:30:tee_attacker_model}. In contrast to other \gls{tee}
implementations, TEECore can detect and react to cache-based side-channel
attacks. This class of attacks is explicitly stated by Intel \gls{sgx}, AMD
\gls{sev}, and Enma not to be part of the attacker model, as these solutions can
neither detect nor defend against these attacks. On the other hand, TEECore
lacks mechanisms to defend against two attack vectors on x86 processors that
\gls{sgx} and \gls{sev} can defend against. First, TEECore depends on the
platform firmware not to be malicious. This dependency comes from the fact that
parts of the firmware have to be trusted by TEECore as it serves as a root of
trust for measurement for Secure Boot. As described in
chapter~\ref{sec:30:tee_boot_chain}, TEECore's \gls{tee} functionality is
dependent on Secure Boot. Aside from Secure Boot, the firmware also controls the
\gls{smm}. With \gls{smi} possibly transferring control at any point to
firmware, TEECore is without any chance against the firmware. All other x86
solutions can work around the firmware, as they implement protection mechanisms
that forbid direct control transfer to \gls{smm} from within their \glspl{tee}.
Additionally, they implement memory encryption so that once the processor
changes from \gls{tee} mode to the \gls{smm}, firmware is unable to decrypt the
memory in use. Memory encryption does not help TEECore defend against malicious
\gls{smm} because \gls{smm} could even read the content of the isolated core's
registers as soon as it manages to interrupt TEECore at the right moment.
Furthermore, other solutions require memory encryption to protect against
memory-tapping attacks. TEECore itself is immune to such attacks as its memory
contents never leave the core and are not sent over the memory bus. If this
happens, TEECore regards this as an attack. Nevertheless, the shared memory
communication path can be tapped. It is therefore the responsibility of the
secure application and the normal partition's applications to secure this
communication channel. Furthermore, other \gls{tee} solutions are not vulnerable
against IPI attacks described in section~\ref{eval:sec}, as they implement
mechanisms to gracefully exit a \gls{tee} before handling any interrupt. Memory
encryption could also help TEECore defend against this kind of attack, but it
would require encrypting values as soon as they are moved from registers to the
cache. To conclude, TEECore solves other problems than \gls{sgx}, \gls{tdx}, and
\gls{sev}. \\

As a result, the idea of combining different approaches might sound appealing as
TEECore allows the detection of side-channel attacks, while other solutions are
not vulnerable to \gls{ipi}-based attacks. In practice, for x86 solutions, this
idea might be complicated to achieve, and additional evaluation would be
required. Intel \gls{sgx}, for example, does not allow measuring events during
enclave execution to prevent them from being profiled. Nevertheless, one
possibility would be for TEECore to host secure applications in enclaves and
implement the remote attestation feature in an additional enclave. Because
enclave memory is encrypted, this could solve the problem of INIT \glspl{ipi}.
Conversely, \gls{sgx} could benefit from the additional isolation TEECore
provides to prevent interrupt-based side channels and make cache-side
channel-based attacks visible. Similarly, TEECore could provide a means for
detecting cache-based side channels to \gls{tdx} and \gls{sev}. For confidential
\glspl{vm} TEECore could implement hypervisor capabilities and execute secure
applications as \glspl{vm}. Again, memory would be encrypted when receiving an
interrupt and returning from \gls{vm} execution. \\

Regarding the \gls{tcb}, TEECore is most similar to Enma. The \gls{tcb} of both
solutions include the hardware, firmware, and bootloader and the \gls{tpm}. In
contrast to Enma, TEECore's \gls{tcb} does not include the \gls{os}.
Nevertheless, TEECore requires all other privileged software to be benign, as
the can mount denial of service attacks against TEECore. Compared to hardware
solutions, only the \gls{tcb} of ARM TrustZone includes the bootloader too.
Other x86 solutions do not require the bootloader to be part of their \gls{tcb}.
Because x86 processors are more like SoCs and do not implement their \gls{tee}
functionality in hardware but software running on dedicated resources on the
processor, e.g., as code running on a dedicated security processor, it is hard
to argue that their \gls{tcb} does not contain any firmware. At this point, one
has to differ between the processor and the platform firmware. A user must trust
at least the processor manufacturer and the software installed on these SoCs.
This allows x86 hardware extensions to exclude platform software that does not
originate from the processor vendor and is excluded from firmware. To summarize,
the \gls{tcb} of TEECore is larger than those of the hardware solutions but
comparable to the \gls{tcb} of Enma.

%%% Local Variables:
%%% TeX-master: "diplom"
%%% End:
