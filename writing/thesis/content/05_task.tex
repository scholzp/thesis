\begin{center}
  \Large
  Task Description
\end{center}

Trusted Execution Environments (TEE) enable tenants to let their software run on
computer systems that are not fully under their control. They ensure that
privileged parties on the target system can neither interfere with the software
nor learn its secrets. Content providers employ TEEs to implement Digital Rights
Management (DRM). In cloud environments TEEs play a role when it comes to build
trust to remote services. In both use cases, the tenant does not trust the owner
of the computer system and/or privileged software. A hardware root of trust is
needed to verify the workload's integrity and and the TEE's identity to the
tenant.\\

Different vendor-specific hardware approaches, such as Intel SGX, AMD SEV, and
Arm TrustZone exist to this day that fulfill the requirements of integrity and
confidentiality to different degrees. These technologies also bring some
downsides. For instance, software needs to be adopted for each vendor to make
use of the different TEE's implementations. Second, major security flaws in the
respective implementations have been found in the past. A lot of the
vulnerabilities of TEE implementations come from the fact that they do not
consider side-channel attacks in their attacker model.\\

A possible solution is creating a trusted execution environment that only uses
core local resources. For example, side-channel attacks from sibling CPUs
utilizing the memory subsystem or buses become impossible when using only CPU
local and exclusive caches in the TEE. For example, the environment can monitor
its confidentiality by leveraging performance counters to detect snooping from
other CPUs or, if a cache line is displaced to a higher level, potentially
shared part, of the memory system. Such an approach enables the creation of a
Trusted Execution Environment on commodity x86 hardware without the need for
dedicated hardware support.\\

In this task, a proof of concept (PoC) implementation of such a Trusted
Execution Environment, which only uses CPU-exclusive resources, is to be
created. The software will run next to a mainstream OS, such as Linux. The PoC
will monitor its confidentiality by using CPU performance counters. In case of
an information leak, the implementation will take appropriate measures. An
appropriate way to exchange information with the OS running next to the trusted
execution environment has to be thought out and implemented. Because
CPU-exclusive resources are scarce, the PoC will be evaluated regarding code
size and what payload the TEE could potentially execute. Furthermore, the TEE
can monitor the system state at run time using performance counters. This
enables the TEE to detect modification and information breaches at run time.
Allowing third parties to safely identify and obtain information on the state of
the TEE requires a form of remote attestation. Remote attestation algorithms
will be evaluated regarding their suitability to attest the state of the TEE for
third parties. A suitable hardware root of trust is to be identified, or if none
exists, a set of minimal required additional hardware will be identified.\\
