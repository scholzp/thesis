% This file is alphabetically sorted by key (which happens to be the acronym)

%
% C ----------------------------------------------------------------------------------------------------------------
\newglossaryentry{g_cpuid}
{
    name=CPUID,
    description={Instruction in x86 CPUs used to query processor features}
}
%
% E ----------------------------------------------------------------------------------------------------------------
\newglossaryentry{g_eoi}
{
    name=end of interrupt,
    description={Signal send to the interrupt controller by the CPU to signal the of an interrupt serving routine}
}
%
% G ----------------------------------------------------------------------------------------------------------------
\newglossaryentry{g_gdt}
{
    name=global descriptor table,
    description={Data structure of x86 and x86\_64 \acrshort{a_isa}s pointed to by global descriptor table register.
            Holds information about memory segments.}
}
%
% I ----------------------------------------------------------------------------------------------------------------
\newglossaryentry{g_idt}
{
    name=interrupt descriptor table,
    description={Data structure that holds a table of interrupt handlers. The interrupt vector is used by the CPU to
            call the respective interrupt handler routine.}
}
\newglossaryentry{g_ipi}
{
    name=inter-processor interrupt,
    description={Interrupt send from one processor in a multiprocessor system to another}
}
\newglossaryentry{g_isa}
{
    name=instruction set architecture,
    description={Architecture of a processor as observable from the outside. Serves as the interface between hard- and
            software and thus describes the instructions supported to program a processor}
}
%
% L ----------------------------------------------------------------------------------------------------------------
\newglossaryentry{g_lapic}
{
    name=local advanced programmable interrupt controller,
    description={The advanced programmable interrupt controller (APIC) is the successor of the legacy programmable
            interrupt controller (PIC) in x86 and x86\_64 systems. In systems with multiple CPU cores the LAPIC is the
            APIC local to the respective core}
}
\newglossaryentry{g_llc}
{
    name=last level cache,
    description={Last level of the cache hierarchy. E.g in a three-level hierarchy, the level 3 (L3) cache is the LLC}
}
\newglossaryentry{g_lmode}
{
    name=long mode,
    description={64 bit operation mode of x86\_64 introduced by AMD with the name AMD64}
}
%
% M ----------------------------------------------------------------------------------------------------------------
\newglossaryentry{g_mmu}
{
    name=memory management unit,
    description={Coprocessor that translates virtual memory addresses to physical memory addresses}
}
\newglossaryentry{g_msr}
{
    name=model specific register,
    description={Non-architectural register which's implementation is specific to the respective processor}
}
%
% N ----------------------------------------------------------------------------------------------------------------
\newglossaryentry{g_nmi}
{
    name=non-maskable interrupt,
    description={Interrupt that can not be masked. That means, delivering of non-maskable interrupts can not be
            deactivated}
}
%
% P ----------------------------------------------------------------------------------------------------------------
\newglossaryentry{g_pf}
{
    name=page fault,
    description={Fault resulting from erroneous memory access. This can be for example a missing page table entry or
            rights violation }
}
\newglossaryentry{g_pmc}
{
    name=performance monitoring counter,
    description={\acrshort{a_msr} to count the occurrence of a specific event in a processor}
}
\newglossaryentry{g_pmi}
{
    name=performance monitoring interrupt,
    description={Interrupt issued upon reaching a threshold of specific event occurrences}
}
\newglossaryentry{g_pmode}
{
    name=protected mode,
    description={Operation mode of x86 and x86\_64 processors. Makes features such as memory protection, virtual memory.
            Introduced with Intel 80286. 32-bit on Intel 80386 and later}
}
%
% R ----------------------------------------------------------------------------------------------------------------
\newglossaryentry{g_rmode}
{
    name=real mode,
    description={Operation mode of x86 and x86\_64 processors. Compatibility mode to the 16-bit Intel 8086, allowing to
            address roughly 1 MiB of memory without protection}
}
%
% S ----------------------------------------------------------------------------------------------------------------
\newglossaryentry{g_sev}
{
    name=SEV-SNP,
    description={Hardware extension in supported AMD processors able to create isolated execution environments for
            virtual machines and prove integrity and identity of the running VM through remote attestation}
}
\newglossaryentry{g_sgx}
{
    name=Intel SGX,
    description={Hardware extension in supported Intel processors able to create isolated execution environments and
            prove integrity and identity of the running program through remote attestation}
}
\newglossaryentry{g_simd}
{
    name={single instruction multiple data},
    description={Form of parallel processing in which multiple processor elements perform the same instruction on
            multiple input data}
}
\newglossaryentry{g_smi}
{
    name=system management mode interrupt,
    description={Interrupt in the x86 and x86\_64 \acrshort{a_isa}. Results in changing to \acrshort{a_smm}}
}
\newglossaryentry{g_smm}
{
    name=system management mode,
    description={High privileged operation mode of x86 and x86\_64 processors. Intended to execute firmware tasks, such
            as power management, transparently to system software. Can only be entered through \acrshort{smi}}
}
\newglossaryentry{g_smram}
{
    name=system management mode RAM,
    description={Hardware protected area of main memory to store \acrshort{a_smm} data and code, such as the
            \acrshort{a_smi} handler}
}
%
% T ----------------------------------------------------------------------------------------------------------------
\newglossaryentry{g_tee}
{
    name=trusted execution environment,
    description={An execution environment that is eligible to run applications isolated and protected from the rest of
            the system. Offers the ability to prove these features through remote attestation}
}
\newglossaryentry{g_tlb}
{
    name=translation lookaside buffer,
    description={Cache structure that stores translations from virtual to physical addresses}
}
