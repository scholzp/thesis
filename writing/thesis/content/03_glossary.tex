% This file is alphabetically sorted by key (which happens to be the acronym)

%
% A ----------------------------------------------------------------------------------------------------------------
\newglossaryentry{g_aik}
{
  name=Attestation Identity Key (AIK),
  description={Pseudo identity key of a TPM created form it's EK}
}
\newglossaryentry{g_aikca}
{
  name=Attestation Identity Key Certification Authority (AIKCA),
  description={Trusted third party that issues certificates for AIKs of a TPM in TEECore}
}
%
% C ----------------------------------------------------------------------------------------------------------------
\newglossaryentry{g_ca}
{
  name=Certification Authority (CA),
  description={Trusted third party that issues certificates in a cryptographic system}
}
\newglossaryentry{g_cpuid}
{
  name=CPUID,
  description={Instruction in x86 CPUs used to query processor features}
}
%
% E ----------------------------------------------------------------------------------------------------------------
\newglossaryentry{g_ek}
{
  name=Endorsement Key (EK),
  description={Key burned to a security device by the manufacturer by which the security device can be identified}
}
\newglossaryentry{g_elf}
{
  name=Executable and Linking Format (ELF),
  description={Binary format of executeable and linkable files in many UNIX systems}
}
\newglossaryentry{g_eoi}
{
  name=End of Interrupt (EOI),
  description={Signal send to the interrupt controller by the CPU to signal the of an interrupt serving routine}
}
%
% G ----------------------------------------------------------------------------------------------------------------
\newglossaryentry{g_gdt}
{
  name=Global Descriptor Table (GDT),
  description={Data structure of x86 and x86\_64 ISAs pointed to by global descriptor table register.
  Holds information about memory segments}
}
%
% I ----------------------------------------------------------------------------------------------------------------
\newglossaryentry{g_ica}
{
  name=Intermediate Certification Authority (ICA),
  description={Certification authority that is not the root certification authority}
}
\newglossaryentry{g_idt}
{
  name=Interrupt Descriptor Table (IDT),
  description={x86 data structure that holds a table of interrupt handlers. The interrupt vector is used by the CPU to
  call the respective interrupt handler routine.}
}
\newglossaryentry{g_idtr}
{
  name=Interrupt Descriptor Table Register (IDTR),
  description={Register in x86 processors that holds the address of the IDT}
}
\newglossaryentry{g_ipi}
{
  name=Inter-processor Interrupt (IPI),
  description={Interrupt send from one processor in a multiprocessor system to another}
}
\newglossaryentry{g_isa}
{
  name=Instruction Set Architecture (ISA),
  description={Architecture of a processor as observable from the outside. Serves as the interface between hard- and
  software and thus describes the instructions supported to program a processor}
}
%
% L ----------------------------------------------------------------------------------------------------------------
\newglossaryentry{g_lapic}
{
  name=Local Advanced Programmable Interrupt Controller (LAPIC),
  description={The advanced programmable interrupt controller (APIC) is the successor of the legacy programmable
    interrupt controller (PIC) in x86 and x86\_64 systems. In systems with multiple CPU cores the LAPIC is the
  APIC local to the respective core}
}
\newglossaryentry{g_llc}
{
  name=Last Level Cache (LLC),
  description={Last level of the cache hierarchy. E.g in a three-level hierarchy, the level 3 (L3) cache is the LLC}
}
\newglossaryentry{g_lmode}
{
  name=Long Mode,
  description={64 bit operation mode of x86\_64 introduced by AMD with the name AMD64}
}
\newglossaryentry{g_lvt}
{
  name=Local Vector Table (LVT),
  description={Register in the LAPIC used to configure how local interrupt are delivered to the processor core}
}
%
% M ----------------------------------------------------------------------------------------------------------------
\newglossaryentry{g_mbi}
{
  name=Multiboot2 Information (MBI),
  description={Data structure in the Multiboot2 standard used to by the bootloader to communicate vital information to
  the operating system.}
}
\newglossaryentry{g_mmu}
{
  name=Memory Management Unit (MMU),
  description={Coprocessor that translates virtual memory addresses to physical memory addresses}
}
\newglossaryentry{g_msr}
{
  name=Model Specific Register (MSR),
  description={Non-architectural register which implementation is specific to the respective processor}
}
%
% N ----------------------------------------------------------------------------------------------------------------
\newglossaryentry{g_nmi}
{
  name=Non-maskable Interrupt (NMI),
  description={Interrupt that cannot be masked. That means, delivering of non-maskable interrupts cannot be
  deactivated}
}
%
% O ----------------------------------------------------------------------------------------------------------------
\newglossaryentry{g_os}
{
  name=Operating System (OS),
  description={Privileged system software that manages resources and allows execution of user space applications}
}
%
% P ----------------------------------------------------------------------------------------------------------------
\newglossaryentry{g_pcr}
{
  name=Platform Configuration Register (PCR),
  description={Register of a TPM available to (system) software to store the results of a measurement in shielded
  memory}
}
\newglossaryentry{g_pf}
{
  name=Page Fault (\#PF),
  description={Fault resulting from erroneous memory access. This can be for example a missing page table entry or
  rights violation}
}
\newglossaryentry{g_pki}
{
  name=Public Key Infrastructure (PKI),
  description={Infrastructure to manage public encryption keys and bind them to identities}
}
\newglossaryentry{g_pmc}
{
  name=Performance Monitoring Counter (PMC),
  description={MSR that counts the occurrence of a specific event in a processor}
}
\newglossaryentry{g_pmi}
{
  name=Performance Monitoring Interrupt (PMI),
  description={Interrupt issued upon reaching a threshold of specific event occurrences}
}
\newglossaryentry{g_pmode}
{
  name=Protected Mode,
  description={Operation mode of x86 and x86\_64 processors. Makes features such as memory protection, virtual memory.
  Introduced with Intel 80286. 32-bit on Intel 80386 and later}
}
%
% R ----------------------------------------------------------------------------------------------------------------
\newglossaryentry{g_rmode}
{
  name=Real Mode,
  description={Operation mode of x86 and x86\_64 processors. Compatibility mode to the 16-bit Intel 8086, allowing to
  address roughly 1 MiB of memory without protection}
}
%
% S ----------------------------------------------------------------------------------------------------------------
\newglossaryentry{g_see}
{
  name=Secure Execution Environment (SEE),
  description={Execution environment that allows for tamper-resistant program execution}
}
\newglossaryentry{g_sev}
{
  name=SEV-SNP,
  description={Hardware extension in supported AMD processors able to create isolated execution environments for
  virtual machines and prove integrity and identity of the running VM through remote attestation}
}
\newglossaryentry{g_sgx}
{
  name=Intel Software Guard Extensions (SGX),
  description={Hardware extension in supported Intel processors able to create isolated execution environments and
  prove integrity and identity of the running program through remote attestation}
}
\newglossaryentry{g_simd}
{
  name=Single Instruction Multiple Data (SIMD),
  description={Form of parallel processing in which multiple processor elements perform the same instruction on
  multiple input data}
}
\newglossaryentry{g_smi}
{
  name=System Management Interrupt (SMI),
  description={Interrupt in the x86 and x86\_64 ISA. Results in changing to SMM}
}
\newglossaryentry{g_smm}
{
  name=System Management Mode (SMM),
  description={High privileged operation mode of x86 and x86\_64 processors. Intended to execute firmware tasks, such
  as power management, transparently to system software. Can only be entered through SMI}
}
\newglossaryentry{g_smram}
{
  name=System Management RAM (SMRAM),
  description={Hardware protected area of main memory to store SMM data and code, such as the
  SMI handler}
}
\newglossaryentry{g_soc}
{
  name=System on Chip (SoC),
  description={Integrated circuit that houses components that all realize different functions}
}
%
% T ----------------------------------------------------------------------------------------------------------------
\newglossaryentry{g_tcb}
{
  name=Trusted Computing Base (TCB),
  description={A small amount of software and hardware that security depends on}
}
\newglossaryentry{g_tcg}
{
  name=Trusted Computing Group (TCG),
  description={Consortium that manages and developes open standards for trusted computing platforms}
}
\newglossaryentry{g_tdx}
{
  name=Intel Trust Domain Extensions (TDX),
  description={Intel specific extension to the x86 ISA that allows creation of confidential virtual machines}
}
\newglossaryentry{g_tee}
{
  name=Trusted Execution Environment (TEE),
  description={An execution environment that is eligible to run applications isolated and protected from the rest of
  the system. Offers the ability to prove these features through remote attestation}
}
\newglossaryentry{g_tlb}
{
  name=Translation Lookaside Buffer (TLB),
  description={Cache structure that stores translations from virtual to physical addresses}
}
\newglossaryentry{g_tpm}
{
  name=Trusted Platform Module (TPM),
  description={Isolated cryptographic coprocessor that is able to measure the state of a platform and attest to its measurements}
}
%
% V ----------------------------------------------------------------------------------------------------------------
\newglossaryentry{g_vm}
{
  name=Virtual Machine (VM),
  description={Software encapsulated computing system}
}
