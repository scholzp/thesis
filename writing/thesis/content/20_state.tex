\chapter{Background}
\label{sec:state}

% Hier werden zwei wesentliche Aufgaben erledigt:

% 1. Der Leser muß alles beigebracht bekommen, was er zum Verständnis
% der späteren Kapitel braucht. Insbesondere sind in unserem Fach die
% Systemvoraussetzungen zu klären, die man später benutzt. Zulässig ist
% auch, daß man hier auf Tutorials oder Ähnliches verweist, die hier auf
% dem Netz zugänglich sind.

% 2. Es muß klar werden, was anderswo zu diesem Problem gearbeitet
% wird. Insbesondere sollen natürlich die Lücken der anderen klar
% werden. Warum ist die eigene Arbeit, der eigene Ansatz wichtig, um
% hier den Stand der Technik weiterzubringen? Dieses Kapitel wird von
% vielen Lesern übergangen (nicht aber vom Gutachter ;-), auch später
% bei Veröffentlichungen ist "Related Work" eine wichtige Sache.

% Viele Leser stellen dann später fest, daß sie einige der Grundlagen
% doch brauchen und blättern zurück. Deshalb ist es gut,
% Rückwärtsverweise in späteren Kapiteln zu haben, und zwar so, daß man
% die Abschnitte, auf die verwiesen wird, auch für sich lesen
% kann. Diese Kapitel kann relativ lang werden, je größer der Kontext
% der Arbeit, desto länger. Es lohnt sich auch! Den Text kann man unter
% Umständen wiederverwenden, indem man ihn als "Tutorial" zu einem
% Gebiet auch dem Netz zugänglich macht.

% Dadurch gewinnt man manchmal wertvolle Hinweise von Kollegen. Dieses
% Kapitel wird in der Regel zuerst geschrieben und ist das Einfachste
% (oder das Schwerste weil erste).

In this chapter, I will provide background information on the topics revolving
around creating a standalone \gls{tee} that uses primitives of the processor
core to isolate itself. For this, I first introduce basics about how x86
processor programming works in section~\ref{sec:state:technical}. Then we
revisit \glspl{tee} in general and how important building blocks look (c.f.
sections~\ref{sec:20:building_blocks} and~\ref{sec:state:tee}). We continue in
section~\ref{sec:20:attacks} to learn about attacks and in
section~\ref{sec:20:mitigations} before I finally summarize this chapter in
section~\ref{sec:20:summary} to highlight approaches best to my knowledge, not
employed in related work so far.

\section{Technical Background}
\label{sec:state:technical}
This chapter will give us an overview of important mechanisms used by modern
CPUs. While there are other widely used architectures, such as ARM on
mobile devices, we will explore these features in the example of the x86\_64
architecture. \\

We decided to do so because our proof of concept implementation will target
the x86\_64 architecture. Moreover, behavior and implementations of similar
concepts can differ between \gls{isa} (\acrshort{a_isa}) and even
implementations of the same \acrshort{a_isa} on the microarchitectural plane.
We will, therefore, note differences in implementations of x86\_64 whenever
appropriate. Until then, we will stick to the general behavior described in
the AMD64 specification so that the Intel 64-bit extensions are largely
compatible.

\subsection{Operation Modes}
\label{sec:state:technical:modes}
The x86 architecture is rooted in the Intel 8086, designed in 1978. As
a 16-bit microprocessor design, the Intel 8086 physically can only address 64
KiB of memory. To overcome this limitation, Intel introduced a
segmentation-based addressing mode that, together with the 20-bit address bus,
allowed the CPU to address slightly more than 1 MiB of main memory. Intel 8086
missed features such as memory protection, code privileges, and multitasking.
These features were important to allow advanced operating systems. These
features extended the instruction set and microarchitecture
implemented by the Intel 80286. The Intel 80286 could address, for example,
16 MiB of physical memory through 24 address bus bits and could address 1 GiB of
virtual memory. Moreover, segment registers now contain an index in a
data structure called \gls{gdt}, short \acrshort{a_gdt}.
The \acrshort{a_gdt} contains descriptors that store information on segments,
such as base address and access rights. The Intel 80286 implements access rights
as so-called rings, and the lower the numerical access level, the higher the
privileges of the executing software are. The introduction of the
\acrshort{a_gdt} allowed virtual memory implementation with memory protection.
A new \gls{mmu}, short \acrshort{a_mmu}, added for the first time on chip in
the 80286 allowed fast virtual to physical address translation. \\

When using the features newly added to the 80286
the processor was incompatible with the original Intel 8086. Because the result
would have been that all software already existing for the Intel 8086 would have
been to at least be recompiled to run on the Intel 80286, Intel designed the new
features as an extension to the older 8086. Deactivating these new features
allowed the Intel 80286 to run programs written for the 8086. For this, Intel
introduced two operation modes in the 80286. The operation mode offering
compatibility was called \textbf{\gls{rmode}} because the address
calculation in the 8086 always yielded real physical addresses. Because the new
80286 offered memory protection, the mode was called \textbf{\gls{pmode}}. To
maintain compatibility with the Intel 8086, the 80286 first boots in
\gls{rmode}, as all successors of the 80286 do. On startup, the code for
entering other operation modes must, therefore, reside in the first MiB of
memory. \\

The successor of the Intel 80286, the Intel 80386 extended the architecture to
32-bit and introduced paging to the architecture, allowing paged virtual memory
(cf. \ref{sec:state:technical:paging}). The registers of the 80386 were also
extended to 32-bit. To differentiate between the registers and their extensions,
they are prefixed with ''E''. For example, the instruction pointer register,
formerly identified by IP, becomes EIP with. The Intel 80386 also added the
\textbf{\gls{smm}} or short \acrshort{a_smm}. The \acrshort{a_smm} is intended
for firmware to execute special tasks such as energy management or other tasks
executed by the firmware. The \acrshort{a_smm} code resides in a specially
protected area of the main memory, referred to as \gls{smram}. \acrshort{a_smm}
executes with the highest privileges in x86, which is ring -2. With these
changes, the Intel 80386 formed the base for the architecture today known as x86
or IA-32.\\

Successors of the 80386 added support for optional floating point units and
\acrshort{a_simd} extensions to the architecture but left the fundamental
\acrshort{a_isa} unchanged. It was not until the year 2000 that AMD released
its extensions of the x86 architecture to 64 bit that introduced a new processor
operation mode called \textbf{\gls{lmode}} and two sub modes called
\textbf{Compatibility Mode} and \textbf{64-Bit mode}. This extension by AMD to
the x86 \acrshort{a_isa} was first called AMD64 and later largely adopted by
Intel as Intel 64. Today it is also known by the name x86\_64.
Compatibility mode allows the execution of legacy 32-bit and 16-bit code.
64-bit mode extends the \acrshort{a_isa} by 64-bit operands and addresses, and
extends all general purpose registers to 64-bit. Moreover, the 64-bit mode
adds eight additional general-purpose registers and allows instruction pointer
relative addressing. x86 allowed the latter only in control transfer
instructions. AMD64 implements floating-point arithmetic through the mandatory
SSE2 \acrshort{a_simd} extension that was introduced to x86 in 2000 with the Intel
Pentium 4 processor. The number of SSE2 registers is doubled compared to the
number in the original x86 extension. Moreover, to allow differentiating between
x86 and AMD64 registers, AMD added the REX prefix to the specification. As a
result, general purpose registers are prefixed with the letter ''R''. For
example, the instruction pointer registers of x86 (IP) becomes RIP. The number
of page table levels to translate virtual to physical addresses was increased
from two to at least four.\\

Unless otherwise stated, all references we make to \gls{lmode} in
the remaining parts of this thesis mean the combination of both sub-modes and
therefore the whole extension implemented by AMD64.

\subsection{Interrupts and Exceptions}
\label{sec:state:technical:interrupts}
Interrupts and Exceptions are used to call system-specific functions and respond
to special conditions in the CPU or system. Exceptions occur while executing
software and result from program or CPU internal errors. Interrupts, on the
other hand, are either the result of software interrupt instructions or the
interaction of the CPU with other system components, such as keyboard input.
Exceptions can be divided into three types:
\begin{enumerate}
    \item Faults: Result of an error with the instruction to execute
    \item Traps: Result of breakpoint and software interrupt instructions
    \item Aborts
\end{enumerate}
Interrupts, on the other hand, can be divided into two classes:
\begin{enumerate}
    \item Maskable Interrupts: Can be masked. Masking an interrupt means that
          software can temporarily disable them. The interrupt controller
          holds them back until interrupts are enabled again.
    \item \Gls{nmi}~(\acrshort{a_nmi}): Software cannot
          turn off \acrshort{a_nmi}s. The interrupt controller delivers~
          \acrshort{a_nmi}s to the CPU unless it currently serves another~
          \acrshort{a_nmi}. When the CPU executes the IRET instruction in the
          interrupt handler, the interrupt controller can deliver the next~
          \acrshort{a_nmi}.
\end{enumerate}
The CPU serves exceptions and interrupts by calling the respective interrupt or
exception handler routine. System software can install an interrupt handler by
writing its address into the \gls{idt}, short \acrshort{a_idt}. The
\acrshort{a_idt} is a data structure containing pointers to interrupt handling
routines. The specification assigns a so-called vector to each interrupt or
exception. With the help of this vector, the CPU determines what handler to call
by using the interrupt vector as an index in the \acrshort{a_idt}. It finds the
\acrshort{a_idt} by reading the value of the interrupt descriptor table register
(IDTR). System software is expected to create the \acrshort{a_idt} and install
handlers for at least the predefined vectors. Then, the address of the
\acrshort{a_idt} will be provided by writing to the IDTR before interrupts can
be activated. \\

Over time, interrupt controllers were improved and adapted to new use cases
similar to CPUs. In \gls{rmode}, the CPU falls back to using the Intel 8259 or a
compatible programmable interrupt controller, short PIC, that delivers
interrupts to the CPU. Once the PIC delivers an interrupt to the CPU, the CPUs
must serve it and signal the PIC with an \gls{eoi}, short \acrshort{a_eoi}, that
the serving routine is done, before it can receive the next interrupt. \\

\begin{figure}
    \begin{center}
        \includegraphics[width=.6\textwidth]{images/lapic_placeholder.png}
        \caption{Illustration of how \acrshort{a_lapic}s integrate in a multiprocessor system}
        \label{fig:state:technical:lapic}
    \end{center}
\end{figure}
\todo{This is a placeholder}

Intel later introduced the advanced programmable interrupt controller, short
APIC with its 486 processor line to allow the system to operate with multiple
CPUs. In modern CPUs, multiprocessor configurations are prevalent, and each processor
core has its own APIC. Because the APIC is CPU local, these APICs are called
local APICs or \gls{lapic}, short \acrshort{a_lapic}. \acrshort{a_lapic}s forward
interrupts from different sources to the respective CPU core. For example, the
\gls{lapic} receives interrupts such as \gls{ipi} (\acrshort{a_ipi}) from other
\acrshort{a_lapic}s, interrupts from devices connected to the IOAPIC, and legacy
interrupts from the PIC. Figure~\ref{fig:state:technical:lapic} shows a
schematic view on how the LAPIC of each CPU core integrates into the system.
System software must change the CPU to \gls{pmode} to
activate the \acrshort{lapic} system. After changing to \gls{pmode}, system
software must write a valid address to the APIC base address register. All APIC
registers are then mapped to the 4-KiB APIC register space starting at the
address specified in APIC base address register. System software can then access
APIC registers with memory reads and writes to the APIC register space.

\subsection{Caches}
\label{sec:state:technical:caches}
Since 1980, the performance growth of memory and processors has diverged
steadily with an ever-growing gap. Hennessy et al. note a difference in
performance growth of factor 1,000 for a single CPU core and memory
technologies. To highlight the problem, they show the difference between the
peak bandwidth offered by memory and the peak memory bandwidth demand of an
Intel i7 processor of the Nehalem architecture employing four CPU cores. This
processor, which Intel introduced in the year 2009, can generate an instruction
stream that peaks in a maximum bandwidth need of 409.6 GB/sec, while its
dual-channel DDR3 memory interface can yield a maximum bandwidth of 25GB per
second.\cite{hennessy2011computer} With processor implementations growing
further in width, the demand for fast memory further grows. For example, from
November 2023 to January 2024, the number of systems in the TOP500 list that
employed CPUs with 96 cores per socket increased from 0 to 3, with the former
maximum number of cores per socket being 72.\cite{top500} To hide latencies and
bridge the gap between CPU demand and actual main memory bandwidth, CPUs today
employ fast local on-chip memory to buffer data they already accessed. If they
reaccess this memory item, they can use these buffers to speed up access. This
on-chip buffer described is called cache.\\

The cache is an integral component organized in a multi-level hierarchy in
modern CPUs. In this hierarchy, the lowest and the nearest to the core level,
called the L1 cache, implements the fastest access. Most modern x86 CPUs divide
their L1 cache into two parts: L1D for data and L1I instructions. With
increasing levels, caches grow in size but tend to be slower. For example, while
the L1 Cache of Nehalem CPUs offers only one cycle of latency, the L3 Cache of
the same CPU has a latency of 35 cycles.\cite{hennessy2011computer}\\

When the CPU tries to access data, it first queries the fastest cache. If the
CPU can locate the data in the cache, this is called a cache hit. On a cache
hit, the CPU can profit from reduced access time and improved bandwidth. The
opposite of a cache hit is called a cache miss, in which the CPU subsequently
queries the next level in the memory hierarchy. If it finds the data needed, it
loads these into the nearest cache for faster access. Caches can only store a
limited number of items, so-called cache lines. Each cache line is either 32 or
64 bytes in size, with x86\_64 processors mainly implement a cache line size of
64 bytes. Cache lines are a copy of the main memory, and the processor uses this
copy for all of its operations unless otherwise configured. The processor loads
data in cache line size granularity from the main memory. Modern x86 CPU designs
organize caches in so-called sets. In a set, all of the cache lines are
exclusively mapped to a specific, non-overlapping address range of physical
memory. The number of cache lines a physical address can be mapped to is
indirectly proportional to the number of sets in a cache, which is called the
associativity of the cache. For example, a cache with one cache line per set is
called a direct mapped cache because a physical address can only be mapped to
one set and its cache line. The hardware cost of a mapped cache is the lowest.
Direct-mapped caches bring the drawback of a higher probability of conflict
misses. A conflict miss in caching describes the situation in which the
processor loads data from the main memory and has to evict lines from the cache
because no suitable cache line is available. Because a physical address range
with multiple cache line-sized chunks is exclusively mapped to one line in a
directly mapped cache, loads from the same range can cause conflict misses.
Conversely, in a cache with a single set containing all cache lines, all
physical addresses can be mapped to each line in the set. Such a cache is called
fully associative and has the highest hardware costs but comes with the
advantage of minimal fragmentation. Often, processors implement a trade-off
between hardware cost and flexibility of the cache. Such caches are called
$n$-way set associative, where $n$ is the number of cache lines in each set. On
a cache fill, the processor first finds the set of cache lines to which it can
map the physical address. Once it identifies the set, the processor can place
the data in any line. \\

Caches use the principle of locality, which consists of two kinds of locality.
The first is called spatial locality, which describes that if a program accesses
data from the main memory, it will access data located near the last loaded data
with a high probability. In this case, the CPU tries to guess the size of the
loaded structure to load parts missing in the cache in advance. If the CPU then
accesses the neighbor of the first data, it already resides in the cache,
lowering the latency. The second principle is temporal locality, which states
that a program soon reuses memory references with a high probability. The CPU
can, therefore, gain performance by storing recently used data in the cache for
reuse.\\

Modern caches are still small compared to the size of the main memory. For
example, CPUs of the recent AMD Zen 5 microarchitecture have an L1
cache size of 92 KiB (data and instruction combined) and 32 MiB of L3 cache
shared between the cores, from which eight physically exist on the same chiplet.
The same CPU supports 128 GiB of main memory, which is seven orders of magnitude
larger than the L1 and 4 orders of magnitude larger than the L3 cache.\todo{cite amd zen five paper (When writing this SLUB Shibboleth was broken...)}
Caches, therefore, can not buffer the content of the whole main memory and need
strategies for their management. For example, CPUs of the Nehalem
microarchitecture use a pseudo least recently used (LRU) strategy to manage
their cache's content.\cite{hennessy2011computer} With this, cache lines are
marked as frequently or less frequently used. The least frequently used lines
are evicted from the cache and replaced by the newly cached data. Evicted
lines are moved one level up in the hierarchy. Other management strategies are
the least frequently used (LFU) or a first-in-first-out (FIFO) policy. Next to
eviction strategies, different strategies for writing data back to main memory
exist:
\begin{itemize}
    \item \textbf{write-back}: Data modified in the cache is stored and
          written back to the main memory later. Until cache and main memory are
          synchronized, the region in main memory is marked as dirty through a
          dedicated status bit. Cache coherency protocols are required to
          allow multiple devices to access the same memory range.
    \item \textbf{write-through}: The Changes in the cache are instantly written
          to the main memory. These writes can slow down program execution
          because of costly main memory writes.
    \item \textbf{cache-disable}: The CPU cache is disabled, and the CPU
          performs all memory operations using the main memory.
\end{itemize}
Legacy x86 control cache settings with configuration bits in the control
register CR0. In modern x86\_64 processors, systems software sets the write-back
strategy on page granularity (c.f.~\ref{sec:state:technical:paging}). X86\_64
processors ignore the write-through setting and use the page-level settings
instead. \cite{amd_manual} The default in x86\_64 processors ist to use a
write-back strategy. For this, processors implement coherency protocols. A
widely used coherency protocol is called MESI. MESI is an acronym of the four
states "modified," "exclusive," "shared," and "invalid." An extension to MESI is
MOESI protocol, which additionally introduces the "owned" state. The states of a
cache line, as specified by the AMD programmers manual, are:
\begin{itemize}
    \item \textbf{Modified}: The copy in the processor's cache is the most
          recent and modified. The copy in the main memory needs to be updated.
          No other processor in the system maintains a copy.
    \item \textbf{Owned}: Similar to shared state, but copy in main memory can
          be stale. All other processors must hold their copy in the shared
          state.
    \item \textbf{Shared}:  The copy is the most recent and correct copy of the
          data. Other processors may hold copies, too. Main memory holds the
          most recent and correct copy, too, if no processor holds a copy with
          \textbf{Owned} state.
    \item \textbf{Exclusive}: The processors and the main memory's copy are the
          most recent and correct copies. No other processor holds a copy.
    \item \textbf{Invalid}: The local copy is invalid. Either main memory or
          other processors hold a valid copy.
\end{itemize}
\todo{inclusiviy}

An interesting difference in the implementations of Intel 64 and AMD64
processors are the cache coherency and inclusion policies. AMD64 specifies MOESI
as a cache coherency protocol to use, as mentioned before.\cite{amd_manual}
Intel 64, on the other hand, specifies MESI as a coherency protocol.
\cite{intel_sdm} The use of different coherency protocols to synchronize data
between multiple CPU cores has a direct effect on the inclusivity of the cache
implementation. On all current x86\_64 multicore CPUs, the \acrfull{a_llc} is
shared among all cores for synchronization and uses one of the coherency
protocols. CPUs produced by Intel largely use an inclusive \acrshort{a_llc}. An
inclusive cache describes a cache containing items in lower cache levels. If an
item is modified in a lower cache level, the changes are automatically
propagated to the higher inclusive cache level. The opposite of an inclusive
cache is an exclusive cache design, as used by most AMD64 CPUS. Exclusive caches
do not necessarily contain items of lower cache levels, and the synchronization
of modified items needs to be propagated in other ways. The additional "owned"
state of the MOESI protocol solves this issue.

\subsection{Hardware Performance counters}
\label{sec:state:technical:hpc}
The first x86 CPU implementing hardware \gls{pmc} \acrshort{a_pmc} and
documenting them was the Intel Pentium Pro in 1995, implementing the P6
microarchitecture.\cite{intel_sdm} The AMD64 \acrshort{a_isa} specificities four
freely programmable architectural hardware \acrshort{a_pmc}s.\cite{amd_manual}
Concrete processor implementation can offer additional counters. Similarly, the
\acrshort{a_isa} specifies architectural events that must be present in every
processor implementing AMD64. Additional events are vendor and
implementation-specific. The four counters can be programmed to count any event
supported by the respective processor implementation. Vendors of x86 CPUs
publish what processor supports what additional events in their manuals.
Moreover, a CPU notes what events it supports in the result of the \gls{CPUID}
instruction. \\

In x86 hardware, Performance counters are implemented by a set of two \gls{msr}s
(\acrshort{a_msr}) per counter. One \acrshort{a_msr} can be programmed by system
software with the event to measure, while a second \acrshort{a_msr} counts the
occurrence of
the respective event. As noted, programming has to be done by system software
with elevated privileges. Reading \acrlong{a_pmc}s can be done
with privileged instruction or from user space with the RDPMC instruction.
In this way, a program can poll the values of counters. The system software
can also program a threshold for a \gls{pmi}, short \acrshort{a_pmi}.
Once the \acrshort{a_pmc} values exceed the threshold, the \acrshort{a_pmi} is
triggered, offering an alternative to expensive polling techniques.\\

When we use hardware \acrshort{a_pmc}, we must use the proper
technique adopted to the environment in which we read the counter values. Das et
al. found in a comprehensive survey that noise from the system is often present,
e.g. context switches influence the values of performance counters.
\cite{das_sok_2019} Moreover, some counter events are over-counted while the CPU
can undercount others.\cite{weaver_non-determinism_2013} It is therefore
important to check the right conditions for using hardware counters and verify
that they work correctly.

\subsection{Paging}
\label{sec:state:technical:paging}
\begin{center}
    \begin{figure}
        \includegraphics[width=\textwidth]{images/paging_placeholder.png}
        \caption{Illustration of Virtual Address Translation with 4 Levels of Page Tables}
        \label{fig:state:technical:paging}
    \end{figure}
    \todo{This is a placeholder}
\end{center}

Historically, paging replaced segmentation in x86 for working with virtual
memory. Virtual memory is an abstraction that does not allow direct addressing
of physical main memory through programs. Instead, a program sees the whole
address space as available. Once the program tries to allocate memory, system
software creates a virtual memory address and hands it to the program. System
software decides through its mapping implementation, through which physical
memory the respective virtual memory is backed. Virtual memory, therefore,
allows system software to decide if a given virtual address is backed by main
memory or to move it to secondary memory, such as disk storage. Moreover, two
or more virtual addresses can be mapped to the same physical memory, allowing
efficient use of shared program libraries. Mapping happens transparently to the
program, which means that the program does not have to worry about what memory
is used to back its virtual memory physically.\\

Paging is an approach to virtual memory that replaces segment-based virtual
memory in modern x86 CPUs. With paging, the system divides memory into chunks
equal to the page size. System software uses so-called pages to manage memory on
allocation requests from programs. Pages are virtual memory chunks mapped to
page frames in physical memory. Access rights are controlled by system software
that sets the respective rights in the pages and is enforced by hardware with
page granularity. Hardware translates virtual to physical addresses. This
process is called page translation. In the following, we will view the paging
implementation of x86\_64 in more detail to explain page translation in more
detail.\\

Hardware uses so-called page tables for page translation that are hierarchically
organized. Each page table forms a level in the table hierarchy and is the size
of one page, storing references (addresses) to the next lower level. The last
page table in this hierarchy stores the address of the physical page frame. By
default, x86\_64 uses pages of size 4 KiB. Because x86\_64 uses 64-bit
addresses, each page table can store 512 entries. Figure
~\ref{fig:state:technical:paging}~shows the page translation of a virtual 64-bit
address in x86\_64 with a page table hierarchy of four levels and a page size of
4 KiB. \\

The virtual address is divided into a 12-bit field used as an index in the
physical page and four 9-bit fields, which the hardware uses as an index to
access the content of the respective page level. The remaining address bits are
sign extensions of bit 52, forming a canonical address. The hardware must
perform a page table walk to resolve the virtual address. For the first step of
a page table walk, the processor locates the address of the first page table, in
this case, the PML4 table. The address of the PML4 is written by system software
to CR3 after it creates the page table. In doing so, the system software does
not write the entire 64-bit address to CR3. Instead, it writes the most
significant bits beginning from bit 12 because memory is accessed at page
granularity, meaning the lower 12 bits are irrelevant to addressing the page
frame. Most x86\_64 implementations furthermore only support $2^{52}$ byte of
addressable memory and therefore use 52 bits of the 64-bit address space. This
results in the 40 bits stored in CR3, as seen in
~Figure~\ref{fig:state:technical:paging}. To access PML4, the CPU reconstructs
its address from CR3. It uses the 9-bit index of the PML4 offset field in the
virtual address to find the Page-Map-Level 4 entry containing the address of the
next page table, the Page-Directory-Pointer table. The CPU accesses this table
and all successor tables similarly until it locates the physical page. To access
the data that belongs to the virtual address, the CPU uses the last 12 bits of
the virtual address to locate the data on the physical page. \\

As the page table walk is expensive, the \acrshort{a_mmu} automatically stores
the translated addresses in a cache called \gls{tlb}, short \acrshort{a_tlb}.
Each CPU manages its page tables and \acrshort{a_tlb}. System software must
maintain consistency between page tables and each \acrshort{a_tlb} by
invalidating individual entries or the entire \acrshort{a_tlb}. If system
software invalidates an entry, the CPU has to complete a page walk for this
virtual address, upon which the MMU updates the \acrshort{a_tlb}. The INVLPGB
can additionally be used to invalidate a range of TLB entries and broadcast
these invalidations to other CPUs. \\

\begin{figure}
    \begin{center}
        \includegraphics[width=.6\textwidth]{images/paging_rights_placeholder.png}
        \caption{Layout of the lowest 12 bits in a page table entry}
        \label{fig:state:technical:paging_rights}
    \end{center}
\end{figure}
\todo{This is a placeholder}

The unused lower 12 bits of each table entry are used for management properties,
such as access rights management. Figure~\ref{fig:state:technical:paging_rights}
shows the layout of the lower 12 bits of a page table entry. The present bit is
another property stored in the lowest 12 if of a page table's entry. It
indicates if the page the respective entry points to was initialized and loaded
to the page table. If this bit equals 0, the CPU generates a \gls{pf}
(\acrshort{a_pf}) exception and expects system software to load the respective
entry. The CPU also generates a \acrshort{a_pf} if software violates access
rights. A second bit important for this thesis is the Page-Level Cache Disable
(PCD) bit. If the PCD bit is set, the CPU cannot cache the respective page. Bit
3 (PWT) controls page-level write-through. If system software sets this bit, the
page table has a write-through caching policy.

\todo{write about bootflow}
\subsection{Bootflow of Modern x86 CPUs}
\label{sec:state:technical:boot}

\section{Chain of Trust}
\label{sec:20:chain_of_trust}
Signatures of real humans differ because humans do not work entirely
deterministic. The process of forging signatures of other humans requires
dictations and practice by the forger because the human factor introduces
entropy to the process of signing. On the contrary, computers are built to work
deterministically. If the same input data is processed the same way, computers
working as intended by humans produce the same output. It is, therefore, that
authentication in the digital realm is implemented through cryptography and the
knowledge of secrets. Communication involves at least two parties, which we want
to call Alice and Bob in the following example.\\

When referring to cryptographic algorithms, a secret is called a key, and based
on the kind of keys used, we can differentiate between two kinds of
cryptographic algorithms. In symmetric algorithms, the same key is used for
encryption and decryption. Alice and Bob share thus the same secret. In
asymmetric algorithms, each of Alice and Bob maintain their secrets. These keys
have a public and a private part, respectively. Alice uses the public part of
Bob's key when they want to encrypt a message for Bob, and only Bob's private
key can decrypt a message in this way.\\

Digital signatures use asymmetric cryptography. Given that Alice wants to sign a
message so that Bob can verify that Alice is the sender, both proceed as follows
to fulfill their goal. First, Alice creates a pair of keys, a public and a
private one. Alice uses their private key to sign a message and distributes
their public key so others can use it to verify Alice's signature. To create
their signature, Alice first generates a pair of keys. Alice then published
their public key to make it available to Bob. When sending a message to Bob,
Alice uses a signing function that takes the message and the private key as
input and produces a signature, which Alice sends along the message to Bob.
Knowing Alice's public key, Bob can now check the validity of the signature with
the help of a signature validation function. This function takes the message,
the signature, and Alice's public key as input and generates the answer to
whether the message was signed with the signature.\\

Another problem arises when the network increases and Bob does not have the
chance to receive Alice's key in person. To prevent man-in-the-middle attacks,
Bob must ensure that the key they received over a network originates from Alice
and not from a malicious entity. The attack surface increases with the number of
different entities Bob wants to communicate with, as Bob needs to verify the
keys of each of their communication partner. A third party, the \gls{ca}, that
Bob trusts, can help with this. The \gls{ca}'s task is to pledge to Bob that a
certified key belongs to Alice or any other chosen entity and not a potentially
malicious third party. Bob trusts the \gls{ca} to do the keys' background check
and only deliver certificates of trusted identities. Alice first registers their
key with the \gls{ca}. The \gls{ca} own a pair of private and public keys too.
It uses these keys to cryptographically bind Alice's public key to their
identity. With this certificate, Bob can validate that a public key belongs to
Alice by validating the certificate with the \gls{ca}s public key. Alice now
delivers their public key and the certificate issued by the \gls{ca} to prove
that the key belongs to them. As long as the \gls{ca} is not malicious, Bob can
detect man-in-the-middle attacks if any malicious entity ships a key not
belonging to Alice. While all participants can create certificates for their key
with a dedicated key pair, such self-signed certificates could allow
man-in-the-middle attacks again. Hence, a trusted third party creates the
certificates.\\

\begin{figure}
    \begin{center}
        \includegraphics[width=.6\textwidth]{images/chain_of_trust_placeholder.png}
        \caption{Certifications in a chain of trust}
        \label{fig:state:technical:chain_of_trust}
    \end{center}
\end{figure}
\todo{This is a placeholder}

Because the single \gls{ca} is trusted by all participants in the network,
compromising it would mean control over all the communication in the network.
The solution to this is distributed trust. Multiple \gls{ca}'s can exist in a
network and compete with each other on the reputation gained from
users.\cite{perlman1999overview} Such a \gls{pki} is used in many network
technologies, such as HTTPS, that use the X.509 standard. Next to leaf
certificates issued to Alice and Bob, a \gls{ca} can also issue certificates on
keys that allow other entities to issue certificates and act as an \gls{ica},
creating a chain of certificates.
Figure~\ref{fig:state:technical:chain_of_trust} shows such a chain of
certificates. If Bob now wants to establish trust in the certificate provided by
Alice, they must first decide if they trust the \gls{ica}. The first step is to
verify that Alice's certificate was issued by the \gls{ica}. In the next step,
Bob needs to check all succeeding certificates until they reach the root
\gls{ca} by validating the \gls{ica}s respective certificates with the public
key of their issuer. In this way, Bob builds a chain of trust by verifying the
validity of the chain of certificates and deciding to trust the root \gls{ca}.
In the following section, we will see that such trust chains allow the
implementation of software attestation schemes.

\subsection{Remote attestation}
\label{sec:20:remote_attestation}
Remote attestation proves claims about a target
by delivering evidence to an appraiser over a network that supports these
claims. Before I further proceed to explain remote attestation, I want to
define the following terms similar to Coker et. al.\cite{coker_principles_2011}
\begin{itemize}
    \item \textbf{Appraiser}: Member of a network. Makes decisions about other
          parties on the base of delivered evidence.
    \item \textbf{Target}: Member of a network. Party about which properties the
          appraiser makes decisions.
    \item \textbf{Attestation}: Action of making claims about the target and
          delivering supportive evidence.
    \item \textbf{Measurement}: Collect evidence through direct and local
          observation
    \item \textbf{Attestation protocol}: Cryptographic protocol transmitting
          evidence about the claim. Trusted by appraiser
\end{itemize}

The appraiser and the target both follow orthogonal goals. While the appraiser
ideally wants to get as much information as possible about the target, the
target wants to preserve its privacy as well as possible and, therefore, conceal
as much information as possible. Both goals can be realized by employing a
trusted third party local to the target who can measure the target. This
third-party party then sends a signed report of the evidence on behalf of the
target. This report can contain the full raw measurement result, a reduced
variant, or a complete measurement substitute. For example, the reduced
measurement could be from a hash of the raw data. The substitute could be a
signature associated with the trusted third party that verifies that the target
fulfills the claims.

\subsection{Hardware Root of Trust}
\label{sec:20:hardware_root_of_trust}
A root of trust is defined by the Trusted Computing Group as follows:
\begin{quote}
    \textit{ A minimal set of system elements that have to be trusted because
        misbehavior is not detectable. \\
    } \mbox{ -- Trusted Computing Group\cite{tpm_architecture}}
\end{quote}

The Trusted Computing Group specifies that a hardware root of trust must be
available to enable remote attestation in a confidential computing
environment.\cite{tpm_architecture} A hardware root of trust is a device in a
computer system that the system can not manipulate. Moreover, it implements
security functionality such as encryption and random number generation. While
misbehavior is impossible to detect, hardware manufacturers can verify that
their devices work as intended by providing certificates. These certificates can
be embedded into the device with the help of tamper-resistant memory, such as
ROM or eFuses. A user can then check the validity of a certificate by consulting
the respective manufacturer's service. \\

Hardware roots of trust are necessary because system software could tamper with
software or memory to manipulate a possible software solution. The following
sections we review the most widely spread solutions to the hardware root of
trust. These implementations rely on dedicated hardware modules such as add-in
cards or unique, secure operation modes in CPU hardware. \\

\section{Trusted Execution Environments}
\label{sec:state:tee}

GlobalPlatform first used the term trusted execution environment to define a
solution for mobile trusted computing solutions.\cite{globaltee} Since then,
many definitions inconsistent and unspecific definitions have been published for
TEEs until the first precise definition was proposed by Sabt
et.al.\cite{sabt2015trusted}. They define the TEE in the following way:
\begin{quote}
    \textit{Trusted Execution Environment (TEE) is a tamper-resistant processing
        environment that runs on a separation kernel. It guarantees the authenticity of
        the executed code, the integrity of the runtime states (e.g., CPU registers,
        memory, and sensitive I/O), and the confidentiality of its code, data, and
        runtime states are stored on persistent memory. In addition, it shall be able
        to provide a remote attestation that proves its trustworthiness for third
        parties. The content of TEE is not static; it can be securely updated. The TEE
        resists all software attacks as well as physical attacks performed on the
        system's main memory. Attacks performed by exploiting backdoor security flaws
        are not possible. \\
    } \mbox{ -- Sabt et al.\cite{sabt2015trusted}}
\end{quote}

The first half of the definition describes a secure execution environment. A
secure execution environment can protect the integrity, authenticity, and
confidentiality of the application hosted by the SEE. Malicious privileged
software is thus neither able to modify the code, tamper with the runtime state,
nor observe code and data through the runtime. Contrary to TEEs, a SEE cannot
prove these claims against an appraiser, lest a third party outside the system.
This is because it does not require a root of trust to present in the system. To
prove its trustworthiness, TEEs employ remote attestation, which is the second
important aspect of the definition.\\

Trusted execution environment consists of several building blocks. The first
building block Sabt et al. propose is a secure boot. This building block allows
TEEs are used to verify that only a specific code of a particular state is
loaded. For a secure boot, a chain of trust is formed by verifying each
component's state. To generalize the secure boot requirement, a TEE must be
capable of verifying what code it loads to the environment. The second building
block is secure scheduling. Task executing in the TEE should not be able to
disrupt the main OS. Moreover, a TEE should implement means to allow
communication between the insecure world outside the TEE and the application
executing inside of it. Secure storage is one more building block. It allows the
application in the TEE to store data in a confidentiality, integrity, and
freshness-conserving way. Trusted I/O paths secure the communication between a
TEE and its users.

\subsection{TPM}
\label{sec:20:tpm}
\todo{A lot of passive to fix here}
The \gls{tpm} is a low-cost cryptographical coprocessor that offers different
cryptographic functions, such as hash functions, asymmetric and symmetric
encryption and decryption functions, asymmetric signing and verification
functions, and key generation functions. Implementations of the \gls{tpm} follow
different versions of specifications created and managed by the \gls{tcg}
consortium.\cite{tpm_architecture} The \gls{tpm} is specified by the Trusted
Computing Group as a system component with a state separate from the host system
on which it reports. The host system cannot directly manipulate the state of the
\gls{tpm} but has to use a defined interface to interact with the \gls{tpm}. To
separate the state between the host system and the \gls{tpm}, the \gls{tpm} is
implemented using dedicated hardware, such as a processor, RAM, ROM, and Flash
memory, all physically protected from the host system. Other physical separation
means can be used to implement \gls{tpm} services, such as unique processor
modes with dedicated memory access rights.\\

The most recent version of the \gls{tpm} specification is 2.0. The first
widespread family of \gls{tpm}s followed specification version 1.2, which was
implemented on modules shipped with personal computers beginning from the year
2005.\cite{arthur2015practical} One major drawback of version 1.2 was the
hard-coded usage of SHA-1 as a hashing algorithm. SHA-1 was first broken in 2005
by Wang et al.\cite{wang2005collision}. In 2011, NIST deprecated SHA-1 because
of security concerns.\cite{nist-sha1} \gls{tpm} 1.2 is constrained with respect
to its data structures to use either RSA or SHA-1\cite{tpm_architecture}
Therefore, when designing \gls{tpm} 2.0, the \gls{tcg} decided to leave the
exact algorithms specific \gls{tpm}s support open for their implementation.
Instead, the specification mandates a \gls{tpm} of version 2.0 to implement at
least one symmetric hashing, one symmetric encryption, and one asymmetric
encryption and signing algorithm. \\

In x86 systems, \gls{tpm} 2.0 is widely spread today and one of Windows 11s
system requirements. Often, \gls{tpm} is not implemented as a dedicated hardware
module but as firmware \gls{tpm}. The firmware \gls{tpm} is part of the Intel
Platform Trust Technology (Intel PTT) on Intel platforms. AMD platforms use an
implementation called fTPM, which is integrated within the platform security
processor.\cite{pirker2024brief} \\

While manufacturing, a vendor burns a \gls{ek} together with a certificate in
the \gls{tpm}. The \gls{ek} can be used to identify the \gls{tpm}, and the
certificate can be used to prove that the \gls{tpm} is genuine with the help of
the vendor's public key. Moreover, the \gls{tpm} uses the \gls{ek} as an input
to its key derivation functions. With the help of these key derivation
functions, a \gls{tpm} can generate keys for other applications such as
attestation, hashing, and signing. It never uses the \gls{ek} directly to
prevent leaking it in any form. A built-in entropy collector enables the
\gls{tpm} to generate random numbers. The \gls{tpm} uses hash functions to
generate a digest of input. Next to its cryptographic properties, the advantage
of a hash function is the output's constant length independent of the input's
length. Thus, the buffer only needs to be large enough to hold the result of the
digest. Message digests are used to store data outside of the \gls{tpm} or
generate certificates of values. The \gls{tpm} uses HMACs when storing data
outside of it. HMACs allow verifying that this data was not altered and
originates from a certain entity with the signing key.\\

The \gls{tcg} specifies two roots of trust for a trusted system. Two of them can
be implemented by the \gls{tpm}. The first is the root of trust for storage. The
\gls{tpm} implements memory that is shielded from the rest of the system, and
that can only be accessed by the \gls{tpm}. The second root of trust is the root
of trust for reporting. With its signing ability, the \gls{tpm} can attest to
values stored in its memory and create certificates for a measurement chain. The
third root of trust is the root of trust for measurement. This root is formed by
the CPU that measures on behalf of the system software or firmware. The
specification describes a core root of trust for measurement as the first
instructions executed in a new chain of trust, typically the firmware. In other
words, the core root of trust is trusted software that is believed to perform
the first measurement of the system correctly.\\

Software that performs the measurements instructs the \gls{tpm} to store its
result in \gls{pcr}. \gls{pcr}s are special registers of the \gls{tpm} that only
allow the \texttt{extend} operation and only reset when resetting the system.
The \texttt{extend} operation takes the current value of the \gls{pcr} and a new
value as input, concatenates both values, and processes the result with the help
of a hash function to create a digest that reflects the current system state.
The \gls{tpm} can assist in creating a chain of trust with the help of
\gls{pcr}s. An example of how to use the \gls{tpm} to build up a chain of trust
is given by Arthur et al.\cite{arthur2015practical} The system software can
extend a certain \gls{pcr} for each state before transferring control to the
next application. The application can then check the respective \gls{pcr} to
contain a known good value, indicating the platform was in a known good state
before the application was launched. If so, it can continue operation. If not,
the application might terminate. The measurement can also contain data about the
application to be launched to make sure that the application's integrity is not
hurt. Remote parties can use the \gls{tpm} attestation function to determine
whether a system was in a known good state at some time. For this, the \gls{tpm}
offers the possibility to sign the content of one or more \gls{pcr}s, producing
a quote. The remote party receives the quote, the public key, and the message
contents. The remote party can validate the certificate by executing the
signature validation function. With the public key of the \gls{tpm}, a remote
party can verify that the \gls{tpm} is genuine, and the vendor pledged to do so.
\todo{Do we need information on how system software interacts with the TPM?}

\subsection{Intel SGX}
\label{sec:20:sgx}
\todo{A lot of passive to fix here}
Intel SGX is an extension in x86\_64 processors manufactured by Intel, that
allows the creation of trusted execution environments. Intel first shipped SGX
in 2015, with processors implementing the Skylake microarchitecture. While
server-grade CPUs are still implementing SGX, Intel marked SGX was deprecated in
2021 in consumer-grade CPUs, beginning from CPUs implementing the Rocket Lake
microarchitecture. Costan et al. did an extensive study of SGX in 2016 and
documented in detail how SGX works and what it's security properties
are.\cite{costan2016intel} \\

Features of SGX include the creation of enclaves. Enclaves are especially
access-protected and encrypted system memory regions, with SGX preventing direct
memory access. In the creation process, memory pages are added to the enclave
page cache (EPC) and assigned to the enclave. Once assigned to an enclave, SGX
protects the memory page from unprivileged access, which includes all access
attempts not originating from the memory-owning enclave. After the system
software adds all pages to the enclave, it is marked as initialized. For the
initialization process, system software uses privileged instructions. After the
enclave is marked initialized, no more pages can be added to the EPC, and
interaction with the enclave is only allowed by using dedicated instructions
available only in user space. The enclave code runs at the permission level of
the application from which the enclave was called. Intel equips each CPU with
unique cryptographic keys that the CPU uses to encrypt code and data placed in
the EPC. These keys reside in memory made of electronic fuses that can not be
reprogrammed. Intel programs this memory in the factory process by burning some
of the fuses. Applications using SGX services do not necessarily need to run as
a whole in an SGX enclave. Because of restrictions on the size of the enclave's
memory, only parts of the data were handed to enclaves. Again, communications
between the enclave and user applications use special CPU instructions. In cases
where applications are split into parts residing in and outside of the enclave,
an application might want to verify the identity before sharing secrets. For
this, SGX implements local attestation.\\

As mentioned, SGX implements processor instructions dedicated to managing and
interacting with enclaves. Furthermore, SGX implementations create at least a
Launch Enclave signed by Intel. Third party enclaves that are not signed by
Intel require the Launch Enclave for successful initialization. It is necessary
in all cases when SGX is used.\\

To implement software attestation, SGX uses a second enclave provided by Intel,
the Quoting Enclave. The quoting enclave is used to verify the state
of an enclave to verify. This is done by generating a report structure. This
structure contains data such as the version and launch state of the enclave and
its identity. It is generated by calling the dedicated \textbf{EREPORT}
instruction, which cryptographically binds the generated report structure to the
enclave. The generated report structure is handed to the quoting enclave for
remote attestation. The quoting enclave then uses private keys to sign the
report to attest that the report was indeed generated from the enclave in
question. A remote party can check the quoting enclave's signature to verify the
enclave's state and identity.\\

\subsection{Confidential VM Extensions}
\label{section:20:confidential_vms}
The goal of confidential Virtual machines is to protect the entire VM from the
influence of a malicious hypervisor or other privileged software. Intel and
AMD offer individual ISA extensions for their processors to host confidential
Virtual Machines. Intel calls its solution Intel TDX, while AMDs solution is
called AMD SEV-SNP.\cite{tdx_whitepaper,kaplan_amd_2020} Both solutions use the
same fundamental building blocks to achieve the goals of confidential VMs.
Misano et al. did a extensive comparison of both
technologies.\cite{misono_confidential_2024} Intel uses the SGX module for its
implementation. Additionally, to interact with a confidential VM, the CPU must
be in the dedicated CPU operation mode called SEAM mode. Memory access is only
allowed in SEAM mode to protect confidential VMs. Once in SEAM mode, the CPU
uses its VMX capabilities to host and interact with the VM. For cryptographical
features, such as signing and key generation, Intel processors utilize the
Intel Management Engine. The Intel management Engine is a coprocessor located
in the CPU package with special firmware and a separate OS that is isolated from
the remaining parts of the system\\

For SEV-SNP, AMD uses the already implemented SEV capabilities. Unlike Intel's
implementation, AMD processors do not utilize a dedicated CPU mode but extend
the existing VM control structure by fields to enable Secure Nested paging. For
cryptography, the integrated AMD Platform Security Processor, short PSP, is
used. Both solutions encrypt the VM's memory to protect the VM from being
manipulated by system software. While in Intel's implementation, each VM is
encrypted separately, AMD's implementation encrypts, once activated, the whole
memory.\\

Both solutions use the trustee's knowledge of the initial state of the VM image.
The assumption that the approach follows is that if the VM is started in a known
state and protected from manipulation by the hypervisor or other privileged
software, then the VM can be trusted in the following. To follow this approach,
a measurement of the initial VM image is created and cryptographically bound to
the respective VM instance through a message authentication code. Before the
trustee interacts with the VM, they request the VM to verify its identity. For
this, the VM requests the cryptographical hardware to sign a report. The signed
report is then handed to the trustee. The trustee trusts the implementation in
the CPU and verifies the signature of the CPU signed report. With this, the
trustee knows if the VM images were expected. In the following, both the trustee
and the VM exchange keys for further communication.\\

\subsection{ARM TrustZone}
\label{sec:20:trustzone}
As another widely spread ISA ARM dominates the mobile sector. Like x86, the ARM
architecture offers technology to allow isolated program execution. On ARM, this
technology is called ARM TrustZone. TrustZone is optional for ARM processor
implementations and slightly differs between the ARM Cortex-A application
processors and the ARM Cortex-M microcontroller-aimed processors. In the
following, we concentrate on the implementation of ARM application
processors. Pinto and Santos did an extensive survey of ARM TrustZone in 2019.
In their work, they describe technical properties of AMR TrustZone and how to
use it for the implementation of \gls{tee}s and hypervisors. Moreover they
explain technical details of Trustzone and review it's security properties
against other \gls{tee}s.\cite{pinto_demystifying_2019}\\

Conceptually, ARM TrustZone-enabled processors offer three processor operation
modes. The most privileged mode is the Secure Monitor mode or short SM mode. The
Secure Monitor mode is the mode in which the processor boots and the firmware
and Bootloader executes. The second most privileged mode is the Secure World.
This mode is intended to execute code isolated from the third and least
privileged mode, the Normal World. The Bootloader is responsible for installing
software intended to run in the Secure World. Isolation is achieved by hardware.
For example, some registers exist twice to allow fast context switches between
the Normal and Secure World. The TrustZone Address Space Controller can be used
to partition memory into regions only accessible from the Secure World and those
accessible by both worlds. Changing between worlds can be done synchronously
using the dedicated SMC (System Monitor Call) instruction or asynchronously due
to an interrupt. The SMC instruction also triggers an interrupt. These
interrupts are served by invoking the SM, which decides upon its configuration
if the interrupt received serves as an entry point to the secure world. If so,
the Secure Monitor invokes secure world code to serve the interrupt.\\

ARM TrustZone does not implement remote attestation in hardware. Such
functionality has to be implemented by the code running in the Secure World. TEE
and remote attestation functionality can be implemented by a bare metal
application or by using a trusted OS to host secure applications in the Secure
World. The first solution minimizes code size, while the second offers the
ability to host multiple applications in the Secure World. The trusted OS would
be responsible for isolating services running in the Secure World against each
other because applications running in the Secure World are not isolated from
other applications in the Secure World by hardware. Using a trusted OS brings
the downside of an increased trusted computing base compared to a bare metal
trusted application. TrustZone does not encrypt the memory of the secure
world.\\

\subsection{Security of Hardware Solutions}
All hardware solutions isolate critical functions to protect them from being
tampered with by privileged software. Because the \gls{tpm} protects only its
state, I do not consider it when comparing the other hardware solutions. All x86
solutions protect against adversaries that can tap the memory bus. These attacks
become infeasible because all three solutions encrypt the memory content of the
respective enclaves or VMs. An adversary must break the respective cryptographic
algorithms to inspect the memory content, as the processor decrypts the memory
only once loaded. While an advisory cannot read the memory content, no solutions
protect memory access patterns from recording. ARM TrustZone does not encrypt
memory, so tapping the memory bus is possible. All solutions are vulnerable to
side-channel attacks. Concerning the trusted computing base, ARM TrustZone
relies conceptually on the functionally largest software stack. All x86
solutions only rely on their respective implementation in the CPU SoC.
Privileged software or firmware running in SMM cannot access the memory of
enclaves or VMs. On the contrary, ARM TrustZone relies on the Bootloader or
firmware to install applications into the secure world. The processor's
implementation does not protect the secure world application from being
manipulated by the firmware. We could argue that ARM TrustZone's TCB is larger
than that of the x86 solutions. As a side note, the x86 is a closed source but
could be considered rather complex. The respective security processors on the
respective x86 SoCs are running a small operating system themselves, making it
hard to compare the TCB. Nevertheless, in the x86, the whole SoC implementation
is manufactured by the same vendor, which the user ultimately has to trust.
Contrary to this, in the ARM world, the user would have to trust the SoC
manufacturer and the vendors of the Bootloader, firmware, and secure world
application.\\

\section{Side Channel Attacks}
\label{sec:20:attacks}
In this section I want to highlight some attacks on \gls{tee} implementations.
This section does not claim to be a review of all attacks out there targeting
\glspl{tee}. Instead, I want to give a short explanation of attacks that I try
to mitigate with my prototype implementation (c.f chapter 3 \todo{ref auf design}).

\subsection{Transient Execution Attacks}
\label{sec:20:transientattacks}
In 2018, researchers published the Spectre and Meltdown
attacks.\cite{kocher_spectre_2020, lipp_meltdown_2020} These attacks were the
first to exploit the side effects of transient execution in modern CPUs and
affected all commodity architectures. For example, CPUs of the vendors AMD,
Intel, Qualcomm, and other ARM designs were affected. This class of new
transient execution side-channel attacks abuses the speculative execution
feature of modern CPU and defines an entirely new class of attacks. Furthermore,
they are the first class of attacks that abuse microarchitectural bugs. We
review this class of attacks in more detail as I aim to implement a TEE that can
defend against such an attack.\\

Both attacks abuse a race condition between transient executed instructions and
memory access (Spectre) or exception delivery (Meltdown). In strictly sequential
instruction processing, each instruction would only be executed if the results
of the one before arrived. With speculative execution, the CPU executes
instructions before the results of previous instructions arrive. With branching
or exception delivery, this can lead to a race condition in which instructions
are executed before the branching decision is committed or an exception arrives.
Both attacks aim to make the window in which instructions are executed
speculatively as large as possible to observe the microarchitectural side
effects later. With carefully crafted code, these race conditions can be used to
access arbitrary memory (Meltdown) or memory readable by the attacked process
(Spectre). \\

Both attacks have in common the fact that they transmit data through a covert
channel. In the original attack descriptions, both attacks use a cache covert
channel to which they transfer data with the help of transient executed
instructions. The class of transient execution attacks is still highly relevant
today, with at At least five attacks have been published since 2023.
\cite{ormandy2023zenbleed,trujillo2023inception, moghimi2023downfall,ragab_ghostrace_2024, wilke2024tdxdown}
TEE solutions are affected, too, because these attacks enable attackers to read
arbitrary memory. The problem persists, and no solution exists to mitigate
transient execution attacks in general.\\

\subsection{Interrupt Based Attacks}
\label{sec:20:interrupt_sca}
In section~\ref{sec:20:transientattacks}, we have seen that data can be leaked
from various \gls{tee} implementations over side channels if an attacker can
interrupt a \gls{tee} in a high temporal resolution. With SGX-step Van et al.
presented the first framework with which an attacker can mount attacks on
\gls{sgx} with instruction-level granularity.\cite{van2017sgx} Next to the
aforementioned covered side channels, this allows the attacker to learn access
patterns of the \gls{tee} and tightly monitor its control flow. Wilke et al. and
Kou et al. implemented similar attacks for AMD and ARM
processors.\cite{wilke2023sev, kou2021load}\\

On x86 processors, the attack scheme uses the \gls{lapic}, while on ARM, it uses
a separate core to send inter-processor interrupts to overcome the boundaries
set by the two worlds in TrustZone. While Kou et al. use these interrupts to
mount high-precision flush+evict attacks, both x86 attacks additionally describe
how to monitor \gls{tee} behavior by manipulating the page tables of the
\gls{tee}. By setting the accessed bit to 0, the attacker can at least learn of
memory accesses on page granularity. All three attacks are, therefore, able to
break the security guarantees of the \gls{tee} while not granting an attacker
more power than the attacker model of the respective solution does.

\section{Attack Mitigations and countermeasures}
\label{sec:20:mitigations}
In this section I want to list review mitigations and countermeasures for known
side channel attacks. These mitigations and countermeasures serve as examples
for my implementation and can be viewed as part of related work, as they show
how to either detect or prevent the usage of side channels and attacks that
utilize them.

\subsection{Isolation through SMM}
\label{sec:20:isolation_smm}
An early work on how to isolate processes was done by Azab et al. in
2011.\cite{azab_sice_2011} The work dates before introducing TEE extensions in
x86 hardware and uses the SMM to isolate tasks. The problem SICE tries to solve
is to protect the memory integrity of isolated tasks and virtual machines. In
principle, SICE uses the strong hardware-enforced isolation of the SMM and its
SMRAM to install applications into it. The authors used an AMD platform for
their practical implementation because AMD platforms allow the adjustment of
SMRAM size and location after the SMM code locks the SMRAM.\cite{bios2014amd} To
switch to the isolated task residing in SMRAM, the firmware SMI handler was
modified to transfer control to the management runtime of the isolation
environment. The strong hardware isolation guaranteed that even a malicious
operating system could not access the memory of the isolated task. A downside to
this approach is that it works only on a small amount of hardware. The
implementation depends on the resizeable SMRAM to react to the growing memory
demands of applications isolated through SICE. As the authors mentioned, they
used AMD platforms for their implementation because of this. Intel platforms did
not support this feature, and adapting SICE to those platforms was left as an
open problem. Moreover, firmware modifications have to be implemented by the
user to install the correct SMI handler. These modifications require the
firmware to be open source to implement the SMI handler. This requirement
further reduces the amount of hardware used for this approach. Moreover the
authors state that practical deployment would require CPU and hardware vendors
to do extensive security reviews on their firmware and SMM implementations.
Additionally, multiple attacks are known targeting the SMM breaking it's
isolation.\cite{wojtczuk2014attacking, wojtczuk2009attacking, wojtczuk2009poisining}

\subsection{Defenses Against Interrupt Based Side Channel Attacks}
\label{sec:20:interrupt_sca}
To defend against interrupt-based side-channel attacks, Cui et al. proposed a
defense solution in 2023 that they call QuanShield.\cite{cui_quanshield_2023}
QuanShield goal is to enable SGX enclaves to detect interrupt-based side-channel
attacks and react adequately. For this, QuanShield isolates a CPU core from the
system to let it run an Intel SGX enclave. The goal of the isolation is to
prevent the scheduler from interrupting the isolated core because no other
workload is to be executed by this core. The authors turned off all other
interrupts as far as possible. The authors used a Linux kernel that runs in
tickless mode to ensure that the kernel does not send scheduling interrupts to
the isolated core. The authors built tickless kernels by using the kernel
KConfig option \textit{CONFIG\_NO\_HZ\_FULL=y}. In this mode, the kernel does
not send scheduling interrupts to cores that are either idle or for which only
one task is ready.\cite{linuxtickless} All remaining interrupts are considered
to be attacks on the enclave. QuanShield uses unused parts of the state save
area to terminate the enclave upon receiving any interrupt. The state save area
is protected enclave memory in which the CPU stores its state on context
switches, e.g., when stopping to execute enclave code. QuanShield stores
non-canonical memory addresses in these unused parts. Once the control returns,
the enclave code uses one of these non-canonical addresses, which results in a
CPU exception and leads to the termination of the enclave, effectively stopping
the attack.\\

QuanShield uses code instrumentation to make the enclave use one of the
manipulated addresses. For this, the authors added code to the LLVM compiler.
The compiler introduces load and store operations on each function entry to make
the code fault as fast as possible. QuanShield uses a library OS to support
legacy applications. It implements the protection mechanism by utilizing SGX-LKL
to manage the second stack in the state save area. SGX-LKL is a Linux kernel
port that can run in an SGX enclave as a LibraryOS, similar to the approach used
by SCONE and Haven to mitigate effects of Iago attacks, which I just want to
name as a side node because these solutions are less relevant for this
work.\cite{priebe2019sgx,arnautov_scone_2016,baumann_shielding_2015,
    checkoway2013iago}

\subsection{Transient Execution Mitigations}
\label{sec:20:def_sca}
Because Spectre and Meltdown target microarchitectural behavior, a complete
redesign is necessary to entirely fix the issue. Software mitigations are
available for specific attacks. For example, the Linux Kernel uses techniques
called Retpoline and Kernel Page Table Isolation (KPTI) to mitigate Spectre
version 2 and Meltdown, respectively.\cite{retpoline} On the other hand, software mitigations
can greatly impact performance, reaching from a 10\% to 800\% overhead,
depending on the workload.\cite{low2018overview}\\

Another approach for systems to defend against side-channel attacks, in general,
is active detection of the attack and reacting appropriately. Quanshield
implements such a solution for interrupt-based side channel attacks described in
chapter~\ref{sec:20:interrupt_sca}. For attacks abusing transient execution,
this approach of deactivating transient execution would come with a
high-performance penalty.\\

Instead, existing solutions attempt to monitor the
cache and other microarchitectural behavior through hardware performance
counters to find any anomaly. The idea behind this approach is that a program
shows specific behavior. This behavior results from the instructions it executes
and their order, which leaves a kind of footprint. When monitoring the hardware
performance counters closely enough, the observer can deduce what parts of code
have been executed by the CPU. If the performance counter traces of the program
are known beforehand, an Observer can compare the values of the counters with
the known state and then reason if the control flow was highjacked, for example,
by an ROP attack. As a side note, monitoring through hardware Performance
counters can also be used for attacks, which is why performance counters are
unavailable while the CPU operates in SGX mode.
\cite{uhsadel2008exploiting,costan2016intel}

Like ROP attacks, transient execution attacks show special behavior when
preparing the attack or side channel. Li et al. and Van Bulck et al. examined
how to trace the behavior of transient execution attacks with performance
counters on the examples of Spectre and Load Value injection attacks,
respectivley.\cite{li_detecting_2021, van_bulck_lvi_2020}
They found that when in the preparation phase, while the attacker trains the
branch predictor, fewer instructions are retired compared to typical workloads.
Moreover, employing the cache side channel leaves traces, too. For a cache-based
side-channel to work, the attacker tries to evict pages that map to addresses
they want to use for the attack. This results in a high amount of TLB flushes.
Later addresses are accessed by the side channel code to retrieve information.
Because the cache was flushed, the number of cache misses increased
significantly. An observer can detect all of these side effects by using
performance counters. Still, an attacker can hide their activities by slowing
down their attack. While the total number of cache misses induced by the attack,
for example, does not change this way, the attack distributes the misses more
evenly over time. Because events such as cache misses and TLB flushes are normal
behavior of a running system, the environment in which an attack runs introduces
noise that can help hide an attack. Thus, detecting said attack becomes nearly
impossible if an attacker distributes the effects of their attack over time.
Consequently, the results of Li et al. and Van Bulck et al. lead us to conclude
that a detection approach using hardware performance counters in this way is
unreliable. Kosasih et al. came to a similar conclusion in their survey of
knowledge in 2024.\cite{kosasih2024sok}


\section{Summery}
Next to the introduction to the technical aspects of x86 processors we have seen
examples for \glspl{tee}, attacks on \glspl{tee} and possible mitigations to
those attacks. \\

All \gls{tee} solutions have in common, that they do not implement the whole
functionality exclusively in hardware. For example, Intel and AMD employ a
dedicated security processors in their x86 \gls{soc} to implement \gls{sgx} and
\gls{sev} features respectively. Together with Enma, which root of trust is the
\gls{tpm}, all of these solutions a signature based remote attestation scheme.
Next to this they all share the property of being vulnerable to side channel
attacks, with Spectre and Meltdown being rather prominent ones. These attacks
abuse observable (micro-) architectural side effects to leak secrets through
covert channels. For some attacks mitigation exist, but they either have a
impact on performance or can't be applied to all side channel attacks.
Additionally, systems can observe these side effects through \glspl{pmc} to some
degree, but as we have seen in section~\ref{sec:20:def_sca}, attacks can avoid
being detected through reducing their throughput or simply if the noise in the
system is to high. With QuanShield interrupt based side channels were mitigated
by Cui et al. by creating an interrupt free environment, which at least
mitigates this attack vector for \gls{sgx}.\\

Putting these together, I identified a approach not tested to the best of my
knowledge. This is to exclude architectural side channels by creating a
completely isolated execution environment, that only uses core local resources
and monitors its integrity by utilizing \glspl{pmc}.

\cleardoublepage

%%% Local Variables:
%%% TeX-master: "diplom"
%%% End:
