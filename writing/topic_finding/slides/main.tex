\documentclass{beamer}
%Information to be included in the title page:
\title{TEE on commodity x86 Hardware}
\author{Pascal Scholz}
\date{\today}

\begin{document}

\frame{\titlepage}

\begin{frame}
    \frametitle{Motivation hinter TEEs}
    Ziel: Die eigene Software auf Computern Dritter laufen lassen, ohne dass diese die Software manipulieren oder Geheimnisse von ihr lernen können. \\
    \bigskip
    Zwei Haupteinsatzgebiete:
    \begin{itemize}
        \item DRM: DRM Anbieter wollen verhindern, dass Inhalte entschlüsselt werden (privater Kontext)
        \item Cloud Computing: Cloudanbietende (mit erhöhten Rechten) sollen Workloads der Servicenutzenden nicht manipulieren/ausspionieren können
    \end{itemize}
\end{frame}

\begin{frame}
    \frametitle{Motivation: Probleme aktuller Techniken}
    \begin{itemize}
        \item Seitenkanalangriffe: Populär seit Spectre/Meltdown Angriffe (2017)
              \begin{itemize}
                  \item Umgeht Speicherschutz durch Missbrauch von speculative execution
                  \item Speicher wird gelesen, Seiteneffekte sind im Cache nachweisbar
              \end{itemize}
        \item TEE wie Intel SGX und ARM Trustzone sind angreifbar
              \begin{itemize}
                  \item BUSted (2023) nutzt geteilten Bus um Speicherzugriffszeiten zu protokollieren
                  \item Foreshadow (2018) lädt Daten in L1 bevor Spekulation abegrochen werden kann
                  \item AEPICleak (2022) leakt SGX Daten über DMA mit xAPIC
              \end{itemize}
    \end{itemize}
    $\rightarrow$ Nutzen des Speichersubsystems öffnet mögliche Einfallstore (Seitenkanäle)\\
    $\rightarrow$ Lücken in Hardware oft umständlich zu schließen
\end{frame}

\begin{frame}
    \frametitle{Hypothese}
    \begin{itemize}
        \item Es ist möglich eine TEE ohne spezielle Hardwareerweiterung in Software zu bauen
              \bigskip
        \item Es ist möglich eine TEE in Software zu bauen, die nicht von Seitenkanalattacken betroffen ist
    \end{itemize}
\end{frame}

\begin{frame}
    \frametitle{Idee: Maximale Isolierung ohne zusätzliche Hardware}
    \begin{itemize}
        \item Nur lokalen Cache nutzen (L1 auf Intel, L2 auf AMD)
        \item Kein Hauptspeicherzugriff während zu schützender Code läuft
        \item Kein Hyperthreading (Hyperthreaads teilen lokalen Chache, Gefahr von Zombieload)
        \item OS kann keinen anderen Code auf den Core schedulen (transient execution ausschließen)
        \item Performance counter um Leak zu erkennen
        \item Wird Leak erkannt, sofortiges Handeln (z.Bsp. Reset der Maschine)
    \end{itemize}
\end{frame}

\begin{frame}
    \frametitle{Nutzen der Implementierung}
    \begin{itemize}
        \item Zur Isolierung von Code (Task Isolation)
              \bigskip
        \item Als TEE
              \begin{itemize}
                  \item Offene Frage: Remote Attestation über Zustand der TEE
              \end{itemize}
    \end{itemize}
\end{frame}

\begin{frame}
    \frametitle{Angreifermodell}
    \begin{itemize}
        \item Schutz vor OS / anderen Anwendungen
              \begin{itemize}
                  \item OS und TEE laufen parallel, OS hat keinen Einfluss auf TEE Core
                  \item Kompromittierte TEE könnte OS beeinflussen
                  \item Alles was benötigt wird ist Core-exklusiv
              \end{itemize}
        \item Parallel laufender HV kann wie ein OS keinen Einfluss nehmen
        \item Kein Schutz innerhalb eines Hypervisors
              \begin{itemize}
                  \item Gast kann nicht sicherstellen, dass HV rdmsr nicht manipuliert
              \end{itemize}
    \end{itemize}
\end{frame}

\begin{frame}
    \frametitle{Forschungsfragen}
    \begin{itemize}
        \item Wie viel Code brauchen wir minimal dafür, wie viel Speicher bleibt für die Payload übrig?
        \item Gelingt eine solche Implementierung oder braucht es doch zusätzliche Hardware?
        \item Konzept für eine Kommunikation des OS mit TEE erdenken und Implementieren?
        \item Konzept für Remote Attestation entwickleln, existierende Algorithmen auswerten?
        \item Können wir mit (globalen) Performance Countern Remote Attestation machen?
        \item Kann/muss das TPM eingebunden werden als Hardware Root of Trust ?
    \end{itemize}
\end{frame}

\begin{frame}
    \frametitle{Was wird implementiert}
    \begin{itemize}
        \item Vor OS-Boot setzt trusted Software (Firmware, Bootloader) TEE auf
        \item Lädt danach OS/HV
        \item Core mit TEE hat eigenen Interrupthandler, keine Einflussnahme durch OS/HV möglich
        \item Kommunikation über Treiber/Kernelmodule zu definierten Zeitpunkten über definierten Speicherbereich
    \end{itemize}
\end{frame}

\begin{frame}
    \frametitle{Angedachter Umfang}
    \begin{itemize}
        \item Implementieren des im lokalem Cache laufenden Kernels
        \item Implementieren der Performance Counter Überwachung, sammt reset
        \item Implementierung um TEE und OS nebeneinander aufzusetzen
        \item Es soll des Konzept anhand eines PoC erforscht werden
              %\item Vielleicht in Rust (wird wohl viel Assembly)?
    \end{itemize}
\end{frame}

\begin{frame}
    \frametitle{Was erstmal nicht dazu gehört}
    \begin{itemize}
        \item Insbesondere sicherer Geheimnis-Austausch zwischen TEE und OS/Applikationen
              \begin{itemize}
                  \item Das ist eher als Future Work nach Auswertung der Remote Attestation Algorithmen zu verstehen
                  \item Eventuell (Public-)Key durch Software hinterlegen, die System aufsetzt
                  \item Kann hierzu das TPM verwendet werden
              \end{itemize}
        \item Der Software, die System, aufsetzt wird erstmal vertraut
    \end{itemize}
\end{frame}

\end{document}