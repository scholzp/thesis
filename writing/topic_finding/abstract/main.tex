\documentclass{article}

\title{Isolation of CPU cores for remote attestation in x86\_64 systems}
\author{Pascal Scholz}

\begin{document}
\maketitle

\section{Abstract}
Remote attestation is used in confidential computing to prove to a client over a computer network that a system in
question fulfills a set of defined properties. In a computer system, software can use remote attestation to verify
that, for example, the bootflow of a system was not compromised by malware. One requirement for remote attestation
is a hardware device that forms the root of trust\cite{coker2011principles}. For the bootflow example, the hardware
root of trust in x86 systems is given by a Trusted Platform Module which, short TPM.

A technology enabling the concept of remote attestation to a per-application basis is Intel's Software Guard
Extensions (SGX). It allows applications to run code in a secure enclave, isolated from the operating system or
hypervisor. SGX protects the integrity of the code to run in an enclave by providing remote attestation about the state
of the enclave. An explication can decide on the provided attestation results if it wants to exchange secrets with the
enclave. Confidentiality of secrets exchanged with the enclave through is protected by cryptography when using
SGX. The root of trust is formed by the correctly implemented special SGX instructions.
Enclave keys are embedded in the firmware, protected by the Intel management engine.\cite{costan2016intel}

In recent years many flaws in the implementation of SGX were discovered by security researchers. With the advent
of side-channel attacks, such as Spectre, a new attack vector was discovered. While not part of the threat model
of SGX, special Spectre attacks can be used to leak enclave content.\cite{chen2019sgxpectre, van2018foreshadow}
Even without side channels, architectural vulnerabilities can be used to leak SGX content and even keys.\cite{borrello2022aepic}

All these vulnerabilities work because they expose data through the memory subsystem. To mitigate this attack vector, I
want to elaborate on the possibility of constructing an enclave-like partition that runs code entirely in local cache
structures. For this, I want to isolate a single CPU core so that it can not be tampered with from other system
components. Software running in this enclave shall be able to maintain its confidentiality by observing
performance counters, from which it can decide to react appropriately if it detects a data breach. Moreover, I want to
review algorithms for remote attestation applicable to the isolated CPU core, so that its state can be attested to
applications running outside of the enclave. Combining both results in an enclave
immune to side channel attacks running on commodity hardware without
the need for additional hardware support. With the additional support of a hardware root of trust like the TPM, this
concept can achieve security guarantees similar to Intel SGX.

\bibliographystyle{plain}
\bibliography{bib}

\end{document}